\chapter{Basic differential geometry} \label{s:bas}

\minitoc

\section{Introduction}

We recall in this appendix basic definitions and results of differential geometry
that are used in the main text.
The reader who has some knowledge of general relativity should be familiar with most of them.
It should be clear that this appendix is not intended to be a monograph on differential geometry.
In particular, contrary to the other parts of these notes, we state many results
without proofs,
referring the reader to classical textbooks on the topic
\cite{Lafon15,Lee13,Lee18,ONeil83,Berge03,ChoquDD77,Eschr11}, as well as
to the differential geometry sections of the general relativity textbooks
\cite{Choqu09,Strau13,Wald84}.

%%%%%%%%%%%%%%%%%%%%%%%%%%%%%%%%%%%%%%%%%%%%%%%%%%%%%%%%%%%%%%%%%%%%%%%%%%%%%%%

\section{Differentiable manifolds} \label{s:bas:manif}

\subsection{Notion of manifold} \label{s:bas:def_manif}

Given an integer $n\geq 1$, a \defin{manifold of dimension $n$}\index{manifold}\index{dimension of a manifold} is a topological space $\M$ obeying the following properties:
\begin{enumerate}
\item $\M$ is a \defin{separated space}\index{separated space} (also called \defin{Hausdorff space}\index{Hausdorff space}): any two distinct points of $\M$
admit disjoint open neighborhoods.
\item $\M$ has a \defin{countable base}\index{countable base}\footnote{In the language of topology, one says that $\M$ is a \emph{second-countable space}.}:
there exists a countable family
$(\mathscr{U}_k)_{k\in\mathbb{N}}$ of open sets of $\M$ such that any open set of $\M$ can be written as the union (possibly infinite) of some members of the above family.
\item Around each point of $\M$, there exists a neighborhood which is
homeomorphic to an open subset of $\R^n$.
\end{enumerate}
Property 1 excludes manifolds with ``forks'' and is very reasonable from a physical point of view: it allows to distinguish between two points even after a small perturbation.
Property~2 excludes ``too large'' manifolds; in particular it permits setting
up the theory of integration on manifolds. It also
allows for a smooth manifold of dimension $n$ to be embedded smoothly into the Euclidean space $\R^{2n}$
(Whitney theorem\index{Whitney theorem}).
Property~3 expresses the essence of a manifold: it means that, locally, one can label the points of $\M$ in a
continuous way by $n$ real numbers $(x^\alpha)_{\alpha\in\{0,\ldots,n-1\}}$,
which are called \defin{coordinates}\index{coordinate} (cf. Fig.~\ref{f:bas:manifold}).
More precisely, given an open subset $\mathscr{U}\subset\M$, a
\defin{coordinate system}\index{coordinate!system} or \defin{chart}\index{chart}
on $\mathscr{U}$ is a homeomorphism\footnote{Let us recall that a  \defin{homeomorphism}\index{homeomorphism} between two topological spaces
(here $\mathscr{U}$ and $\mathscr{V}$) is a bijective map $\Phi$ such
that both $\Phi$ and $\Phi^{-1}$ are continuous.}
\be
    \begin{array}{rccl}
    \Phi: & \mathscr{U}\subset \M & \longrightarrow &
                \mathscr{V}\subset\R^n \\
        & p & \longmapsto & (x^0, \ldots, x^{n-1}) ,
    \end{array}
\ee
where $\mathscr{V}$ is an open subset of $\R^n$.

\begin{figure}
\centerline{\includegraphics[width=0.8\textwidth]{bas_manifold.pdf}}
\caption[]{\label{f:bas:manifold} \footnotesize
Locally a manifold resembles $\R^n$ ($n=2$ on the figure), but not necessarily at the global level.}
\end{figure}

\begin{remark}
In relativity, it is customary to label the $n$ coordinates by an index
ranging from $0$ to $n-1$. Actually, this convention is mostly used when $\M$ is the spacetime manifold ($n=4$ in standard general relativity). The computer-oriented reader will have noticed the similarity
with the index ranging of arrays in the C/C++ or Python programming languages.
\end{remark}


\begin{remark} \label{r:bas:topol_manif}
Strictly speaking the definition given above is that of a \defin{topological manifold}\index{topological!manifold}\index{manifold!topological --}. We are saying \emph{manifold} for short.
\end{remark}


Usually, one needs more than one coordinate system to cover $\M$.
An \defin{atlas}\index{atlas} on $\M$ is a set of pairs
$(\mathscr{U}_i,\Phi_i)_{i\in I}$,  where $I$ is a set (not necessarily finite), $\mathscr{U}_i$ an open set of $\M$ and $\Phi_i$ a chart on $\mathscr{U}_i$,
such that the union of all $\mathscr{U}_i$ covers $\M$:
\be
    \bigcup_{i\in I} \mathscr{U}_i = \M.
\ee

The above definition of a manifold lies at the \emph{topological} level
(cf.~Remark~\ref{r:bas:topol_manif}), meaning that one has the notion of continuity, but not of differentiability. To get the latter, one should rely on the smooth structure of $\R^n$, via the atlases:
a \defin{smooth manifold}\index{smooth!manifold}\index{manifold!smooth --},
is a manifold $\M$ equipped with an atlas
$(\mathscr{U}_i,\Phi_i)_{i\in I}$ such that for any non-empty intersection
$\mathscr{U}_i \cap \mathscr{U}_j$, the map
\be \label{e:bas:transition_map}
    \Phi_i \circ \Phi_j^{-1} : \Phi_j(\mathscr{U}_i \cap \mathscr{U}_j)
    \subset \R^n \longrightarrow \Phi_i(\mathscr{U}_i \cap \mathscr{U}_j)
    \subset \R^n
\ee
is smooth (i.e. $C^\infty$).
Note that the above map is from an open set of $\R^n$ to an open set of $\R^n$, so that the
invoked $C^\infty$ differentiability is nothing but that of $\R^n$.
Such a map is called a \defin{change of coordinates}\index{change!of coordinates}\index{coordinate!change} or, in the mathematical literature, a
\defin{transition map}\index{transition map}.
The atlas $(\mathscr{U}_i,\Phi_i)_{i\in I}$  is called a
\defin{smooth atlas}\index{smooth!atlas}\index{atlas!smooth --}.
In the following, we consider only smooth manifolds.

\begin{remark}
Strictly speaking a smooth manifold is a pair $(\M,\mathcal{A})$  where
$\mathcal{A}$ is a (maximal) smooth atlas on $\M$.
Indeed, a given (topological) manifold $\M$
can have non-equivalent differentiable structures, as shown by Milnor (1956) \cite{Milno56}
for $\mathbb{S}^7$, the unit sphere of dimension~7: there exist smooth manifolds, the so-called \emph{exotic spheres}\index{exotic!sphere},
that are homeomorphic to $\mathbb{S}^7$ but not diffeomorphic
to $\mathbb{S}^7$.  On the other side, for $n\leq 6$, there is a unique smooth
structure for the sphere $\mathbb{S}^n$.
Moreover, any manifold of dimension $n\leq 3$ admits a unique smooth structure.
Amazingly, in the case of $\R^n$, there exists a unique smooth structure (the standard one) for any $n\not=4$, but for $n=4$ (the standard spacetime case!) there exist uncountably many non-equivalent smooth structures, the so-called
\emph{exotic $\R^4$}\index{exotic!$\R^4$} \cite{Taube87}.
\end{remark}

\begin{remark} \label{r:bas:analytic}
When discussing the no-hair theorem in Chap.~\ref{s:sta}, one
refers to the concept of
\emph{analytic manifold},
which is a special case of smooth manifold. Indeed, an
\defin{analytic manifold}\index{analytic!manifold}\index{manifold!analytic --}
is defined as a manifold equipped with an atlas for which all the changes of coordinates
$\Phi_i \circ \Phi_j^{-1}$ are real analytic functions.
Let us recall that an \defin{analytic function}\index{analytic!function}
is a $C^\infty$ function $f$ for
which the Taylor series about any point $x$ in its domain converges to $f$
in some neighborhood of $x$.
\end{remark}

Given two smooth manifolds, $\M$ and $\M'$, of
respective dimensions $n$ and $n'$, a map
$\phi : \M \rightarrow \M'$ is called a \defin{smooth map}\index{smooth!map} iff
for every $p\in\M$, there exist a chart $(\mathscr{U},\Phi)$ around $p$ in the smooth atlas of $\M$
and a chart $(\mathscr{U}',\Phi')$ around $\phi(p)$ in the smooth atlas of $\M'$ such that
$\phi(\mathscr{U}) \subset \mathscr{U}'$ and
$\Phi'\circ \phi \circ \Phi^{-1}$ is a smooth map
$\Phi(\mathscr{U}) \subset \R^n \to \R^{n'}$.
The map $\phi$ is said to be a \defin{diffeomorphism}\index{diffeomorphism} iff
it is bijective and both $\phi$ and $\phi^{-1}$ are smooth. This implies $n=n'$.


\subsection{Manifold with boundary} \label{s:bas:manif_boundary}

A (topological) \defin{manifold with boundary}\index{manifold!with boundary} $\M$ is defined in the same way
as a topological manifold, except that condition~3 in the definition given
at the beginning of Sec.~\ref{s:bas:def_manif} is replaced by
\begin{itemize}
\item[3'.] Around each point of $\M$, there exists a neighborhood which is
homeomorphic to an open subset\footnote{By \emph{open subset of} $\mathbb{H}^n$, it is meant a set $A\subset \mathbb{H}^n$ that is open with respect to the topology of $\mathbb{H}^n$; $A$ is then not necessarily open when considered
as a subset of $\R^n$ (for instance $A=\mathbb{H}^n$).}
of the closed half-space
\be
    \mathbb{H}^n := \left\{ (x^1,\ldots,x^n) \in \R^n,\quad x^n \geq 0 \right\} .
\ee
\end{itemize}
A point $p\in \M$ is said to be a \defin{boundary point} of $\M$ iff
there exists a homeomorphism $\Phi$ from an open neighborhood of $p$
to an open subset of $\mathbb{H}^n$ such that
$\Phi(p)\in \partial \mathbb{H}^n$, where
\be
    \partial\mathbb{H}^n := \left\{ (x^1,\ldots,x^n) \in \R^n,\quad x^n = 0 \right\} .
\ee
This definition is independent from the choice of $\Phi$ (cf. Theorem~1.37 of
Ref.~\cite{Lee13}).
The set of all boundary points of $\M$ is naturally called the
\defin{boundary of}\index{boundary!of a manifold} $\M$ and is denoted by
$\partial\M$.
\begin{remark}
\label{r:bas:manifold_boundary}
The boundary $\partial\M$ should not be confused with the \emph{topological boundary}\index{boundary!topological --}
of $\M$, i.e. the boundary of $\M$ as a topological space, which is
the closure of $\M$ minus the interior of $\M$; since
both sets coincide with $\M$, the topological boundary of $\M$
is obviously $\varnothing$.
\end{remark}

A \defin{smooth manifold with boundary}\index{smooth!manifold with boundary}\index{manifold!smooth -- with boundary} is a manifold with boundary endowed with a
smooth atlas, with the understanding that a transition map
\[
    \Phi_i \circ \Phi_j^{-1} : \Phi_j(\mathscr{U}_i \cap \mathscr{U}_j)
    \subset \mathbb{H}^n \longrightarrow \Phi_i(\mathscr{U}_i \cap \mathscr{U}_j)
    \subset \mathbb{H}^n
\]
is said to be \emph{smooth} iff
it can be extended around each point of its domain
(including the points of $\partial\mathbb{H}^n$) into a smooth map
from an open subset of $\R^n$ to $\R^n$.

\subsection{Curves and vectors on a manifold} \label{s:bas:vectors}

On a smooth manifold, vectors are defined as tangent vectors to a curve.
Given an interval $I\subset\R$, a
\defin{curve}\index{curve} is a subset $\Li\subset \M$ that is the image of a smooth map
$I \to  \M$:
\be
    \begin{array}{rccl}
    P: & I & \longrightarrow & \M \\
        & \lambda & \longmapsto & p = P(\lambda) \in \Li.
    \end{array}
\ee
Hence $\Li = P(I) := \{ P(\lambda) |\ \lambda\in I\}$. The function $P$ is called a
\defin{parametrization}\index{parametrization} of $\Li$ and the real
variable $\lambda$ is called a \defin{parameter along $\Li$}\index{parameter along a curve}. Given a coordinate system $(x^\alpha)$
in a neighborhood of a point $p\in\Li$, the parametrization $P$ is
defined by $n$ functions $X^\alpha : \ I \rightarrow \R$ such that
\be \label{e:bas:curve_param_equation}
  x^\alpha(P(\lambda)) = X^\alpha(\lambda) .
\ee

\begin{remark} \label{r:bas:curve_def}
In the literature, a curve is often defined
as a map $P:\ I \rightarrow  \M$ and not as
the image of $P$. According to that definition, different parametrizations
give birth to different curves.
\end{remark}

A \defin{scalar field}\index{scalar!field} on $\M$ is a function
$f:\ \M \rightarrow \R$. Unless specified, we shall always consider \emph{smooth} scalar fields,
i.e. smooth maps as defined in Sec.~\ref{s:bas:def_manif}. At a point $p=P(\lambda)\in\Li$, the \defin{tangent vector to $\Li$}\index{vector!tangent --}\index{tangent!vector} associated with the parametrization
$P$ is the operator $\w{v}$ which maps every scalar field $f$ defined around
$p$ to the real number
\be \label{e:bas:def_vector}
  \w{v}(f) := \left. \frac{\D f}{\D \lambda} \right|_{\Li} =
  \lim_{\veps\rightarrow 0} \frac{1}{\veps}
  \left[ f(P(\lambda+\veps)) - f(P(\lambda)) \right] .
\ee
Given a coordinate system $(x^\alpha)$ around some point $p\in\M$, there are
$n$ curves $\Li_\alpha$ through $p$ associated with $(x^\alpha)$ and called the
\defin{coordinate lines}\index{coordinate!line}\index{line!coordinate --}:
for each $\alpha\in\{0,\ldots,n-1\}$, $\Li_\alpha$ is defined as the curve through $p$ parameterized by $\lambda = x^\alpha$ and having constant coordinates
$x^\beta$ for all $\beta\not=\alpha$.
The tangent vector to $\Li_\alpha$ parameterized by $x^\alpha$ is
denoted $\wpar_\alpha$. From the definition (\ref{e:bas:def_vector}), its
action on a scalar field $f$ is
\[
  \wpar_\alpha(f) =
  \left. \frac{\D f}{\D x^\alpha} \right|_{\Li_\alpha}
  = \left. \frac{\D f}{\D x^\alpha} \right|_{x^\beta={\rm const}\atop \beta\not=\alpha} .
\]
Considering $f$ as a function of
the coordinates $(x^0,\ldots,x^{n-1})$ (whereas strictly speaking it is a function
of the points on $\M$) we recognize in the last term the partial derivative of
$f$ with respect to $x^\alpha$. Hence
\be \label{e:bas:wpar_partial}
 \encadre{ \wpar_\alpha(f) = \der{f}{x^\alpha} } .
\ee
Similarly, we may rewrite (\ref{e:bas:def_vector}) as
\[
\w{v}(f) = \lim_{\veps\rightarrow 0} \frac{1}{\veps}
  \left[ f(X^0(\lambda+\veps),\ldots,X^{n-1}(\lambda+\veps))
  - f(X^0(\lambda),\ldots,X^{n-1}(\lambda)) \right]
  = \der{f}{x^\alpha} \frac{\D X^\alpha}{\D \lambda} ,
\]
i.e., in view of Eq.~(\ref{e:bas:wpar_partial})
\be \label{e:bas:v_f_partial_f}
  \encadre{ \w{v}(f) = v^\alpha \, \wpar_\alpha(f) } ,
\ee
where
\be \label{e:bas:va_dXadlamb}
  v^\alpha := \frac{\D X^\alpha}{\D \lambda} , \qquad 0 \leq \alpha\leq n-1 .
\ee
In Eq.~(\ref{e:bas:v_f_partial_f}),
we are using the \defin{Einstein summation convention}\index{Einstein!summation convention}: a repeated index implies a summation over all the values taken by this index
(here, from $\alpha=0$ to $\alpha=n-1$).
The identity (\ref{e:bas:v_f_partial_f}) being valid for any scalar field $f$, we conclude that
\be \label{e:bas:v_va_wpar_a}
  \encadre{ \w{v} = v^\alpha \, \wpar_\alpha } .
\ee
Since every tangent vector to a curve at $p$ is expressible as (\ref{e:bas:v_va_wpar_a}), we conclude that
the set of all tangent vectors to a curve at $p$ is a vector space of dimension $n$ and that $(\wpar_\alpha)$ constitutes a basis of it. This vector space is
called the
\defin{tangent space to $\M$ at $p$}\index{tangent!space}\index{vector!tangent -- space} and is denoted $T_p\M$.
The elements of $T_p\M$ are simply called \defin{vectors at}\index{vector} $p$.
The basis $(\wpar_\alpha)$ is called the \defin{natural basis}\index{natural basis}\index{basis!natural} associated with
the coordinates $(x^\alpha)$ and the coefficients $v^\alpha$ in (\ref{e:bas:v_va_wpar_a}) are called the \defin{components of the vector $\w{v}$ with respect to the coordinates $(x^\alpha)$}\index{component!w.r.t. a coordinate system}.
The tangent vector space is represented at two different points in Fig.~\ref{f:bas:tang_space}.

\begin{figure}
\centerline{\includegraphics[width=0.8\textwidth]{bas_tang_space.pdf}}
\caption[]{\label{f:bas:tang_space} \footnotesize
The vectors at two points $p$ and $q$ on the
manifold $\M$ belong to two different vector spaces:
the tangent spaces $T_p\M$ and $T_q\M$.}
\end{figure}

Contrary to what happens for an affine space, one cannot define a vector connecting two distinct points $p$ and $q$ on a generic manifold, except if $p$ and $q$ are infinitesimally close to each other. Indeed, in the latter case, one defines
the \defin{infinitesimal displacement vector from $p$ to $q$}\index{infinitesimal! displacement vector}\index{vector!infinitesimal --} as the vector $\D\w{x}\in{{T_p\M}}$ whose action on a scalar field $f$ is
\be \label{e:bas:dell_f}
  \D\w{x}(f) := \left. \D f \right| _{p\rightarrow q} = f(q) - f(p) .
\ee
Since $p$ and $q$ are infinitesimally close, there is a unique (piece of) curve
$\Li$ going
from $p$ to $q$ and one has
\be \label{e:bas:dell_v_dlamb}
  \encadre{\D\w{x} = \w{v} \, \D\lambda },
\ee
where $\lambda$ is a parameter along $\Li$, $\w{v}$ the associated tangent
vector at $p$ and $\D\lambda$ the parameter
increment from $p$ to $q$: $p=P(\lambda)$ and $q=P(\lambda+\D\lambda)$.
The relation (\ref{e:bas:dell_v_dlamb}) follows immediately from the definitions
(\ref{e:bas:def_vector}) and (\ref{e:bas:dell_f}) for respectively $\w{v}$ and $\D\w{x}$.
Given a coordinate system, let $(x^\alpha)$ be the coordinates
of $p$ and $(x^\alpha+\D x^\alpha)$ those of $q$. Then from Eqs.~(\ref{e:bas:dell_f})
and (\ref{e:bas:wpar_partial}),
$\D\w{x}(f) =   \D f  = \dert{f}{x^\alpha} \, \D x^\alpha
  = \D x^\alpha \, \wpar_\alpha(f)$.
The scalar field $f$ being arbitrary, we conclude that
\be \label{e:bas:dell_dxa_wpar}
  \encadre{ \D\w{x} = \D x^\alpha \, \wpar_\alpha } .
\ee
In other words, the components of the infinitesimal displacement vector $\D\w{x}$ with respect
to the coordinates $(x^\alpha)$ are nothing but the infinitesimal coordinate
increments $\D x^\alpha$.

\subsection{Linear forms} \label{s:bas:linear_form}

A fundamental operation on vectors consists in mapping them to real numbers, and this in a linear way. More precisely, at each point $p\in\M$, one defines a \defin{linear form}\index{linear form}\index{form!linear --}
as a mapping\footnote{We are using the same bra-ket notation as in quantum mechanics to denote the action of a linear form on a vector.}
\be \label{e:bas:def_lin_form}
    \begin{array}{rccl}
    \w{\omega}: & T_p\M & \longrightarrow & \R \\
        & \w{v} & \longmapsto & \langle \w{\omega}, \w{v} \rangle
    \end{array}
\ee
that is linear:
$\langle\w{\omega},\lambda \w{v} + \w{u}\rangle =  \lambda \langle\w{\omega},\w{v}\rangle +  \langle\w{\omega},\w{u}\rangle$ for all $\w{u},\w{v}\in{T_p\M}$ and $\lambda\in\R$. The set of all linear forms at $p$ constitutes a $n$-dimensional vector
space, which is called the \defin{dual space of $T_p\M$}\index{dual!vector space} and denoted by ${T_p^*\M}$.
Given the natural basis $(\wpar_\alpha)$ of $T_p\M$ associated with some coordinates
$(x^\alpha)$, there is a unique basis of ${T_p^*\M}$, denoted by $(\dd x^\alpha)$, such that
\be \label{e:bas:dual_basis_nat}
  \encadre{ \langle \dd x^\alpha ,\wpar_\beta\rangle = \delta^\alpha_{\ \  \beta} } ,
\ee
where $\delta^\alpha_{\ \ \beta}$ is the \defin{Kronecker symbol}\index{Kronecker symbol} :
$\delta^\alpha_{\ \  \beta} = 1$ if $\alpha=\beta$ and $0$ otherwise.
The basis $(\dd x^\alpha)$ is called the \defin{dual basis}\index{dual!basis}\index{basis!dual} of the basis
$(\wpar_\alpha)$. The notation $\dd x^\alpha$ stems from the fact that if we apply
the linear form $\dd x^\alpha$ to the infinitesimal displacement vector
(\ref{e:bas:dell_dxa_wpar}), we get nothing but the number $\D x^\alpha$:
\be \label{e:bas:dxa_dxa}
    \langle\dd x^\alpha,\D\w{x} \rangle = \langle\dd x^\alpha , \D x^\beta \, \wpar_\beta
    \rangle
    = \D x^\beta \underbrace{\langle \dd x^\alpha , \wpar_\beta \rangle}_{\delta^\alpha_{\ \ \beta}}
    = \D x^\alpha .
\ee
\begin{remark}
Do not confuse the linear form $\dd x^\alpha$ with the infinitesimal increment
$\D x^\alpha$ of the coordinate $x^\alpha$.
\end{remark}

The dual basis can be used to expand any linear form $\w{\omega}$, thereby defining its
\defin{components $\omega_\alpha$ with respect to the coordinates $(x^\alpha)$}\index{component!of a linear form}:
\be \label{e:bas:def_comp_form}
  \w{\omega} = \omega_\alpha \, \dd x^\alpha .
\ee
In terms of components, the action of a linear form on a vector takes then a very simple form:
\be
  \encadre{ \langle\w{\omega},\w{v}\rangle  = \omega_\alpha v^\alpha }.
\ee
This follows immediately from (\ref{e:bas:def_comp_form}),
(\ref{e:bas:v_va_wpar_a}) and (\ref{e:bas:dual_basis_nat}).

A field of linear forms, i.e. a (smooth) map which associates to each point $p\in\M$
an element of the dual space $T_p^*\M$ is called a \defin{1-form}\index{1-form}.
Given a smooth scalar field $f$ on $\M$, there exists a 1-form canonically associated with it, called
the \defin{differential of $f$}\index{differential} and denoted $\dd f$ or $\wnab f$.
At each point $p\in\M$, $\dd f$ is the unique linear form which, once
applied to the infinitesimal displacement vector $\D\w{x}$ from $p$ to a
nearby point $q$, gives the change in $f$ between points $p$ and $q$:
\be \label{e:bas:df}
  \D f := f(q) - f(p) = \langle \dd f, \D\w{x} \rangle .
\ee
Since $\D f = \dert{f}{x^\alpha} \, \D x^\alpha$, Eq.~(\ref{e:bas:dxa_dxa}) implies that the components of the differential with respect to the dual basis are nothing but the partial derivatives of $f$ with respect to the coordinates $(x^\alpha)$ :
\be \label{e:bas:grad_f_der_f}
  \encadre{ \dd f = \der{f}{x^\alpha} \, \dd x^\alpha } .
\ee

\begin{remark}
In non-relativistic physics, the concept of \defin{gradient}\index{gradient}
of a scalar field is commonly used instead of the
differential, the former being a vector field and the latter a 1-form.
This is so because one associates implicitly a vector
$\vw{\omega}$ to any 1-form $\w{\omega}$ via the Euclidean scalar product
of $\R^3$: $\forall \overrightarrow{v}\in \R^3,\ \langle \w{\omega},\overrightarrow{v}\rangle = \vw{\omega}\cdot\overrightarrow{v}$.
Accordingly, formula (\ref{e:bas:df}) is rewritten as
$\D f = \vw{\nabla} f \cdot \D\w{x}$. But we should keep in mind
that, at the fundamental level, the key quantity is the differential 1-form
$\wnab f = \dd f$, since Eq.~(\ref{e:bas:df}) does not require any metric on the
manifold $\M$ to be
meaningful. On the contrary, the gradient $\vw{\nabla} f$ is a derived quantity, obtained
from the differential $\wnab f$ by metric duality.
\end{remark}

\begin{remark}
For a fixed value of $\alpha$, the coordinate $x^\alpha$ can be considered as a scalar field on $\M$.
If we apply (\ref{e:bas:grad_f_der_f}) to $f=x^\alpha$, we then get $\dd x^\alpha = \dd x^\alpha$.
Hence the dual basis to the natural basis $(\wpar_\alpha)$ is formed by the
differentials of the coordinates. This justifies the notation $\dd x^\alpha$ used for its elements.
\end{remark}

By combining Eqs.~(\ref{e:bas:grad_f_der_f}), (\ref{e:bas:v_va_wpar_a}) and (\ref{e:bas:dual_basis_nat}), it is
easy to see that the 1-form $\dd f$ acting on a vector $\w{v}$ is nothing but $\w{v}$ acting on the scalar field
$f$:
\be \label{e:bas:df_v_eq_v_f}
    \encadre{ \langle\dd f, \w{v} \rangle = \w{v}(f) }.
\ee

Natural bases are of course not the only bases in the vector
space $T_p\M$. One may use a basis $(\w{e}_\alpha)$ that is not related to any coordinate system on $\M$, for instance an orthonormal basis with respect to some metric.
There exists then a unique basis $(\w{e}^\alpha)$
of the dual space ${T_p^*\M}$ such that\footnote{Notice that,
according to the standard usage, the symbol for the vector $\w{e}_\alpha$ and that for the linear form $\w{e}^\alpha$ differ only by the position of the index $\alpha$.}
\be \label{e:bas:dual_basis}
  \encadre{ \langle \w{e}^\alpha , \w{e}_\beta \rangle = \delta^\alpha_{\ \ \beta} } .
\ee
$(\w{e}^\alpha)$ is called the \defin{dual basis}\index{dual!basis}\index{basis!dual} to
$(\w{e}_\alpha)$.
The relation (\ref{e:bas:dual_basis_nat}) is a special case
of (\ref{e:bas:dual_basis}), for which $\w{e}_\alpha = \wpar_\alpha$ and $\w{e}^\alpha = \dd x^\alpha$.


\subsection{Tensors} \label{s:bas:tensors}

Tensors are generalizations of both vectors and linear forms.
Let $(k,\ell)\in\mathbb{N}^2$ with $(k,\ell)\neq (0,0)$.
At a point $p\in\M$,
a \defin{tensor of type}\index{tensor} $(k,\ell)$, also called \defin{tensor $k$ times
contravariant and $\ell$ times covariant}\index{contravariant}\index{covariant}\index{type of a tensor}, is a
map
\be \label{e:bas:def_tensor}
    \begin{array}{rccc}
    \w{T}: & \underbrace{{T_p^*\M}\times\cdots\times{T_p^*\M}}_{k {\ \rm times}}
    \times \underbrace{T_p\M\times\cdots\times{T_p\M}}_{\ell {\ \rm times}} &
     \longrightarrow & \R  \\
    & (\w{\omega}_1,\ldots,\w{\omega}_k,\w{v}_1,\ldots,\w{v}_\ell) &
         \longmapsto &
    \w{T}(\w{\omega}_1,\ldots,\w{\omega}_k, \w{v}_1,\ldots,\w{v}_\ell)
    \end{array}
\ee
that is linear with respect to each of its arguments. The integer $k+\ell$ is
called the tensor \defin{valence}\index{valence}, or sometimes the
tensor \defin{rank}\index{rank of a tensor} or \defin{order}\index{order of a tensor}. Let us recall the canonical duality
$T_p^{**}\M={T_p\M}$, which means that every vector $\w{v}$ can be considered
as a linear form on the space ${T_p^*\M}$, via
${T_p^*\M}\rightarrow \R$,
$\w{\omega}\mapsto \langle \w{\omega},\w{v}\rangle$.
Accordingly a vector is a tensor of type $(1,0)$. A linear form is a
tensor of type $(0,1)$. A tensor of type $(0,2)$ is called a
\defin{bilinear form}\index{bilinear form}\index{form!bilinear --}. It maps pairs of vectors to real numbers, in a linear way for each
vector.

Given a basis $(\w{e}_\alpha)$ of $T_p\M$
and the corresponding dual basis $(\w{e}^\alpha)$ in ${T_p^*\M}$,
any tensor $\w{T}$ of type $(k,\ell)$ can be expanded as
\be \label{e:bas:def_components}
    \encadre{\w{T} = T^{\alpha_1\ldots\alpha_k}_{\qquad\ \; \beta_1\ldots\beta_\ell}
        \; \w{e}_{\alpha_1} \otimes \ldots \otimes \w{e}_{\alpha_k}
                \otimes
        \w{e}^{\beta_1} \otimes \ldots \otimes \w{e}^{\beta_\ell} } ,
\ee
where the \defin{tensor product}\index{tensor!product}\index{product!tensor --} $ \w{e}_{\alpha_1} \otimes \ldots \otimes \w{e}_{\alpha_k} \otimes
\w{e}^{\beta_1} \otimes \ldots \otimes \w{e}^{\beta_\ell}$ is the tensor of
type $(k,\ell)$ for which the image of  $(\w{\omega}_1,\ldots,\w{\omega}_k,\w{v}_1,\ldots,\w{v}_\ell)$ as in
(\ref{e:bas:def_tensor}) is the real number
\[
    \prod_{i=1}^k \langle \w{\omega}_i,\w{e}_{\alpha_i}\rangle \;\times\;
    \prod_{j=1}^\ell \langle \w{e}^{\beta_j},\w{v}_j\rangle .
\]
Notice that all the products in the above formula are simply products in $\R$.
The $n^{k+\ell}$ real coefficients  $T^{\alpha_1\ldots\alpha_k}_{\qquad\ \; \beta_1\ldots\beta_\ell}$ in (\ref{e:bas:def_components}) are called the \defin{components
of the tensor $\w{T}$ with respect to the basis $(\w{e}_\alpha)$}\index{component!of a tensor}.
These components are unique and fully characterize
the tensor $\w{T}$.

\begin{remark}
The notations $v^\alpha$ and $\omega_\alpha$ already introduced for the components
of a vector $\w{v}$ [Eq.~(\ref{e:bas:v_va_wpar_a})]
or a linear form $\w{\omega}$ [Eq.~(\ref{e:bas:def_comp_form})] are of course the
particular cases $(k,\ell)=(1,0)$ or $(k,\ell)=(0,1)$ of (\ref{e:bas:def_components}), with,
in addition, $\w{e}_\alpha=\wpar_\alpha$ and $\w{e}^\alpha = \dd x^\alpha$.
\end{remark}


\subsection{Tensor fields on a manifold} \label{s:bas:fields}

A \defin{tensor field of type}\index{tensor!field}\index{field!tensor --} $(k,\ell)$ is a map $\w{T}$
which associates to each point $p\in\M$ a tensor of type $(k,\ell)$ on $T_p\M$.
A \defin{vector field}\index{vector!field}\index{field!vector --} is naturally a tensor field
of type $(1,0)$.
By convention, a scalar field\index{scalar!field}\index{field!scalar --} is considered as a tensor field of type $(0,0)$.
We shall consider only smooth fields. We shall denote by $\left. \w{T} \right| _p$ the tensor
representing the value of the tensor field $\w{T}$ at a point $p\in\M$.

Given a non-negative integer $p$, a
\defin{differential form of degree $p$}\index{form!differential --}\index{differential!form},
or
\defin{$p$-form}\index{$p$-form}\index{form!$p$-form}, is a tensor field of type $(0,p)$, i.e.
a field of $p$-linear forms, that is fully antisymmetric whenever $p\geq 2$.
This definition generalizes that of a 1-form given in Sec.~\ref{s:bas:linear_form}.

A \defin{frame field}\index{frame!field}\index{field!frame --} or
\defin{moving frame}\index{moving!frame} is a $n$-tuple of vector fields
$(\w{e}_\alpha)$ such that at each point $p\in\M$, $(\left. \w{e}_\alpha\right| _p)$ is
a basis of the tangent space $T_p\M$.
If $n=4$, a frame field is also called a \defin{tetrad}\index{tetrad} and if $n=3$,
it is called a
\defin{triad}\index{triad}.

Given a vector field $\w{v}$ and a scalar field $f$, the function
$\M\rightarrow \R$, $p\mapsto \left.\w{v}\right|_p(f)$ clearly defines a scalar field on
$\M$, which we denote naturally by $\w{v}(f)$.
We may then define the \defin{commutator of two vector fields}\index{commutator} $\w{u}$
and $\w{v}$ as the vector field $[\w{u},\w{v}]$ whose action on a scalar field $f$ is
\be \label{e:bas:def_commutator}
  [\w{u},\w{v}](f) := \w{u}(\w{v}(f)) - \w{v}(\w{u}(f)) .
\ee
With respect to a coordinate system $(x^\alpha)$, it is not difficult, via
(\ref{e:bas:v_va_wpar_a}), to see that the components of the commutator are
\be \label{e:bas:commut_comp}
  \encadre{ [\w{u},\w{v}]^\alpha = u^\mu \der{v^\alpha}{x^\mu}
    - v^\mu \der{u^\alpha}{x^\mu} } .
\ee

\subsection{Immersions, embeddings and submanifolds} \label{s:bas:embed}

Let $\M$ and $\mathscr{N}$ be two smooth manifolds
and $\Phi:\ \M \longrightarrow \mathscr{N}$
be a smooth map, as defined in Sec.~\ref{s:bas:def_manif}.
At a given point $p\in\M$, the \defin{differential}\index{differential!of a smooth map}
of $\Phi$ is the linear map
\be
    \left.\D\Phi \right| _p:\ T_p\M \longrightarrow T_{\Phi(p)}\mathscr{N}
\ee
that ``approximates'' $\Phi$ in the following sense: if $\D\w{x}\in T_p\M$ is the
infinitesimal displacement vector from $p$ to some (infinitesimally close) point $q$
(cf. Sec.~\ref{s:bas:vectors}), then
\be \label{e:bas:def_diff_map}
    \left.\D\Phi \right| _p(\D\w{x}) = \D\w{y},
\ee
where $\D\w{y}$ is the infinitesimal displacement vector of $T_{\Phi(p)}\mathscr{N}$
connecting $\Phi(p)$ to $\Phi(q)$ (cf. Fig.~??).
If $(x^\alpha)$ is a coordinate chart of $\M$ around $p$ and $(y^\beta)$ a chart
of $\mathscr{N}$ around $\Phi(p)$, such that $\Phi$ is expressed as
$y^\beta = Y^\beta(x^\alpha)$, it follows from Eqs.~(\ref{e:bas:dell_dxa_wpar}) and
(\ref{e:bas:def_diff_map}) that
the matrix of the linear map $\left.\D\Phi \right| _p$
with respect to the bases $(\wpar_\alpha)$ of $T_p\M$ and $(\wpar_\beta)$ of $T_{\Phi(p)}\mathscr{N}$
is the Jacobian matrix $(\partial Y^\beta / \partial x^\alpha)$.
Using the characterization of vectors by their action on scalar fields
[Eq.~(\ref{e:bas:def_vector})], it is then easy to see
that
\be
    \forall \w{v}\in T_p\M,\ \forall f \in C^\infty(\mathscr{N},\mathbb{R}),\quad
    \left.\D\Phi \right| _p(\w{v})(f) =
        \w{v}\left(f\circ \Phi \right) .
\ee
This property could be taken as an alternative definition of $\left.\D\Phi \right| _p$.

The smooth map $\Phi$ is called an \defin{immersion}\index{immersion} iff
the differential $\left.\D\Phi \right| _p$ is injective at any point $p\in\M$.
Moreover, $\Phi$ is called an \defin{embedding}\index{embedding} iff (i) $\Phi$
is an immersion and (ii) $\Phi$ is a homeomorphism $\M \rightarrow \Phi(\M)$.
Note that an embedding is necessarily injective, contrary to an immersion.

A \defin{submanifold} of $\M$ is a subset $\mathscr{S}\subset \M$ such
that (i) $\mathscr{S}$ is a manifold in the subspace topology and (ii)
$\mathscr{S}$ has a smooth structure with respect to which
the inclusion map $\iota: \mathscr{S} \rightarrow \M$ is an embedding.
One can show that $\mathscr{S}$ is a submanifold of $\M$ iff there exists
a manifold $\mathscr{S}_0$ (a priori not a subset of $\M$) and an
embedding $\Phi: \mathscr{S}_0 \rightarrow \M$, such that
$\mathscr{S} = \Phi(\mathscr{S}_0)$.

\begin{remark}
Strictly speaking, the above definition regards an
\defin{embedded submanifold}\index{embedded!submanifold}\index{submanifold!embedded --};
there is also the wider concept of \defin{immersed submanifold}\index{immersed!submanifold}\index{submanifold!immersed --} (see e.g. Chap~5 of \cite{Lee13}).
\end{remark}


One has necessarily $\dim \mathscr{S} \leq \dim \M$. The non-negative integer
$m = \dim\M - \dim\mathscr{S}$ is called the \defin{codimension}\index{codimension}
of the submanifold $\mathscr{S}$. A submanifold of codimension 1 is called
a \defin{hypersurface}\index{hypersurface}. A submanifold of dimension 1 is
(the image of) a curve in $\M$, but note that not all curves are submanifolds:
a curve with self-crossing points is not a submanifold.

\subsection{Pushforwards and pullbacks} \label{s:bas:push_pull}

Given a smooth map $\Phi: \M\to\mathscr{N}$ and a vector $\w{v}$ in the tangent
space to $\M$ at a point $p$, the image of $\w{v}$ by the differential
of $\Phi$ at $p$ is called the \defin{pushforward of $\w{v}$ on $\mathscr{N}$ by $\Phi$}\index{pushforward}
and is denoted by $\Phi_* \w{v}$:
\be \label{e:bas:def_pushforward}
    \Phi_* \w{v} := \left.\D\Phi \right| _p (\w{v}) .
\ee
The pushforward can be geometrically interpreted as follows:
consider $\w{v}$ as the tangent vector
to a curve $\Li$ through $p$ associated to some parameter $\lambda$ (cf. Sec.~\ref{s:bas:vectors})
and let $q$ be
the point of $\Li$ separated from $p$ by the infinitesimal parameter increase $\D\lambda$.
One has then $\w{v} = \D\w{x} / \D\lambda$, where $\D\w{x}$ is the infinitesimal
vector connecting $p$ to $q$. Thanks to the linearity of $\left.\D\Phi \right| _p$
and the defining relation (\ref{e:bas:def_diff_map}), one has
$\Phi_* \w{v} = \D\w{y} / \D\lambda$, i.e. $\Phi_* \w{v}$ is the finite vector obtained
by rescaling the infinitesimal vector connecting $\Phi(p)$ to $\Phi(q)$ by $\D\lambda$. Note that
$\Phi_* \w{v}$ is a vector tangent to the curve $\Phi(\Li)\subset \mathscr{N}$: it is precisely
the tangent vector to $\Phi(\Li)$ at $\Phi(p)$ associated with the parametrization of $\Phi(\Li)$
by $\lambda$.

Let now $\w{T}$ be a fully covariant tensor field on $\mathscr{N}$, i.e. a tensor field
of type $(0,\ell)$ on $\mathscr{N}$ for some $\ell \geq 0$. The
\defin{pullback of $\w{T}$ on $\M$ by $\Phi$}\index{pullback}
is the tensor field of type $(0,\ell)$ on $\M$ denoted by
$\Phi^*\w{T}$ and
defined by
\be \label{e:bas:def_pullback}
    \forall p\in\M,\
    \forall (\w{v}_1,\ldots\w{v}_\ell)\in \left( T_p\M \right) ^\ell,\quad
    \left. \Phi^* \w{T} \right| _p (\w{v}_1,\ldots\w{v}_\ell) := \left. \w{T} \right| _{\Phi(p)}
    \left( \Phi_* \w{v}_1, \ldots, \Phi_* \w{v}_\ell \right) .
\ee
For a scalar field $f: \mathscr{N}\to \R$, the above definition with $\ell = 0$ reduces
to $\left. \Phi^* f \right| _p := \left. f \right| _{\Phi(p)}$,
i.e. to $ \Phi^* f(p) := f(\Phi(p))$, so that
the pullback is nothing but the map composition:
$\Phi^* f = f \circ \Phi$.

\begin{remark} \label{r:bas:comp_push_pull}
Via the pushforward operation,
a smooth map $\Phi : \M \to \mathscr{N}$ naturally carries tangent vectors from $\M$ to $\mathscr{N}$
(essentially because it carries curves in $\M$ to curves in $\mathscr{N}$, as discussed above), but
it carries covariant tensors, among which linear forms, in the reverse way, i.e. from
$\mathscr{N}$ to $\M$ (pullback operation). Another difference is that
the pushforward is a pointwise operation and does not extend a priori to vector \emph{fields}
(cf. Fig.~??),
while the pullback is well defined for any covariant tensor field. A case where the
pushforward extends to vector fields though is when $\Phi$ is a diffeomorphism.
\end{remark}


%%%%%%%%%%%%%%%%%%%%%%%%%%%%%%%%%%%%%%%%%%%%%%%%%%%%%%%%%%%%%%%%%%%%%%%%%%%%%%%

\section{Pseudo-Riemannian manifolds} \label{s:bas:pRiemManif}

\subsection{Metric tensor} \label{s:bas:metric}

A \defin{pseudo-Riemannian manifold}\index{pseudo-Riemannian manifold}\index{manifold!pseudo-Riemannian} is a pair $(\M,\w{g})$ where $\M$ is a smooth manifold
and $\w{g}$ is a \defin{metric tensor}\index{metric!tensor}\index{metric} on $\M$,
i.e. a tensor field obeying the following properties:
\begin{enumerate}
\item $\w{g}$ is a tensor field of type $(0,2)$: at each point $p\in\M$, $\left. \w{g} \right| _p$ is a
bilinear form acting on pairs of vectors in the tangent space $T_p\M$:
\be
    \begin{array}{rccl}
    \left. \w{g} \right| _p: & {T_p\M}\times{T_p\M} & \longrightarrow & \R \\
        & (\w{u},\w{v}) & \longmapsto & \left. \w{g} \right| _p (\w{u},\w{v}) =: \w{g}(\w{u},\w{v})
    \end{array}
\ee
\item $\w{g}$ is \defin{symmetric}\index{symmetric}: $\w{g}(\w{u},\w{v}) = \w{g}(\w{v},\w{u})$.
\item $\w{g}$ is \defin{non-degenerate}\index{non-degenerate!bilinear form}: at any point
$p\in\M$,
a vector $\w{u}$ such that
$\forall \w{v}\in{T_p\M},\ \w{g}(\w{u},\w{v}) = 0$ is necessarily the null vector.
\end{enumerate}
The properties of being symmetric and non-degenerate are typical of a
\defin{scalar pro\-duct}\index{scalar!product}. Accordingly,
one says that two vectors
$\w{u}$ and $\w{v}$ are \defin{$\w{g}$-orthogonal}\index{g-orthogonal} (or simply \defin{orthogonal}\index{orthogonal} if there is no ambiguity) iff $\w{g}(\w{u},\w{v}) = 0$.
Moreover, when there is no ambiguity on the metric (usually the spacetime metric), we are
using a dot to denote the scalar product of two vectors taken with $\w{g}$:
\be
  \forall (\w{u},\w{v})\in{T_p\M}\times{T_p\M},\quad
  \encadre{ \w{u}\cdot\w{v} := \w{g}(\w{u},\w{v}) } .
\ee

In a given basis $(\w{e}_\alpha)$ of $T_p\M$, the components of $\w{g}$
is the matrix $(g_{\alpha\beta})$ defined by
formula (\ref{e:bas:def_components}) with $(k,\ell)=(0,2)$:
\be \label{e:bas:g_components}
  \encadre{ \w{g} = g_{\alpha\beta} \, \w{e}^\alpha \otimes \w{e}^\beta }.
\ee
For any pair $(\w{u},\w{v})$ of vectors we have then
$\w{g}(\w{u},\w{v}) = g_{\alpha\beta} u^\alpha v^\beta$.
In particular, considering the natural basis associated with some coordinate system
$(x^\alpha)$, the scalar square of an infinitesimal displacement vector $\D\w{x} = \D x^\alpha\wpar_\alpha$
[cf. Eqs.~(\ref{e:bas:dell_f}) and (\ref{e:bas:dell_dxa_wpar})] is
\be \label{e:bas:line_element}
  \encadre{ \D s^2 := \w{g}(\D\w{x},\D\w{x}) = g_{\alpha\beta} \, \D x^\alpha \, \D x^\beta }.
\ee
This formula is called the \defin{line element}\index{line!element} expression
on the pseudo-Riemannian manifold $(\M,\w{g})$. It is often used to define the metric tensor in
general relativity texts. Note that contrary to what the notation may suggest, $\D s^2$
can be negative.

For the dual basis associated with the coordinates
$(x^\alpha)$, one has $\w{e}^\alpha = \dd x^\alpha$ (cf. Sec.~\ref{s:bas:linear_form}), so that
Eq.~(\ref{e:bas:g_components}) can be rewritten as
\be \label{e:bas:g_components_natural}
    \w{g} = g_{\alpha\beta} \, \dd x^\alpha \otimes \dd x^\beta .
\ee
One can recast this relation in a form which reminds the line element (\ref{e:bas:line_element}) by
introducing the symmetric product notation (cf. e.g. Refs.~\cite{Lee18} or \cite{Strau13}):
\be \label{e:bas:sym_tensor_prod}
    \dd x^\alpha \dd x^\beta := \frac{1}{2} \left(\dd x^\alpha \otimes \dd x^\beta + \dd x^\beta \otimes \dd x^\alpha \right) \qand
    (\dd x^\alpha)^2 := \dd x^\alpha \otimes \dd x^\alpha .
\ee
Formula~(\ref{e:bas:g_components_natural}) then becomes
\be \label{e:bas:g_components_dx}
    \encadre{ \w{g} = g_{\alpha\beta} \, \dd x^\alpha \dd x^\beta } .
\ee
Applying this relation to the pair of infinitesimal vectors $(\D\w{x},\D\w{x})$,
one gets the line element (\ref{e:bas:line_element}), by virtue of the identity
$\langle\dd x^\alpha,\D\w{x} \rangle =  \D x^\alpha$ [Eq.~(\ref{e:bas:dxa_dxa})].


\subsection{Signature and orthonormal bases} \label{s:bas:signature}

An important feature of the metric tensor is its \defin{signature}\index{signature}:
in all bases of $T_p\M$ where the components $(g_{\alpha\beta})$ form a diagonal matrix, there is necessarily the same number, $s$ say, of negative components
and the same number, $n-s$, of positive components. The independence of $s$ from the choice
of the basis where $(g_{\alpha\beta})$ is diagonal is a basic result of linear algebra named \defin{Sylvester's law of inertia}\index{Sylvester's law of inertia}. One writes:
\be
  \mathrm{sign}\; \w{g} = (\underbrace{-,\ldots,-}_{\mbox{$s$ times}},
  \underbrace{+,\ldots,+}_{\mbox{$n-s$ times}}) .
\ee

If $s=0$, $\w{g}$ is called a \defin{Riemannian metric}\index{Riemannian!metric} and
$(\M,\w{g})$ a \defin{Riemannian manifold}\index{Riemannian!manifold}. In this case, $\w{g}$ is
\defin{positive-definite}, which means that $\w{g}(\w{v},\w{v}) \geq 0$ for all $\w{v}\in{T_p\M}$
and $\w{g}(\w{v},\w{v}) = 0$ iff $\w{v}=0$.
A standard example of Riemannian metric is of course the scalar product of the Euclidean space
$\R^n$.

If $s=1$, $\w{g}$ is called a \defin{Lorentzian metric}\index{Lorentzian!metric} and
$(\M,\w{g})$ a \defin{Lorentzian manifold}\index{Lorentzian!manifold}. One may then have
$\w{g}(\w{v},\w{v}) < 0$; vectors for which this occurs are called \defin{timelike}\index{timelike!vector},
whereas vectors for which $\w{g}(\w{v},\w{v}) > 0$ are called \defin{spacelike}\index{spacelike!vector},
and those for which $\w{g}(\w{v},\w{v}) = 0$ are called \defin{null}\index{null!vector}. The subset of $T_p\M$ formed by all null
vectors is termed the \defin{null cone}\index{null!cone} of $\w{g}$ at $p$.

A coordinate $x^\alpha$ of a coordinate system $(x^0,\ldots,x^{n-1})$ is said to be a \defin{timelike coordinate}\index{timelike!coordinate}
(resp. \defin{spacelike coordinate}\index{spacelike!coordinate} or \defin{null coordinate}\index{null!coordinate}) iff
the hypersurfaces defined by $x^\alpha = \mathrm{const}$ are
spacelike\footnote{Cf. Sec.~\ref{s:def:hor_as_null} for the definition of
spacelike, timelike and null hypersurfaces.} (resp. timelike or null).

\begin{remark}
Being timelike, spacelike or null is a property of the coordinate $x^\alpha$ per se (i.e. considering $x^\alpha$ as a scalar field $\mathscr{U}\subset\M\to \R$); on the contrary the causal type of the coordinate vector $\wpar_\alpha$ depends on the
coordinate system $(x^0,\ldots,x^{n-1})$ to which $x^\alpha$ belongs.
More precisely, whatever the causal type of $x^\alpha$, the vector
$\wpar_\alpha$ can be made spacelike, timelike or null by a proper choice of the complementary coordinates
$(x^\beta)_{\beta\neq\alpha}$.
%cf. Remark~\ref{r:sch:r_causal_type} on p.~\pageref{r:sch:r_causal_type}
\end{remark}

A basis $(\w{e}_\alpha)$ of $T_p\M$ is called a \defin{$\w{g}$-orthonormal basis}\index{g-orthonormal basis} (or simply \defin{orthonormal basis}\index{orthonormal basis} if there
is no ambiguity on the metric) iff\footnote{No summation on $\alpha$ is implied.}
\be
   \begin{array}{lcll}
  \w{g}(\w{e}_\alpha,\w{e}_\alpha) = -1 &\quad \mbox{for}\quad & 0 \leq \alpha \leq s-1 \\
  \w{g}(\w{e}_\alpha,\w{e}_\alpha) = 1 & \quad \mbox{for}\quad &s \leq \alpha \leq n-1 \\
  \w{g}(\w{e}_\alpha,\w{e}_\beta)  = 0 & \quad \mbox{for}\quad & \alpha\not=\beta .
  \end{array}
\ee

\subsection{Metric duality} \label{s:bas:metric_dual}

Since the bilinear form $\w{g}$ is non-degenerate, its matrix $(g_{\alpha\beta})$ in
any basis $(\w{e}_\alpha)$ is invertible and the inverse\index{inverse metric} is denoted by $(g^{\alpha\beta})$:
\be
  \encadre{ g^{\alpha\mu} g_{\mu\beta} = \delta^\alpha_{\ \ \beta} }.
\ee
The metric $\w{g}$ induces an isomorphism between
$T_p\M$ (vectors) and ${T_p^*\M}$ (linear forms) which, in  index notation,
corresponds to the lowering\index{lowering an index}\index{index!lowering} or
raising of the index\index{raising an index}\index{index!raising} by contraction
with $g_{\alpha\beta}$ or $g^{\alpha\beta}$.
In the present book, an index-free symbol will always denote
a tensor with a fixed covariance type (such as a vector, a 1-form,
a bilinear form, etc.). We will therefore use a different symbol
to denote its image under the metric isomorphism.
In particular, we denote by an underbar the
isomorphism $T_p\M \rightarrow {T_p^*\M}$
and by an arrow the reverse isomorphism ${T_p^*\M} \rightarrow T_p\M$:
\begin{enumerate}
\item For any vector $\w{u}$ in $T_p\M$, $\uu{u}$ stands for
the unique linear form such that
\be \label{e:bas:underbar}
    \forall \w{v} \in T_p\M,\quad \langle \uu{u}, \w{v}
        \rangle = \w{g}(\w{u},\w{v}) .
\ee
However, we will omit the underbar on the components
of $\uu{u}$, since
the position of the index allows us to distinguish between vectors
and  linear forms, following the standard usage:
if the components of
$\w{u}$ in a given basis $(\w{e}_\alpha)$ are denoted by $u^\alpha$,
the components of $\uu{u}$ in the dual basis $(\w{e}^\alpha)$
are then denoted by $u_\alpha$ and are given by
\be \label{e:bas:u_dual}
  u_\alpha = g_{\alpha\mu} u^\mu .
\ee
\item For any linear form $\w{\omega}$ in ${T_p^*\M}$, $\vw{\w{\omega}}$
stands for the unique vector of $T_p\M$ such that
\be \label{e:bas:arrow_form}
    \forall \w{v} \in T_p\M,\quad
        \w{g}(\vw{\w{\omega}},\w{v}) =
        \langle \w{\omega}, \w{v} \rangle .
\ee
As for the underbar, we will omit the arrow on the components
of $\vw{\w{\omega}}$ by denoting them $\omega^\alpha$; they are given by
\be \label{e:bas:arrow_form_comp}
  \omega^\alpha = g^{\alpha\mu} \omega_\mu .
\ee
\item We extend the arrow notation to {\em bilinear} forms on $T_p\M$ (type-$(0,2)$ tensor):
for any bilinear form $\w{T}$,
we denote by $\vw{T}$ the tensor of type $(1,1)$ such that
\be \label{e:bas:arrow_endo}
    \forall (\w{u},\w{v}) \in T_p\M\times{T_p\M}, \quad
    \w{T}(\w{u},\w{v}) = \vw{T}(\uu{u},\w{v}) = \w{u} \cdot \vw{\w{T}}(\w{v}) ,
\ee
and by $\vvw{T}$ the tensor of type $(2,0)$ defined by
\be \label{e:bas:arrow_double}
    \forall (\w{u},\w{v}) \in {T_p\M}\times{T_p\M}, \quad
    \w{T}(\w{u},\w{v}) = \vvw{\w{T}}(\uu{u},\uu{v}) .
\ee
Note that in the second equality of (\ref{e:bas:arrow_endo}), we have considered $\vw{T}$
as an endomorphism ${T_p\M}\rightarrow {T_p\M}$, which is always possible for a tensor of
type $(1,1)$.
If $T_{\alpha\beta}$ are the components of $\w{T}$
in some basis $(\w{e}_\alpha)$, the components of $\vw{T}$ and $\vvw{T}$ are respectively
\begin{subequations}
\begin{align}
  & (\vw{T})^\alpha_{\ \  \beta} = T^\alpha_{\ \  \beta} = g^{\alpha\mu} T_{\mu\beta}
    \label{e:bas:arrow_endo_index}\\
  & (\vvw{T})^{\alpha\beta} = T^{\alpha\beta} = g^{\alpha\mu} g^{\beta\nu} T_{\mu\nu} .
    \label{e:bas:arrow_double_index}
\end{align}
\end{subequations}
\end{enumerate}

\begin{remark}
In mathematical literature, the isomorphism induced by $\w{g}$ between
${T_p\M}$ and ${T_p^*\M}$ is called the \defin{musical isomorphism}\index{musical isomorphism},
because a flat and a sharp symbols are used instead of,
respectively, the underbar and the arrow introduced above:
\[
  \w{u}^\flat = \uu{u} \qquad\mbox{and}\qquad \w{\omega}^\sharp = \vw{\w{\omega}} .
\]
\end{remark}


\subsection{Levi-Civita tensor} \label{s:bas:Levi-Civita_tensor}

Let us assume that $\M$ is an \defin{orientable manifold}\index{orientable!manifold}, i.e. that there exists a $n$-form\footnote{Cf. Sec.~\ref{s:bas:fields} for the definition of a $n$-form.} on $\M$ ($n$ being
$\M$'s dimension) that is continuous on $\M$ and nowhere vanishing.
Then, given a metric $\w{g}$ on $\M$, one can show that there exist only two
$n$-forms $\weps$ such that for any $\w{g}$-orthonormal basis $(\w{e}_\alpha)$,
\be \label{e:bas:eps_base_pm_un}
  \weps(\w{e}_0,\ldots, \w{e}_{n-1}) = \pm 1 .
\ee
Picking one of these two $n$-forms amounts to choosing an
\defin{orientation}\index{orientation!of a manifold} for $\M$. The chosen $\weps$
is then called the \defin{Levi-Civita tensor}\index{Levi-Civita!tensor} associated with
the metric $\w{g}$. It is also called the \defin{volume form}\index{volume!form}\index{form!volume --}
of $\w{g}$ (cf. Sec.~\ref{s:bas:ext_deriv}).
The sign in the right-hand side of (\ref{e:bas:eps_base_pm_un})
gives then the orientation of the basis $(\w{e}_\alpha)$.
More generally, given a (not necessarily orthonormal) basis $(\w{e}_\alpha)$ of $T_p\M$,
one has $\weps(\w{e}_0,\ldots, \w{e}_{n-1}) \neq 0$
and one says that the basis is \defin{right-handed}\index{right-handed basis}\index{basis!right-handed --}
(resp. \defin{left-handed}\index{left-handed basis}\index{basis!left-handed --})
iff $\weps(\w{e}_0,\ldots, \w{e}_{n-1}) > 0$ (resp. $<0$).
The components of $\weps$ with respect to $(\w{e}_\alpha)$ are
\be \label{e:bas:eps_sqrt_g}
  \encadre{ \epsilon_{\alpha_1\; \ldots\; \alpha_n} = \pm \sqrt{|g|} \;  [\alpha_1, \ldots, \alpha_n] },
\ee
where $\pm$ must be $+$ (resp. $-$) for a right-handed (resp. left-handed) basis,
$g$ stands for the determinant of the matrix of $\w{g}$'s components with respect
to the basis $(\w{e}_\alpha)$:
\be \label{e:bas:def_det_g}
  \encadre{ g := \det (g_{\alpha\beta}) }
\ee
and the symbol $[\alpha_1, \ldots, \alpha_n]$ takes the value $0$ if any two indices
$(\alpha_1, \ldots, \alpha_n)$ are equal and takes the value $1$ (resp. $-1$) if
$(\alpha_1, \ldots, \alpha_n)$ is an even (resp. odd) permutation of
$(0,\ldots,n-1)$.

\subsection{Vector normal to a hypersurface} \label{s:bas:hyp_normal}

In a pseudo-Riemannian manifold, one can associate to any hypersurface
$\Sigma$ (cf. Sec.~\ref{s:bas:embed}) a unique normal direction, which can
be represented by a nonvanishing vector field $\w{n}$ defined on $\Sigma$ as follows.
Locally the hypersurface $\Sigma$ can be considered as a level set, i.e.
there exists a smooth scalar field $f:\M \rightarrow \R$, such that
$\dd f \neq 0$ on $\Sigma$ and
for any point $p$ in the considered local region of $\M$, the following equivalence
holds
\be
    p\in \Sigma \iff f(p) = 0 .
\ee
Then, a vector field $\w{v}$ on $\M$ is tangent to $\Sigma$ iff
the value of $f$ stays to $0$ for a small displacement
$\D\lambda$ along $\w{v}$; thanks to Eqs.~(\ref{e:bas:dell_f}),
(\ref{e:bas:dell_v_dlamb}) and (\ref{e:bas:v_va_wpar_a}), this is equivalent to
$\w{v}(f) = v^\mu\,  \dert{f}{x^\mu} = 0$,
or to $\w{g}(\w{n},\w{v}) = 0$,
where we have let appear the gradient vector $\w{n} := \vw{\nabla} f$; in
terms of components with respect to a coordinate system $(x^\alpha)$:
\be \label{e:bas:gradient_comp}
    n^\alpha = \nabla^\alpha f = g^{\alpha\mu} \der{f}{x^\mu} .
\ee
The vector field $\w{n}$ is called a \defin{normal}\index{normal!to a hypersurface}
to $\Sigma$. All normal vectors to $\Sigma$ are collinear to each other.

It follows from the definitions given in Sec.~\ref{s:bas:signature} that
for any coordinate $x^\alpha$ of a given chart $(x^0,\ldots,x^{n-1})$,
\begin{subequations}
\label{e:bas:char_causal_type_coord}
\begin{align}
& x^\alpha \ \mbox{timelike coordinate}  \iff \vw{\nabla} x^\alpha \ \mbox{timelike vector}
  \iff g^{\alpha\alpha} < 0 ,  \\
& x^\alpha \ \mbox{null coordinate}  \iff \vw{\nabla} x^\alpha \ \mbox{null vector}
  \iff g^{\alpha\alpha} = 0  , \label{e:bas:char_null_coord} \\
& x^\alpha \ \mbox{spacelike coordinate}  \iff \vw{\nabla} x^\alpha \ \mbox{spacelike vector}
  \iff g^{\alpha\alpha} > 0 , \label{e:bas:char_spacelike_coord}
\end{align}
\end{subequations}
where $g^{\alpha\alpha}$ is the component $(\alpha,\alpha)$ of the inverse metric
(no summation on $\alpha$). It appears here because, according to Eq.~(\ref{e:bas:gradient_comp}),
$\w{g}(\vw{\nabla} x^\alpha , \vw{\nabla} x^\alpha ) = g_{\mu\nu} g^{\mu\rho} \dert{x^\alpha}{x^\rho}
 g^{\nu\sigma} \dert{x^\alpha}{x^\sigma} = \delta^\rho_{\ \, \nu} \, \delta^\alpha_{\ \rho} \, g^{\nu\sigma}\,
 \delta^\alpha_{\ \sigma} = g^{\alpha\alpha}$.

%%%%%%%%%%%%%%%%%%%%%%%%%%%%%%%%%%%%%%%%%%%%%%%%%%%%%%%%%%%%%%%%%%%%%%%%%%%%%%%

\section{The three basic derivatives}

Three derivative operators acting on tensor fields can be defined on a
pseudo-Riemannian manifold $(\M,\w{g})$. One of them depends on the metric $\w{g}$:
the \emph{covariant derivative} $\wnab$ (Sec.~\ref{s:bas:cov_deriv}).
Another one depends on the choice of a
reference vector field: the \emph{Lie derivative} $\Lie{}$ (Sec.~\ref{s:bas:Lie}).
The third one, called the \emph{exterior derivative} and denoted by $\dd$, depends only on the smooth-manifold structure, i.e.
it is independent of any (metric or vector) field; on the other hand,
it is applicable only to a specific kind of tensor fields, namely differential forms (Sec.~\ref{s:bas:ext_deriv}).


\subsection{Covariant derivative} \label{s:bas:cov_deriv}


\subsubsection{Affine connection on a manifold} \label{s:bas:affine_connect}

Let us denote by $\mathfrak{X}(\M)$ the space of smooth
vector fields on $\M$. $\mathfrak{X}(\M)$ is an infinite-dimensional
vector space over $\R$.
Given a vector field $\w{v}\in\mathfrak{X}(\M)$, it is not possible from the manifold structure
alone to define its variation between two neighboring points $p$ and $q$. Indeed
a formula like $\D \w{v} := \left. \w{v} \right| _q - \left. \w{v} \right| _p$ is meaningless because
the vectors $\left. \w{v} \right| _q$ and $\left. \w{v} \right| _p$ belong to two distinct vector spaces,
$T_q\M$ and $T_p\M$ respectively (cf. Fig.~\ref{f:bas:tang_space}), so that their
subtraction is a priori not defined.
Note that this issue does not arise for a scalar field [cf. Eq.~(\ref{e:bas:df})].
The solution is to introduce an extra-structure on $\M$, called an
\emph{affine connection}. The term \emph{connection} arises because, by defining the variation of vector fields, this structure
\emph{connects} the various tangent spaces on the manifold. More precisely, an
 \defin{affine connection}\index{affine!connection}\index{connection!affine --} on $\M$ is a map
\be \label{e:bas:def_nabla}
    \begin{array}{cccc}
    \wnab \ : & \mathfrak{X}(\M)\times\mathfrak{X}(\M) & \longrightarrow & \mathfrak{X}(\M) \\
        & (\w{u},\w{v}) & \longmapsto & \wnab_{\w{u}} \,\w{v}
    \end{array}
\ee
that satisfies the following properties:
\begin{enumerate}
\item $\wnab$ is bilinear (considering $\mathfrak{X}(\M)$ as a vector space over $\R$).
\item For any scalar field $f$ and any pair $(\w{u},\w{v})$ of vector fields:
\be
  \wnab_{f\w{u}}\, \w{v} = f \wnab_{\w{u}}\, \w{v} .
\ee
\item For any scalar field $f$ and any pair $(\w{u},\w{v})$ of vector fields, the
following Leibniz rule holds:
\be
  \wnab_{\w{u}}\, (f\w{v}) =
    \langle \dd f, \,\w{u}\rangle\,  \w{v} + f \wnab_{\w{u}}\, \w{v} ,
\ee
where $\dd f$ is the differential of $f$ as defined in Sec.~\ref{s:bas:linear_form}.
\end{enumerate}
The vector $\wnab_{\w{u}} \,\w{v}$ is called the \defin{covariant derivative of $\w{v}$
along $\w{u}$}\index{covariant!derivative!along a vector}\index{derivative!covariant --}.
\begin{remark} \label{r:bas:def_connection}
Property~2 is not implied by property~1, for $f$ is a scalar field, not a real number. Actually, property~2 ensures that the value of $\wnab_{\w{u}} \,\w{v}$
at a given point $p\in\M$ depends only on the vector $\left. \w{u} \right| _p \in{T_p\M}$ and
not on the behavior of $\w{u}$ around $p$; therefore the role of $\w{u}$ is only to
give the direction of the derivative of $\w{v}$.
\end{remark}

Given an affine connection, the variation of a vector field $\w{v}$ between
two neighboring points, $p$ and $q$ say, is defined as
\be
  \D \w{v} := \wnab_{\D\w{x}} \, \w{v} ,
\ee
where
$\D\w{x}$ is the infinitesimal displacement vector connecting $p$ and $q$
[cf Eq.~(\ref{e:bas:dell_f})].
One says that $\w{v}$ is \defin{parallelly transported from $p$ to $q$ with respect to the connection $\wnab$}\index{parallel transport}\index{parallelly transported} iff $\D\w{v} = 0$.

Given a frame field $(\w{e}_\alpha)$ on $\M$, the
\defin{connection coefficients}\index{connection!coefficients}
of an affine connection $\wnab$ with respect to $(\w{e}_\alpha)$ are the
$n^3$ scalar fields $\Gamma^\gamma_{\ \ \alpha\beta}$ defined by the
expansion, at each point $p\in\M$, of the vector
$\left. \wnab_{\w{e}_\beta}\, \w{e}_\alpha \right| _p$ onto the basis $(\left. \w{e}_\alpha \right| _p)$:
\be
    \encadre{ \wnab_{\w{e}_\beta}\, \w{e}_\alpha =:
    \Gamma^\mu_{\ \ \alpha\beta} \, \w{e}_\mu }.
\ee
An affine connection is entirely defined by its connection coefficients in a given
frame field. In other words, there are as many affine connections on a manifold of dimension $n$ as there are possibilities of choosing $n^3$ scalar fields $\Gamma^\gamma_{\ \ \alpha\beta}$.

Given $\w{v}\in\mathfrak{X}(\M)$, the \defin{covariant derivative of $\w{v}$ with respect to the affine connection $\wnab$}\index{covariant!derivative} is the tensor field  $\wnab\w{v}$
of type $(1,1)$ defined by the following action at each point $p\in\M$:
\be \label{e:bas:def_cov_deriv}
    \begin{array}{cccc}
    \left. \wnab\w{v} \right| _p \ : & {T_p^*\M}\times{T_p\M} & \longrightarrow & \R \\
        & (\w{\omega},\w{u}) & \longmapsto &
    \langle \w{\omega},\, \left. \wnab_{\w{\tilde u}} \, \w{v} \right| _p \rangle ,
    \end{array}
\ee
where $\w{\tilde u}$ is any vector field which performs some extension of $\w{u}$ around
$p$: $\left.  \w{\tilde u} \right| _p = \w{u}$. As already noted
(cf. Remark~\ref{r:bas:def_connection}), $\left. \wnab_{\w{\tilde u}} \, \w{v} \right| _p$ is
independent of the choice of $\w{\tilde u}$, so that the mapping (\ref{e:bas:def_cov_deriv}) is well-defined. By comparing with (\ref{e:bas:def_tensor}),
we verify that $\left. \wnab\w{v}\right| _p $ is a tensor of type $(1,1)$.

The \defin{covariant derivative} is extended to all tensor fields by
(i) demanding that for a scalar field it coincides with the differential: $\wnab f := \dd f$
and (ii) using the Leibniz rule.
As a result, the covariant derivative\index{derivative!covariant --} of a tensor field $\w{T}$ of type $(k,\ell)$ is
a tensor field $\w{\nabla}\w{T}$ of type $(k,\ell+1)$.
Its components with respect a given frame field $(\w{e}_\alpha)$
are denoted
\be \label{e:bas:nab_T_comp_gam}
\nabla_\gamma T^{\alpha_1\ldots\alpha_k}_{\qquad\ \; \beta_1\ldots\beta_\ell}
    :=
(\w{\nabla}\w{T})^{\alpha_1\ldots\alpha_k}_{\qquad\ \; \beta_1\ldots\beta_\ell\gamma}
\ee
and are given by
\bea
\nabla_\gamma T^{\alpha_1\ldots\alpha_k}_{\qquad\ \; \beta_1\ldots\beta_\ell}&=&
 \w{e}_\gamma (T^{\alpha_1\ldots\alpha_k}_{\qquad\ \; \beta_1\ldots\beta_\ell})
+ \sum_{i=1}^k \Gamma^{\alpha_i}_{\ \ \ \sigma\gamma}\; T^{\alpha_1\ldots
\!{{{\scriptstyle i\atop\downarrow}\atop \scriptstyle\sigma}\atop\ }\!\!
\ldots\alpha_k}_{\qquad\ \ \ \  \  \  \ \beta_1\ldots\beta_\ell} \nonumber \\
& & -  \sum_{i=1}^\ell \Gamma^\sigma_{\ \ \ \beta_i\gamma} \;
T^{\alpha_1\ldots\alpha_k}_{\qquad\ \; \beta_1\ldots
\!{\ \atop {\scriptstyle\sigma \atop {\uparrow\atop \scriptstyle i}} }\!\!
\ldots\beta_\ell}  , \label{e:bas:der_cov_coord}
\eea
where $\w{e}_\gamma (T^{\alpha_1\ldots\alpha_k}_{\qquad\ \; \beta_1\ldots\beta_\ell})$
stands for the action of the vector $\w{e}_\gamma$ on the scalar field
$T^{\alpha_1\ldots\alpha_k}_{\qquad\ \; \beta_1\ldots\beta_\ell}$ resulting from the
very definition of a vector (cf. Sec.~\ref{s:bas:vectors}).
In particular, if $(\w{e}_\alpha)$ is a natural frame associated with some
coordinate system $(x^\alpha)$, then $\w{e}_\alpha = \wpar_\alpha$ and
$\w{e}_\gamma (T^{\alpha_1\ldots\alpha_k}_{\qquad\ \; \beta_1\ldots\beta_\ell})
= \dert{T^{\alpha_1\ldots\alpha_k}_{\qquad\ \; \beta_1\ldots\beta_\ell}}{x^\gamma}$
[cf. Eq.~(\ref{e:bas:wpar_partial})].

\begin{remark} \label{r:bas:nab_index}
Notice the position of the index $\gamma$ in Eq.~(\ref{e:bas:nab_T_comp_gam}): it is the
last one on the right-hand side. According to (\ref{e:bas:def_components}), $\wnab\w{T}$ is
then expressed as
\be \label{e:bas:nab_T_expand}
    \w{\nabla}\w{T} =
    \nabla_{\gamma} \,
        T^{\alpha_1\ldots\alpha_k}_{\qquad\ \; \beta_1\ldots\beta_\ell}
        \; \w{e}_{\alpha_1} \otimes \ldots \otimes \w{e}_{\alpha_k}
                \otimes
        \w{e}^{\beta_1} \otimes \ldots \otimes \w{e}^{\beta_\ell}
        \otimes \w{e}^\gamma  .
\ee
Because $\w{e}^\gamma$ is the
{\em last} 1-form of the tensorial product on the right-hand side, the
notation
$T^{\alpha_1\ldots\alpha_k}_{\qquad\ \; \beta_1\ldots\beta_\ell;\gamma}$ instead of
$\nabla_{\gamma} \,
T^{\alpha_1\ldots\alpha_k}_{\qquad\ \; \beta_1\ldots\beta_\ell}$
would have been more appropriate.
The index convention (\ref{e:bas:nab_T_expand}) agrees with that
of MTW \cite{MisneTW73} [cf. their Eq.~(10.17)].
\end{remark}

The \defin{covariant derivative of the tensor field $\w{T}$ along a
vector}\index{covariant!derivative!along a vector} $\w{v}$
is defined by
\be \label{e:bas:directional_der}
    \wnab_{\w{v}}\w{T} := \wnab\w{T}
        (\underbrace{.,\ldots,.}_{k+\ell\ {\rm slots}},\w{u}) .
\ee
The components of $\w{\nabla}_{\w{v}}\w{T}$ are then
$v^\mu \nabla_{\mu}
T^{\alpha_1\ldots\alpha_k}_{\qquad\ \; \beta_1\ldots\beta_\ell}$.
Note that $\w{\nabla}_{\w{v}}\w{T}$ is a tensor field of the same type as $\w{T}$.
In the particular case of a scalar field $f$, one has
$\wnab_{\w{v}} f = \langle \wnab f , \w{v} \rangle = \w{v}(f) $.

The \defin{divergence}\index{divergence!tensor} with respect to the affine connection $\wnab$ of a tensor field $\w{T}$ of type $(k,\ell)$ with $k\geq 1$ is the tensor field
denoted $\wnab\cdot\w{T}$ of type $(k-1,\ell)$ and whose components with respect to any
frame field are given by
\be \label{e:bas:def_divergence}
  (\wnab\cdot\w{T})^{\alpha_1\ldots\alpha_{k-1}}_{\qquad\quad \beta_1\ldots\beta_\ell}
  = \wnab_\mu T^{\alpha_1\ldots\alpha_{k-1}\mu}_{\qquad\quad\ \  \;  \beta_1\ldots\beta_\ell} .
\ee
\begin{remark} \label{r:bas:divergence_last}
For the divergence, the contraction is performed on the \emph{last} upper index of $\w{T}$.
\end{remark}

\subsubsection{Levi-Civita connection} \label{s:bas:Levi-Civita_connect}

On a pseudo-Riemannian manifold $(\M,\w{g})$ there is a unique affine connection
$\wnab$ such that
\begin{enumerate}
\item $\wnab$ is \defin{torsion-free}\index{torsion-free}, i.e. for any scalar field $f$,
$\wnab\wnab f$ is a field of \emph{symmetric} bilinear forms:
\be \label{e:bas:torsion-free}
  \nabla_\alpha\nabla_\beta f = \nabla_\beta\nabla_\alpha f .
\ee
\item The covariant derivative of the metric tensor vanishes identically:
\be \label{e:bas:nabla_g_zero}
  \encadre{ \wnab\w{g} = 0 } .
\ee
\end{enumerate}
$\wnab$ is called the \defin{Levi-Civita connection associated with $\w{g}$}\index{Levi-Civita!connection}\index{connection!Levi-Civita --}.

With respect to the Levi-Civita connection, the Levi-Civita tensor $\weps$ introduced
in Sec.~\ref{s:bas:Levi-Civita_tensor} shares the same property as $\w{g}$:
\be \label{e:bas:nab_eps}
  \encadre{ \wnab\weps = 0 } .
\ee

Given a coordinate system $(x^\alpha)$ on $\M$, the connection coefficients of the
Levi-Civita connection with respect to the natural basis $(\wpar_\alpha)$
are called the \defin{Christoffel symbols}\index{Christoffel symbols} of $\w{g}$; they
can be evaluated
from the partial derivatives of the metric components with respect to the coordinates:
\be \label{e:bas:Christoffel}
  \Gamma^\gamma_{\ \ \alpha\beta} = \frac{1}{2} g^{\gamma\mu}
    \left( \der{g_{\mu\beta}}{x^\alpha} + \der{g_{\alpha\mu}}{x^\beta}
    - \der{g_{\alpha\beta}}{x^\mu} \right) .
\ee
Note that the Christoffel symbols are symmetric with respect to the lower two indices.

For the Levi-Civita connection, the expression for the divergence of a vector takes
a rather simple form in a natural basis.
Indeed, combining Eqs.~(\ref{e:bas:def_divergence}) and (\ref{e:bas:der_cov_coord}),
we get for $\w{v}\in\mathfrak{X}(\M)$,
$\wnab\cdot\w{v} = \nabla_\mu v^\mu = \dert{v^\mu}{x^\mu} + \Gamma^\mu_{\ \ \sigma\mu} v^\sigma $.
Now, from (\ref{e:bas:Christoffel}),  we have
\be \label{e:bas:trGam_det_g}
  \Gamma^\mu_{\ \ \alpha\mu} =  \frac{1}{2} g^{\mu\nu} \der{g_{\mu\nu}}{x^\alpha}
  = \frac{1}{2} \der{}{x^\alpha} \ln|g|
  = \frac{1}{\sqrt{|g|}}\der{} {x^\alpha} \sqrt{|g|} ,
\ee
where $g := \det(g_{\alpha\beta})$ [Eq.~(\ref{e:bas:def_det_g})].
The last but one equality follows from the general law of variation of the determinant of any
invertible matrix $A$:
\be \label{e:bas:variation_det}
    \encadre{ \delta(\ln |\det A|) = \mathrm{tr} (A^{-1} \times \delta A) } ,
\ee
where $\delta$ denotes any variation that fulfills the Leibniz rule (i.e. a derivation),
$\mathrm{tr}$ stands for the trace and $\times$ for the matrix product.
We conclude that\index{divergence!vector}
\be \label{e:bas:div_vect}
  \encadre{ \wnab\cdot\w{v} = \frac{1}{\sqrt{|g|}} \der{}{x^\mu} \left(
  \sqrt{|g|} \; v^\mu \right) . }
\ee
Similarly, for an antisymmetric tensor field of type $(2,0)$,
\[
   \nabla_\mu A^{\alpha\mu}
  = \der{A^{\alpha\mu}}{x^\mu} +
  \underbrace{\Gamma^\alpha_{\ \ \sigma\mu} A^{\sigma\mu}}_{0}
  + \Gamma^\mu_{\ \ \sigma\mu} A^{\alpha\sigma}
  = \der{A^{\alpha\mu}}{x^\mu} +  \frac{1}{\sqrt{|g|}}\der{} {x^\sigma} \sqrt{|g|}
  \;  A^{\alpha\sigma} ,
\]
where we have used the fact that $\Gamma^\alpha_{\ \ \sigma\mu}$ is symmetric in
$(\sigma,\mu)$, whereas $A^{\sigma\mu}$ is antisymmetric.
Hence the simple formula for the divergence of an \emph{antisymmetric} tensor field
of type $(2,0)$:
\be \label{e:bas:div_antisym}
  \encadre{  \nabla_\mu A^{\alpha\mu} =  \frac{1}{\sqrt{|g|}} \der{}{x^\mu} \left(
  \sqrt{|g|} \; A^{\alpha\mu} \right)  } .
\ee


\subsection{Lie derivative} \label{s:bas:Lie}

As discussed in Sec.~\ref{s:bas:affine_connect}, the concept of a derivative of a vector field on a manifold $\M$
requires the introduction of some extra-structure on $\M$. The extra-structure considered in
Sec.~\ref{s:bas:affine_connect} is an affine connection, possibly
provided by some metric tensor (case of the Levi-Civita connection).
Another possible extra-structure is a ``reference''
vector field, with respect to which the derivative is to be defined. This leads to the
concept of the \emph{Lie derivative}, which we discuss here.


\subsubsection{Lie derivative of a vector field} \label{s:bas:Lie_der_vector}

Consider a vector field $\w{u}$ on $\M$, which shall serve as our ``reference flow''.
Let $\w{v}$ be another vector field on $\M$, the variation of which is to be studied.
One can use $\w{u}$ to transport $\w{v}$ from one point $p$ to
a neighboring one $q$ and then define the variation of $\w{v}$
as the difference between the actual value $\left. \w{v} \right| _q$ of $\w{v}$ at $q$ and the transported
value via $\w{u}$. More precisely, given an infinitesimal
parameter $\veps$, we define the \defin{flow map along}\index{flow map} $\w{u}$
as the map $\Phi_\veps: \M \to \M$ such that
$\Phi_\veps(p)=q$, where $q$ is the point connected to $p$ by the infinitesimal
displacement vector
$\overrightarrow{pq}=\veps \left. \w{u} \right| _p$ (cf. Sec.~\ref{s:bas:vectors} and
Fig.~\ref{f:bas:deriv}).
The natural reference vector to compare $\left. \w{v} \right| _q$ with is then the
pushforward\index{pushforward} of
$\left. \w{v} \right| _p$ by $\Phi_\veps$, i.e. the vector
$\Phi_{\veps*} \left. \w{v} \right| _p$ defined in Sec.~\ref{s:bas:push_pull}.
Since $\left. \w{v} \right| _q = \left. \w{v} \right| _{\Phi_\veps(p)}$ and
$\Phi_{\veps*} \left. \w{v} \right| _p$ belong to the same vector space $T_q\M$,
one may subtract the latter from the former and define
the \defin{Lie derivative}\index{Lie!derivative}\index{derivative!Lie --}
of $\w{v}$ along $\w{u}$ at $p$ by
\be \label{e:bas:def_Lie_der}
    \encadre{
    \left. \Lie{u} \w{v} \right| _p := \lim_{\veps\rightarrow 0} \frac{1}{\veps}
    \left( \left. \w{v} \right| _{\Phi_\veps(p)}
        - \Phi_{\veps*} \left. \w{v} \right| _p \right) }.
\ee

\begin{figure}
\centerline{\includegraphics[width=0.6\textwidth]{bas_lie_deriv.pdf}}
\caption[]{\label{f:bas:deriv}
\footnotesize
Geometrical construction of the Lie derivative of a
vector field $\w{v}$ along a vector field $\w{u}$:
given an infinitesimal parameter $\D\lambda$, each extremity of the arrow
$\D\lambda \left. \w{v} \right| _p$ is dragged by some small parameter $\veps$
along $\w{u}$, to form
the vector denoted by $\Phi_{\veps*}\D\lambda \left. \w{v} \right| _p$ (cf. Sec.~\ref{s:bas:push_pull}).
The latter is then compared with
the actual value of $\D\lambda\, \w{v}$ at the point $q$, the difference (divided
by $\veps\D\lambda$) defining the Lie derivative $\Lie{u}\w{v}$ at $p$.}
\end{figure}

\begin{remark}
The term between parentheses
in the right-hand side of Eq.~(\ref{e:bas:def_Lie_der})
is the difference between two vectors of $T_q\M$ and thus
a vector of $T_q\M$. One gets a vector of $T_p\M$ in the left-hand side
only because $q = \Phi_\veps(p)$ tends to $p$ in the limit $\veps \to 0$. Some authors (e.g. \cite{Lee13,Wald84}) prefer to define the Lie derivative from the difference of vectors in $T_p\M$, by carrying
$\left. \w{v} \right| _q$ to $T_p\M$ via the pushforward by $\Phi_\veps^{-1}=\Phi_{-\veps}$ and comparing it with
$\left. \w{v} \right| _p$; this results in the formula
\be \label{e:bas:def_Lie_der_alt}
    \left. \Lie{u} \w{v} \right| _p := \lim_{\veps\rightarrow 0} \frac{1}{\veps}
    \left( (\Phi_\veps^{-1})_* \left. \w{v} \right| _{\Phi_\veps(p)}
    -  \left. \w{v} \right| _p\right) .
\ee
Thanks to the limit $\veps\to 0$, formulas~(\ref{e:bas:def_Lie_der}) and (\ref{e:bas:def_Lie_der_alt})
are equivalent. Formula~(\ref{e:bas:def_Lie_der}) is preferred here in so far as it corresponds
to the geometrical construction displayed in Fig.~\ref{f:bas:deriv}.
\end{remark}

Let $(x^\alpha)$ be a coordinate system adapted to the
field $\w{u}$ in the sense that the first
vector of the natural basis $(\wpar_\alpha)$ associated with $(x^\alpha)$
is $\wpar_0 = \w{u}$. The coordinate
expression of the flow map $\Phi_\veps$ is then $(x^\alpha) \to (x^\alpha + \veps \delta^\alpha_{\ \,  0})$.
The corresponding Jacobian matrix is the identity matrix so that the coordinate components of
$\Phi_{\veps*} \left. \w{v} \right| _p$ are identical to those of $\left. \w{v} \right| _p$
(cf. Sec.~\ref{s:bas:embed}). It follows that the component expression of Eq.~(\ref{e:bas:def_Lie_der})
is
\bea
     \left(  \left. \Lie{u} \w{v} \right| _p \right)^\alpha & = &
   \lim_{\veps\rightarrow 0} \frac{1}{\veps}
    \left[ (\left. \w{v} \right| _{\Phi_\veps(p)})^\alpha
        - ( \left. \w{v} \right| _p )^\alpha \right] \nonumber \\
   &  = & \lim_{\veps\rightarrow 0} \frac{1}{\veps}
    \left[ v^\alpha(x^0 + \veps, x^1, \ldots, x^{n-1})
    - v^\alpha(x^0, x^1, \ldots, x^{n-1}) \right] . \nonumber
\eea
Hence, in adapted coordinates, the Lie derivative is simply obtained by taking the partial derivative of the vector components with respect to $x^0$:
\be \label{e:bas:Lie_adapted_vec}
    \Liec{u} v^\alpha  = \der{v^\alpha}{x^0} ,
\ee
where we have used the standard notation for the components of a Lie derivative:
$\Liec{u} v^\alpha := \left( \Lie{u} \w{v} \right)^\alpha$.
Besides, using the fact that the components of $\w{u}$
are $u^\alpha=(1,0,\ldots,0)$ in the adapted coordinate system, we notice that the components
of the commutator of $\w{u}$ and $\w{v}$, as given by (\ref{e:bas:commut_comp}), are
$[\w{u},\w{v}]^\alpha = \dert{v^\alpha}{x^0}$.
This is exactly (\ref{e:bas:Lie_adapted_vec}): $[\w{u},\w{v}]^\alpha = \Liec{u} v^\alpha$. We conclude that the Lie derivative of a vector with respect to another
one is actually nothing but the commutator\index{commutator} of these two vectors:
\be \label{e:bas:Lie_commut}
    \encadre{ \Lie{u} \w{v} = [\w{u},\w{v}] } .
\ee
Thanks to formula (\ref{e:bas:commut_comp}), we may then express the components of the Lie
derivative in an arbitrary coordinate system:
\be \label{e:bas:Lie_vect}
    \encadre{ \Liec{u} v^\alpha = u^\mu \der{v^\alpha}{x^\mu}
    - v^\mu \der{u^\alpha}{x^\mu} } .
\ee

In view of the symmetry of the Christoffel symbols,
the partial derivatives in Eq.~(\ref{e:bas:Lie_vect}) can be
replaced by the covariant derivative $\wnab$, yielding
\be \label{e:bas:Lie_vect_nab}
  \Liec{u} v^\alpha = u^\mu \nabla_\mu v^\alpha
    - v^\mu \nabla_\mu u^\alpha .
\ee

\subsubsection{Generalization to any tensor field} \label{s:bas:lie_der_tensor}

Let $\w{T}$ be a tensor field of type $(0,\ell)$ on $\M$, with $\ell \geq 0$.
The \defin{Lie derivative}\index{Lie!derivative}\index{derivative!Lie --} of $\w{T}$ along $\w{u}$ is then
defined by comparing the pullback of $\w{T}$ by the flow map $\Phi_\veps$ at some point $p$
(cf. Sec.~\ref{s:bas:push_pull})
to the actual value of $\w{T}$ at the same point:
\be \label{e:bas:def_Lie_der_covar}
    \encadre{ \Lie{u} \w{T} := \lim_{\veps\rightarrow 0} \frac{1}{\veps}
    \left( \Phi_\veps^*\w{T} - \w{T} \right) }.
\ee
For a scalar field $f$, one has $\Phi_\veps^* f = f\circ \Phi_\veps$
(cf. Sec.~\ref{s:bas:push_pull}), so that the above definition with $\ell=0$
yields
$\Lie{u} f = \w{u}(f) = \langle\dd f, \w{u} \rangle = u^\mu \partial_\mu f$ [cf. Eq.~(\ref{e:bas:def_vector})].

Finally, the Lie derivative is extended to any tensor field by means of the Leibniz
rule. As a result, the Lie derivative $\Lie{u}\w{T}$ of a tensor field $\w{T}$ of type
$(k,\ell)$ is a tensor field of the same type, the components of which
with respect to a given coordinate system $(x^\alpha)$ are
\bea
\Liec{u} T^{\alpha_1\ldots\alpha_k}_{\qquad\ \; \beta_1\ldots\beta_\ell} &=&
u^\mu \der{}{x^\mu} T^{\alpha_1\ldots\alpha_k}_{\qquad\ \; \beta_1\ldots\beta_\ell}
- \sum_{i=1}^k T^{\alpha_1\ldots
\!{{{\scriptstyle i\atop\downarrow}\atop \scriptstyle\sigma}\atop\ }\!\!
\ldots\alpha_k}_{\qquad\ \ \ \  \  \  \; \beta_1\ldots\beta_\ell}
 \; \der{u^{\alpha_i}}{x^\sigma} \nonumber \\
 & &  +  \sum_{i=1}^\ell T^{\alpha_1\ldots\alpha_k}_{\qquad\ \; \beta_1\ldots
\!{\ \atop {\scriptstyle\sigma \atop {\uparrow\atop \scriptstyle i}} }\!\!
\ldots\beta_\ell}
\; \der{u^{\sigma}}{x^{\beta_i}} . \label{e:Lie_der_comp}
\eea
In particular, for a 1-form,
\be \label{e:Lie_der_1form}
    \Liec{u} \omega_\alpha = u^\mu \der{\omega_\alpha}{x^\mu}
    + \omega_\mu \der{u^\mu}{x^\alpha} .
\ee
In coordinates adapted to the vector field $\w{u}$, we have
$u^\alpha = (1,0,\ldots,0)$, so that
$u^\mu \dert{}{x^\mu} = \dert{}{x^0}$ and $\dert{u^\alpha}{x^\beta} = 0$.
Accordingly, Eq.~(\ref{e:Lie_der_comp}) reduces to
\be \label{e:bas:Lie_adapted}
    \Liec{u} T^{\alpha_1\ldots\alpha_k}_{\qquad\ \; \beta_1\ldots\beta_\ell}
     = \der{}{x^0} T^{\alpha_1\ldots\alpha_k}_{\qquad\ \; \beta_1\ldots\beta_\ell}
     \qquad \mbox{(coordinates adapted to $\w{u}$)}.
\ee
This generalizes Eq.~(\ref{e:bas:Lie_adapted_vec}) obtained for vector fields.

As for the vector case [Eq.~(\ref{e:bas:Lie_vect})], the
partial derivatives in Eq.~(\ref{e:Lie_der_comp}) can be
replaced by the covariant derivative $\wnab$ (or any other connection without torsion),
yielding
\bea
\Liec{u} T^{\alpha_1\ldots\alpha_k}_{\qquad\ \; \beta_1\ldots\beta_\ell} &=&
u^\mu \nabla_\mu T^{\alpha_1\ldots\alpha_k}_{\qquad\ \; \beta_1\ldots\beta_\ell}
- \sum_{i=1}^k T^{\alpha_1\ldots
\!{{{\scriptstyle i\atop\downarrow}\atop \scriptstyle\sigma}\atop\ }\!\!
\ldots\alpha_k}_{\qquad\ \ \ \  \  \  \; \beta_1\ldots\beta_\ell}
 \; \nabla_\sigma u^{\alpha_i} \nonumber \\
 & & +  \sum_{i=1}^\ell T^{\alpha_1\ldots\alpha_k}_{\qquad\ \; \beta_1\ldots
\!{\ \atop {\scriptstyle\sigma \atop {\uparrow\atop \scriptstyle i}} }\!\!
\ldots\beta_\ell}
\; \nabla_{\beta_i} u^{\sigma} . \label{e:bas:Lie_der_comp_nab}
\eea
A special case of this formula is worth considering, namely
$\w{T}=\w{g}$ (the metric tensor). Since $\nabla_\mu g_{\alpha\beta} = 0$,
and $g_{\sigma\beta} \nabla_\alpha u^\sigma = \nabla_\alpha u_\beta$ (both thanks to
Eq.~(\ref{e:bas:nabla_g_zero})), one gets the so-called \defin{Killing expression}\index{Killing!expression}
of the Lie derivative of the metric tensor:
\be \label{e:bas:Lie_g_Killing}
   \encadre{ \Liec{u} g_{\alpha\beta} = \nabla_\alpha u_\beta + \nabla_\beta u_\alpha }.
\ee


\subsection{Exterior derivative} \label{s:bas:ext_deriv}

In Sec.~\ref{s:bas:fields}, we have introduced the
\emph{differential forms}\index{differential!form}\index{form!differential --}
or \emph{$p$-forms}
as tensor fields of type $(0,p)$, with $p\ge 0$,
that are antisymmetric in all their arguments as soon as $p\ge 2$.
Differential forms play a special role in the theory of integration on
manifolds. Indeed, the primary definition of an integral over a manifold $\M$ of
dimension $n$ is the integral of a $n$-form.
The Levi-Civita tensor $\weps$
introduced in Sec.~\ref{s:bas:Levi-Civita_tensor} is a $n$-form, whose integral
over a $n$-dimensional submanifold $\mathscr{D}\subset\M$ gives the volume of $\mathscr{D}$
with respect to the metric $\w{g}$.
The electromagnetic field is a 2-form (cf. Sec.~\ref{e:fra:electrovacuum})
and the vorticity of a fluid is described by a 2-form as well
in relativistic hydrodynamics (cf. e.g. Ref.~\cite{Gourg13}).

Being tensor fields, differential forms are subject to the covariant
and Lie derivatives discussed above. In addition, they are subject to a third kind
of derivative defined as follows.  Given any $p$-form $\w{\omega}$,
its \defin{exterior derivative}\index{exterior!derivative}\index{derivative!exterior --}
is the $(p+1)$-form $\dd\w{\omega}$
whose action on a $(p+1)$-tuple of vector fields $(\w{v}_1,\ldots,\w{v}_{p+1})$
is given by
\begin{align}
\dd\w{\omega}(\w{v}_1,\ldots,\w{v}_{p+1}) := & \sum_{i=1}^{p+1} (-1)^{i+1}
    \w{v}_i\left(\w{\omega}(\w{v}_1,\ldots,\w{v}_{i-1},\w{v}_{i+1},\ldots,\w{v}_{p+1})\right) \nonumber \\
    & + \sum_{i,j=1\atop i < j}^{p+1}
    (-1)^{i+j} \w{\omega}\left([\w{v}_i,\w{v}_j],\w{v}_1,\ldots,\w{v}_{i-1},\w{v}_{i+1},
        \ldots,\w{v}_{j-1},\w{v}_{j+1},\ldots,\w{v}_{p+1}\right) , \label{e:bas:def_ext_der}
\end{align}
where in the first line $\w{v}_i\left(\cdots\right)$ stands for the action of the vector field
$\w{v}_i$ on the scalar field $\w{\omega}(\w{v}_1,\ldots,\w{v}_{i-1},\w{v}_{i+1},\ldots,\w{v}_{p+1})$
[cf. Eq.~(\ref{e:bas:def_vector})].
For $p=0$ (scalar field), Eq.~(\ref{e:bas:def_ext_der}) with $\w{\omega}\to f$  reduces
to $\langle \dd f, \w{v}_1 \rangle := \w{v}_1(f)$, i.e. one recovers Eq.~(\ref{e:bas:df_v_eq_v_f}). Hence, for a
scalar field, the exterior derivative is nothing but the differential. For $p=1$, Eq.~(\ref{e:bas:def_ext_der})
becomes
\be \label{e:bas:def_ext_der_p1}
    \dd\w{\omega}(\w{v}_1,\w{v}_2) := \w{v}_1 \left( \langle \w{\omega}, \w{v}_2 \rangle \right)
        - \w{v}_2 \left( \langle \w{\omega}, \w{v}_1 \rangle \right)
        - \left\langle \w{\omega},\, [\w{v}_1,\w{v}_2] \right\rangle .
\ee
In terms of components with respect to a coordinate system $(x^\alpha)$,
the definition (\ref{e:bas:def_ext_der}) yields
\begin{subequations}
\label{e:bas:def_der_ext_comp}
\begin{align}
    \mbox{0-form: \ }  & (\dd\w{\omega})_\alpha =
        \der{\omega}{x^\alpha} \label{e:bas:def_ext_0f} \\
    \mbox{1-form: \ }  & (\dd\w{\omega})_{\alpha\beta} =
    \der{\omega_\beta}{x^\alpha} - \der{\omega_\alpha}{x^\beta}
             \label{e:bas:def_ext_1f} \\
    \mbox{2-form: \ }  & (\dd\w{\omega})_{\alpha\beta\gamma} =
    \der{\omega_{\beta\gamma}}{x^\alpha} -
    \der{\omega_{\alpha\gamma}}{x^\beta} +
    \der{\omega_{\alpha\beta}}{x^\gamma} \label{e:bas:def_ext_2f} \\
     \mbox{$p$-form: \ }  &
     (\dd\w{\omega})_{\alpha_1\cdots\alpha_{p+1}} =
     \sum_{i=1}^{p+1} (-1)^{i+1} \der{}{x^{\alpha_i}}
     \omega_{\alpha_1\cdots\alpha_{i-1}\alpha_{i+1}\cdots\alpha_{p+1}} .
\end{align}
\end{subequations}

\begin{remark}
The exterior derivative appeals only to the
manifold structure; it does not depend upon the metric tensor $\w{g}$, nor upon
any other extra structure on $\M$.
Nevertheless, one may replace
all partial derivatives in formulas (\ref{e:bas:def_der_ext_comp})
by covariant derivatives taken with respect to the Levi-Civita connection $\wnab$
of $\w{g}$,
thanks to the symmetry of the Christoffel symbols on their last two indices:
\begin{subequations}
\begin{align}
    \mbox{0-form: \ } & (\dd\w{\omega})_\alpha =
        \nabla_\alpha \omega \label{e:bas:def_ext_0f_nab} \\
    \mbox{1-form: \ } & (\dd\w{\omega})_{\alpha\beta} =
        \nabla_\alpha \omega_\beta - \nabla_\beta \omega_\alpha
            \label{e:bas:def_ext_1f_nab} \\
    \mbox{2-form: \ } & (\dd\w{\omega})_{\alpha\beta\gamma} =
    \nabla_\alpha\omega_{\beta\gamma} -
    \nabla_\beta\omega_{\alpha\gamma} +
    \nabla_\gamma\omega_{\alpha\beta} \label{e:bas:def_ext_2f_nab} \\
         \mbox{$p$-form: \ }  &
     (\dd\w{\omega})_{\alpha_1\cdots\alpha_{p+1}} =
     \sum_{i=1}^{p+1} (-1)^{i+1} \nabla_{\alpha_i}
     \omega_{\alpha_1\cdots\alpha_{i-1}\alpha_{i+1}\cdots\alpha_{p+1}} .
\end{align}
\end{subequations}
\end{remark}

A fundamental property of the exterior derivative is to be nilpotent:
\be \label{e:ext_der_nilpot}
    \encadre{ \dd\dd\w{\omega} = 0 }.
\ee
A $p$-form $\w{\omega}$ is said to be \defin{closed}\index{closed!differential form} iff $\dd\w{\omega}=0$,
and \defin{exact}\index{exact diff. form} iff there exists a $(p-1)$-form $\w{\sigma}$ such that
$\w{\omega} = \dd\w{\sigma}$. Thanks to property (\ref{e:ext_der_nilpot}),
any exact $p$-form is closed. The \defin{Poincaré lemma}\index{Poincaré!lemma} states that the converse is true,
at least locally, on a contractible open set.

Given a vector field $\w{v}$ on an oriented pseudo-Riemannian manifold $(\M,\w{g})$ of dimension $n$,
the tensor field formed by setting the first argument of the
Levi-Civita tensor $\weps$ to $\w{v}$, while leaving free the other slots, i.e. $\weps(\w{v},.,\ldots,.)$,
is a $(n-1)$-form
denoted\footnote{More generally, the dot notation stands for the contraction on adjacent indices; here: $(\w{v}\cdot\weps)_{\alpha_1\cdots\alpha_{n-1}} := v^\mu \eps_{\mu\alpha_1\cdots\alpha_{n-1}}$.}
$\w{v}\cdot\weps$. Its exterior
derivative is thus a $n$-form, which is expressible in terms of the divergence
of $\w{v}$ [cf. Eq.~(\ref{e:bas:div_vect})] as
\be \label{e:bas:dveps_divv}
    \encadre{ \dd (\w{v}\cdot\weps) = (\wnab\cdot\w{v})\, \weps }.
\ee

The exterior derivative enters in the famous \defin{Stokes' theorem}\index{Stokes!theorem}:
if $\mathscr{D}$ is an oriented $d$-dimensional manifold with boundary (cf. Sec.~\ref{s:bas:manif_boundary}), then for any
$(d-1)$-form $\w{\omega}$ on $\mathscr{D}$,
\be \label{e:bas:Stokes}
   \encadre{ \int_{\partial\mathscr{D}} \w{\omega} =
    \int_{\mathscr{D}} \dd\w{\omega} },
\ee
where $\partial\mathscr{D}$ is $\mathscr{D}$'s boundary
oriented according to the \defin{outward convention}: if $\weps_{\mathscr{D}}$ is the $d$-form defining
the orientation of $\mathscr{D}$ (cf. Sec.~\ref{s:bas:Levi-Civita_tensor}), the $(d-1)$-form
defining the orientation of $\partial\mathscr{D}$ is chosen to be $\w{v}\cdot\weps_{\mathscr{D}}$,
where $\w{v}$ is an outward-pointing\footnote{By \defin{outward-pointing}\index{outward-pointing}, it is meant
that at each point $p\in\partial\mathscr{D}$, $\left.\w{v}\right| _p$ is not tangent to $\partial\mathscr{D}$
and there exists a curve in $\mathscr{D}$ starting from $p$ whose tangent vector at $p$ is
$-\left.\w{v}\right| _p$.}
vector field along $\partial\mathscr{D}$.
Note that each side of
(\ref{e:bas:Stokes}) is (of course!) well-defined,
as the integral of a $p$-form over an oriented $p$-dimensional manifold.

\begin{remark}
If $\mathscr{D}$ is a submanifold of an oriented pseudo-Riemannian manifold
$(\M,\w{g})$ of dimension $n$, the orientation of $\mathscr{D}$ is naturally set by choosing
$\weps_{\mathscr{D}} = \weps$ if $d=n$, where $\weps$
is the Levi-Civita tensor of $\w{g}$ (cf. Sec.~\ref{s:bas:Levi-Civita_tensor}).
If $d<n$, a free family of $n-d$ vector fields $(\w{u}_1,\ldots,\w{u}_{n-d})$
nowhere tangent to $\mathscr{D}$ has to be chosen first and $\weps_{\mathscr{D}}$ is then
defined by
$\weps_{\mathscr{D}} = \weps(\w{u}_1,\ldots,\w{u}_{n-d},.,\ldots,.)$.
Accordingly, the orientation of $\partial\mathscr{D}$ is set by
$\weps_{\partial\mathscr{D}} = \w{v}\cdot\weps_{\mathscr{D}} = \weps(\w{u}_1,\ldots,\w{u}_{n-d},\w{v},.,\ldots,.)$.
\end{remark}

Another important formula involving the exterior derivative is
the \defin{Cartan identity}\index{Cartan!identity}, which
expresses the Lie derivative of a $p$-form
$\w{\omega}$ along a vector field $\w{u}$ as
\be \label{e:bas:Cartan}
    \encadre{ \Lie{u}\w{\omega} = \w{u}\cdot\dd\w{\omega}
    + \dd(\w{u}\cdot\w{\omega}) }.
\ee
As above, a dot denotes the contraction on adjacent indices:
$ \w{u}\cdot\dd\w{\omega}$ is the $p$-form $\dd \w{\omega}(\w{u},.,\ldots,.)$
and
$\w{u}\cdot\w{\omega}$ is the $(p-1)$-form $\w{\omega}(\w{u},.,\ldots,.)$.
Notice that if $\w{\omega}$ is
a 1-form, Eq.~(\ref{e:bas:Cartan}) is readily obtained
by combining Eqs.~(\ref{e:Lie_der_1form}),
(\ref{e:bas:def_ext_0f}) and (\ref{e:bas:def_ext_1f}).
An immediate consequence of the Cartan identity and the nilpotence property
(\ref{e:ext_der_nilpot}) is that the Lie derivative
and the exterior derivative commute, i.e. for any vector field $\w{u}$
and any $p$-form $\w{\omega}$:
\be \label{e:bas:Lie_ext_commute}
    \Lie{u} \dd \w{\omega} = \dd \, \Lie{u}\! \w{\omega} .
\ee

%%%%%%%%%%%%%%%%%%%%%%%%%%%%%%%%%%%%%%%%%%%%%%%%%%%%%%%%%%%%%%%%%%%%%%%%%%%%%%%

\section{Curvature} \label{s:bas:curvat}

\subsection{General definition}

The \defin{Riemann curvature tensor}\index{Riemann!curvature}\index{curvature!tensor} of
an affine connection $\w{\nabla}$ (cf. Sec.~\ref{s:bas:affine_connect}) is defined by
\be \label{e:bas:def_Riemann}
     \begin{array}{cccc}
    \mathrm{\bf Riem} \ : & \mathfrak{X}^*(\M)\times\mathfrak{X}(\M)^3 &
    \longrightarrow & C^\infty(\M,\R) \\
        & (\w{\omega},\w{w},\w{u},\w{v})
        & \longmapsto & \bigg\langle \w{\omega} , \
                \w{\nabla}_{\w{u}} \w{\nabla}_{\w{v}} \w{w}
        -  \w{\nabla}_{\w{v}} \w{\nabla}_{\w{u}} \w{w}
        - \w{\nabla}_{[\w{u},\w{v}]} \w{w} \bigg\rangle ,
    \end{array}
\ee
where $\mathfrak{X}^*(\M)$ stands for the space of 1-forms on $\M$, $\mathfrak{X}(\M)$ for the space of vector
fields on $\M$ and  $C^\infty(\M,\R)$ for the space of
smooth scalar fields on $\M$. The above
formula does define a tensor field on $\M$, i.e. the value
of $\mathrm{\bf Riem}(\w{\omega},\w{w},\w{u},\w{v})$ at a given
point $p\in\M$ depends only upon the values of the fields
$\w{\omega}$, $\w{w}$, $\w{u}$ and $\w{v}$ at $p$ and not
upon their behaviors away from $p$, as the covariant derivatives in
Eq.~(\ref{e:bas:def_Riemann}) might suggest.
We denote the components of this tensor in
a given basis $(\w{e}_\alpha)$
by $R^\gamma_{\ \  \delta \alpha\beta}$, instead of
${\rm Riem}^\gamma_{\ \  \delta \alpha\beta}$.
The definition (\ref{e:bas:def_Riemann}) leads to the
following formula, named the \defin{Ricci identity}\index{Ricci!identity}:
\be \label{e:bas:Ricci_ident}
    \forall\w{w}\in\mathfrak{X}(\M),\quad
       \encadre{ \left(\nabla_\alpha\nabla_\beta
        - \nabla_\beta\nabla_\alpha\right) w^\gamma
        = R^\gamma_{\ \  \mu \alpha\beta} \, w^\mu }.
\ee
\begin{remark}
In view of this identity, one may say that the Riemann tensor measures the lack of
commutativity of two successive covariant derivatives of a vector field.
On the opposite,
for a scalar field and a torsion-free connection,
two successive covariant derivatives always commute [cf. Eq.~(\ref{e:bas:torsion-free})].
\end{remark}
In a coordinate basis, the components of the Riemann tensor are given in terms of the connection
coefficients by
\be \label{e:bas:Riemann_comp}
    \encadre{ R^\alpha_{\ \  \beta\mu\nu}  =
    \der{\Gamma^\alpha_{\ \  \beta\nu}}{x^\mu}
    - \der{\Gamma^\alpha_{\ \  \beta\mu}}{x^\nu}
    + \Gamma^\alpha_{\ \  \sigma\mu} \Gamma^\sigma_{\ \  \beta\nu}
    - \Gamma^\alpha_{\ \  \sigma\nu} \Gamma^\sigma_{\ \  \beta\mu}  } .
\ee

From the definition (\ref{e:bas:def_Riemann}), the Riemann tensor is
clearly antisymmetric with respect to its last two arguments:
\be \label{e:bas:Riemann_antisym34}
  \mathrm{\bf Riem}(.,.,\w{u},\w{v}) = - \mathrm{\bf Riem}(.,.,\w{v},\w{u}) .
\ee
In addition, it satisfies the cyclic property
\be \label{e:bas:Riemann_cyclic}
\mathrm{\bf Riem}(.,\w{u},\w{v},\w{w})
+\mathrm{\bf Riem}(.,\w{w},\w{u},\w{v})
+\mathrm{\bf Riem}(.,\w{v},\w{w},\w{u}) = 0 .
\ee
The covariant derivatives of the Riemann tensor obey the \defin{Bianchi identity}\index{Bianchi identity}
\be \label{e:bas:Bianchi}
    \encadre{ \nabla_\rho  R^\alpha_{\ \  \beta\mu\nu}
    + \nabla_\mu R^\alpha_{\ \ \beta\nu\rho}
    + \nabla_\nu R^\alpha_{\ \ \beta\rho\mu} =0 }.
\ee

\subsection{Case of a pseudo-Riemannian manifold}

The Riemann tensor of the Levi-Civita connection obeys the additional antisymmetry:
\be \label{e:bas:Riemann_antisym12}
    \mathrm{\bf Riem}(\w{\omega},\w{w},.,.)
    = - \mathrm{\bf Riem}(\uu{w},\vw{\omega},.,.) .
\ee
Combined with (\ref{e:bas:Riemann_antisym34}) and (\ref{e:bas:Riemann_cyclic}), this implies
the symmetry property
\be \label{e:bas:Riemann_sym}
  \mathrm{\bf Riem}(\w{\omega},\w{w},\w{u},\w{v}) =
  \mathrm{\bf Riem}(\uu{u},\w{v},\vw{\omega},\w{w}) .
\ee

A pseudo-Riemannian manifold $(\M,\w{g})$ with a vanishing Riemann tensor is called
a \defin{flat manifold}\index{flat!manifold}; in this case, $\w{g}$ is said to be
a \defin{flat metric}\index{flat!metric}. If in addition, $\w{g}$ has a Riemannian (resp. Lorentzian) signature,
it is called an
\defin{Euclidean metric}\index{Euclidean!metric} (resp. \defin{Minkowski metric}\index{Minkowski!metric}).

\subsection{Ricci tensor} \label{s:bas:Ricci_tensor}

The \defin{Ricci tensor}\index{Ricci!tensor} of the affine connection $\wnab$ is
the field of bilinear forms $\w{R}$ defined by
\be \label{e:bas:def_Ricci}
     \begin{array}{cccc}
    \w{R} \ : & \mathfrak{X}(\M)\times\mathfrak{X}(\M) &
    \longrightarrow & C^\infty(\M,\R) \\
        & (\w{u},\w{v})
        & \longmapsto &
                \mathrm{\bf Riem}(\w{e}^\mu,\w{u},\w{e}_\mu,\w{v}) ,
    \end{array}
\ee
where $(\w{e}_\alpha)$ is a vector frame on $\M$ and $(\w{e}^\alpha)$
its dual counterpart.
This definition is independent of the choice of $(\w{e}_\alpha)$.
In terms of components:
\be \label{e:bas:def_Ricci_comp}
    \encadre{ R_{\alpha\beta} := R^\mu_{\ \  \alpha\mu\beta} }.
\ee
\begin{remark}
Following standard usage, we denote the components
of the Riemann and Ricci tensors by the same letter $R$, the
number of indices allowing us to distinguish between the two tensors.
On the other hand, we are using different symbols, $\mathrm{\bf Riem}$ and
$\w{R}$, when employing the index-free notation.
\end{remark}

The Ricci tensor naturally appears in the
\defin{contracted Ricci identity}\index{Ricci!identity!contracted --}\index{contracted!Ricci identity}:
\be \label{e:bas:contract_Ricci_ident}
    \forall\w{w}\in\mathfrak{X}(\M),\quad
    \encadre{ \nabla_\mu \nabla_\alpha w^\mu - \nabla_\alpha \nabla_\mu w^\mu =
        R_{\mu\alpha} w^\mu },
\ee
which is obtained by taking the trace of the Ricci identity (\ref{e:bas:Ricci_ident}) on the indices $(\alpha,\gamma)$ and relabelling $\beta\rightarrow\alpha$.


For the Levi-Civita connection associated with the metric $\w{g}$, property (\ref{e:bas:Riemann_sym}) implies that the Ricci tensor is symmetric:
\be
  \w{R}(\w{u},\w{v}) = \w{R}(\w{v},\w{u}) .
\ee
In addition, one defines the
\defin{Ricci scalar}\index{Ricci!scalar}
(also called \defin{scalar curvature}\index{scalar!curvature}\index{curvature!scalar})
as the trace of the Ricci tensor with respect to the metric $\w{g}$:
\be \label{e:bas:def_Ricci_scal}
  R :=g^{\mu\nu} R_{\mu\nu} .
\ee
The Bianchi identity (\ref{e:bas:Bianchi}) implies the divergence-free property
\be \label{e:bas:Bianchi_contr}
  \encadre{ \wnab\cdot\vw{G} = 0 },
\ee
where $\vw{G}$ in the type-$(1,1)$ tensor associated by metric duality
[cf. (\ref{e:bas:arrow_endo})] to
the \defin{Einstein tensor}\index{Einstein!tensor}:
\be \label{e:bas:Einstein_tensor}
  \encadre{ \w{G} := \w{R} - \frac{1}{2} R\, \w{g} } .
\ee
Equation~(\ref{e:bas:Bianchi_contr}) is called the \defin{contracted Bianchi identity}\index{contracted!Bianchi identity}\index{Bianchi identity!contracted}.

\subsection{Weyl tensor} \label{s:bas:Weyl}

Let $(\M,\w{g})$ be a pseudo-Riemannian manifold of dimension $n$.

For $n=1$, the Riemann tensor vanishes identically, i.e. $(\M,\w{g})$  is
necessarily flat.
The reader having in mind a curved line in the Euclidean plane $\R^2$ might be
surprised by the above statement. This is because the Riemann tensor
represents the  \emph{intrinsic} curvature\index{intrinsic curvature}\index{curvature!intrinsic} of a manifold. For a curved line $\Li$, the
nonzero curvature is the
\emph{extrinsic} curvature\index{extrinsic curvature}\index{curvature!extrinsic --}, i.e. the
curvature resulting from the embedding of $\Li$ in $\R^2$.

For $n=2$, the Riemann tensor is entirely determined by the
Ricci scalar $R$, according to the formula:
\be \label{e:bas:Riem_n_2}
  R^\gamma_{\ \; \delta\alpha\beta} =  \frac{R}{2} \left(
    \delta^\gamma_{\ \  \alpha} \, g_{\delta\beta}   -
    \delta^\gamma_{\ \  \beta} \, g_{\delta\alpha}
         \right) \qquad (n=2) .
\ee

For $n=3$, the Riemann tensor is entirely determined by the
Ricci tensor, according to
\be  \label{e:bas:Riem_dim_3}
    R^\gamma_{\ \; \delta\alpha\beta}  =
     R^\gamma_{\ \  \alpha} \, g_{\delta\beta}
       - R^\gamma_{\ \  \beta}\,  g_{\delta\alpha}
       + \delta^\gamma_{\ \  \alpha} \, R_{\delta\beta}
       - \delta^\gamma_{\ \  \beta}  \, R_{\delta\alpha}
         + \frac{R}{2} \left(
  \delta^\gamma_{\ \  \beta} \, g_{\delta\alpha}
       - \delta^\gamma_{\ \  \alpha} \, g_{\delta\beta}   \right)
   \quad (n=3) .
\ee

For $n\geq 4$, the Riemann tensor can
be split into (i) a ``trace-trace'' part, represented
by the Ricci scalar $R$ [Eq.~(\ref{e:bas:def_Ricci_scal})],
(ii) a ``trace'' part,
represented by the Ricci tensor $\w{R}$
[Eq.~(\ref{e:bas:def_Ricci_comp})], and (iii) a ``traceless'' part,
which is constituted by the \defin{Weyl conformal curvature tensor}\index{Weyl curvature tensor}\index{conformal!curvature}, $\w{C}$:
\bea
    R^\gamma_{\ \; \delta\alpha\beta}   & = &
        C^\gamma_{\ \; \delta\alpha\beta}
    + \frac{1}{n-2} \left( R^\gamma_{\ \  \alpha} \, g_{\delta\beta}
       - R^\gamma_{\ \  \beta}\,  g_{\delta\alpha}
       + \delta^\gamma_{\ \  \alpha} \, R_{\delta\beta}
       - \delta^\gamma_{\ \  \beta} \, R_{\delta\alpha}   \right)
                            \nonumber \\
     &&   + \frac{1}{(n-1)(n-2)} R \left(
  \delta^\gamma_{\ \  \beta} \, g_{\delta\alpha}
       - \delta^\gamma_{\ \  \alpha} \, g_{\delta\beta}   \right) . \label{e:bas:Weyl}
\eea
The above relation may be taken as the definition of $\w{C}$.
It implies that $\w{C}$ is traceless: $C^\mu_{\ \  \alpha\mu\beta}=0$.
The other possible traces are zero thanks to the symmetries of
the Riemann tensor.
\begin{remark}
The decomposition (\ref{e:bas:Weyl}) is meaningful for $n=3$ as well;
by comparing with (\ref{e:bas:Riem_dim_3}), we see that it results in
a vanishing Weyl tensor: $\w{C} = 0$ for $n=3$.
\end{remark}

