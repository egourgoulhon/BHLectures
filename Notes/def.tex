\chapter{The concept of black hole}

\minitoc

\section{Introduction}

\section{General framework}

\subsection{Spacetime}

In these lectures we consider a $n$-dimensional \defin{spacetime}\index{spacetime},
i.e. a pair $(\M, \w{g})$, where $\M$ is a $n$-dimensional smooth manifold
(cf. Sec.~\ref{s:bas:manif} in Appendix~\ref{s:bas}) and $\w{g}$ is a
Lorentzian metric on $\M$ (cf. Sec.~\ref{s:bas:pRiemManif} in
Appendix~\ref{s:bas}). In many parts, the dimension $n$ will be set to 4
--- the standard spacetime value --- but we shall also consider spacetimes with
$n>4$, especially in Chap.~??.

We shall specify further the spacetime structure later on, in particular its
asymptotics and the orientability of some of its parts.

\subsection{Einstein equation}


\subsection{Worldlines}



\section{Null hypersurfaces as one-way membranes}

A naive definition of a black hole, involving only words, could be
\begin{quote}
A \defin{black hole}\index{black!hole} is a localized region of spacetime
from which neither massive particles nor light may escape.
\end{quote}
There are essentially two features in this definition: localization
and inescapability. Let us for a moment focus on the latter.
It implies the existence of a \emph{boundary}, which no
(massive or not) particle emitted in the black hole region can cross.
This boundary is called the
\defin{event horizon}\index{event!horizon}\index{horizon!event --} and
quite often referred to simply as the \defin{horizon}.
It is a \defin{one-way membrane}\index{one-way membrane}\index{membrane!one-way --},
in the sense that it can be crossed from the black hole ``exterior'' towards
the black hole region, but not in the reverse way. The one-way membrane must
a hypersurface of the spacetime manifold $\M$, for it has to divide $\M$ in two regions:
the interior (the black hole itself) and the exterior region.
Let us recall that a hypersurface is a submanifold of $\M$ of codimension 1
(cf. Sec.~\ref{s:bas:embed} in Appendix~\ref{s:bas}).

To discuss further which hypersurface could act as a black hole boundary,
one should recall that, on a Lorentzian manifold $(\M,\w{g})$, there are
three types of hypersurfaces. The classification
depends on the type of metric induced by $\w{g}$ on the hypersurface, the
\defin{induced metric}\index{induced!metric}\index{metric!induced --} being
nothing but the restriction $\left.\w{g}\right| _{\Sigma}$ of $\w{g}$
to vector fields tangent to $\Sigma$.
The hypersurface $\Sigma$ is said to be
\begin{itemize}
\item \defin{spacelike} iff $\left.\w{g}\right| _{\Sigma}$ is positive definite,
i.e. iff $\mathrm{sign} \left.\w{g}\right| _{\Sigma} = (+,+,+)$,
i.e. iff $(\Sigma,  \left.\w{g}\right| _{\Sigma})$ is a Riemannian manifold;
\item \defin{timelike} iff $\left.\w{g}\right| _{\Sigma}$ is a Lorentzian metric,
i.e. iff $\mathrm{sign} \left.\w{g}\right| _{\Sigma} = (-,+,+)$,
i.e. iff $(\Sigma,  \left.\w{g}\right| _{\Sigma})$ is a Lorentzian manifold;
\item \defin{null} iff $\left.\w{g}\right| _{\Sigma}$ is generate,
i.e. iff $\mathrm{sign} \left.\w{g}\right| _{\Sigma} = (0,+,+)$.
\end{itemize}
The hypersurface kind can also be deduced from the type of a normal vector
$\w{n}$ to the hypersurface (cf. Sec.~\ref{s:bas:hyp_normal})
