\chapter{Geodesics} \label{s:geo}

\minitoc

\section{Introduction}

Geodesics play a key role in general relativity, since they represent
the worldlines of test particles, either massive ones or masseless ones (photons)
(cf. Sec.~\ref{s:fra:worldlines}). Moreover, in the context of black holes,
null geodesics play a proeminant role, as the generator of the event horizons
(cf. Sec.~\ref{s:glo:properties_H}). We review here the definition and
properties of geodesics on a generic pseudo-Riemannian manifold.
Contrary to Appendix~\ref{s:bas}, proofs of the various properties will be provided.


\section{Definition and first properties}

On a Riemannian manifold, i.e. a manifold equipped with a positive definite metric
(cf. Sec.~\ref{s:bas:signature}), a geodesic is the curve of minimal length between
two points. It is also a curve along which tangent vectors are transported parallel
to themselves. A typical example is a geodesic in the Euclidean space: this is
necessarily a straight line; it is then obvious that its tangent vectors
keep a fixed direction. In a \emph{pseudo}-Riemannian manifold, such as the
spacetimes of general relativity, one uses
the second property to define geodesics:
\begin{greybox}
A smooth curve $\mathcal{C}$ of a pseudo-Riemannian manifold $(\M,\w{g})$ is
called a \defin{geodesic} iff it admits a parametrization $P$ whose associated
tangent vector field $\w{v}$ is transported parallellely to itself along
$\mathcal{C}$:
\be \label{e:geo:geod_eq_v}
    \encadre{\wnab_{\w{v}}\,\w{v} = 0},
\ee
where $\wnab$ is the Levi-Civita connection associated with $\w{g}$.
The parametrization $P$ is then called an
\defin{affine parametrization}\index{affine!parametrization}\index{parametrization!affine --}
and the corresponding parameter $\lambda$ is called an
\defin{affine paramater}\index{affine!parameter}\index{parameter!affine --} of the
geodesic $\mathcal{C}$. Note that the relation between $\w{v}$ and $\lambda$ is
\be \label{e:geo:v_dxdlambda}
    \w{v} = \frac{\D\w{x}}{\D\lambda} ,
\ee
where $\D\w{x}$ is the infinitesimal displacement along $\mathcal{C}$ corresponding
to the change $\D\lambda$ in the parameter $\lambda$
(cf. Eq.~(\ref{e:bas:dell_v_dlamb})).
\end{greybox}
In the above definition, let us recall that a \emph{curve} $\mathcal{C}$ is the image of a map (called
a \emph{parametrization}) $P:\ I\rightarrow \M$,
where $I$ is an interval of $\R$, spanned by $\lambda$ (cf Sec.~\ref{s:bas:vectors}).

The qualifier \emph{affine} in the above definition stems from the following
property:
\begin{greybox}
Any two affine parametrizations of a geodesic $\mathcal{C}$ are necessarily
related by an affine transformation:
\be \label{e:geo:affine_transf}
    \lambda' = a \lambda + b,
\ee
where $a$ and $b$ are two real constants such that $a\not = 0$.
\end{greybox}
\begin{proof}
Let $P: I \to  \mathcal{C}$, $\lambda\mapsto P(\lambda)$ and
$P': I'\to \mathcal{C}$,
$\lambda'\mapsto P'(\lambda')$ be two
parametrizations of $\mathcal{C}$. They are necessarily related by a
diffeomorphism $I\to I'$, $\lambda \mapsto \lambda'(\lambda)$. It follows
from (\ref{e:geo:v_dxdlambda}) that the tangent vector fields $\w{v}$ and $\w{v'}$
associated with these two parametrizations are related by
\[
    \w{v} = \frac{\D\lambda'}{\D\lambda} \, \w{v'} .
\]
Using the rules 2 and 3 governing the connection $\wnab$ (cf. Sec.~\ref{s:bas:affine_connect}),
we get then
\be \label{e:geo:acc_v_acc_vp}
    \wnab_{\w{v}}\,\w{v} = \frac{\D^2\lambda'}{\D\lambda^2} \, \w{v'}
    + \left( \frac{\D\lambda'}{\D\lambda} \right)^2 \wnab_{\w{v'}}\,\w{v'} .
\ee
If both parametrizations are affine, then $\wnab_{\w{v}}\,\w{v} = 0$ and
$\wnab_{\w{v'}}\,\w{v'} = 0$, so that the above identity reduces to
${\D^2\lambda'}/{\D\lambda^2} = 0$, which implies
the affine law (\ref{e:geo:affine_transf}).
\end{proof}

An important property of geodesics is
\begin{greybox}
Let $\mathcal{C}$ be a geodesic of $(\M,\w{g})$ and $\w{v}$ a tangent vector field
associated with an affine parametrization of $\mathcal{C}$. Then the
scalar square of $\w{v}$ with respect to the metric $\w{g}$
is constant along $\mathcal{C}$:
\be
    \w{g}(\w{v},\w{v}) = \mathrm{const}.
\ee
\end{greybox}
\begin{proof}
The variation of $\w{g}(\w{v},\w{v})$ along $\mathcal{C}$ is given
by
\bea
 \wnab_{\w{v}} \left[ \w{g}(\w{v},\w{v}) \right] & = & v^\mu \nabla_\mu (g_{\rho\sigma} v^\rho v^\sigma) \nonumber \\
            & = & v^\mu \underbrace{\nabla_\mu g_{\rho\sigma}}_{0} v^\rho v^\sigma
                + g_{\rho\sigma} \underbrace{v^\mu \nabla_\mu v^\rho}_{0} v^\sigma
                + g_{\rho\sigma} v^\rho \underbrace{v^\mu \nabla_\mu v^\sigma}_{0}  \nonumber \\
            & = & 0 ,  \nonumber
\eea
where we have used the facts that $\wnab$ is the Levi-Civita connection of $\w{g}$ [Eq.~(\ref{e:bas:nabla_g_zero})] and $\w{v}$ obeys the geodesic equation (\ref{e:geo:geod_eq_v}).
\end{proof}


Geodesics can be characterized by any of their tangent vectors, i.e.
tangent vectors not necessarily associated with an affine parametrization, as follows:
\begin{greybox}
A curve $\mathcal{C}$ is a geodesic
iff the tangent vector field $\w{v}$ associated with any parametrization
of $\mathcal{C}$ obeys
\be \label{e:geo:v_pregeodesic}
    \wnab_{\w{v}}\,\w{v} = \kappa\,  \w{v},
\ee
where $\kappa$ is a scalar field along $\mathcal{C}$.
\end{greybox}
\begin{proof}
Let $P: I\to \mathcal{C}$, $\lambda\mapsto P(\lambda)$ be the parametrization of $\mathcal{C}$
corresponding to the tangent vector field $\w{v}$: $\w{v} = \D\w{x}/\D\lambda$.
If $\mathcal{C}$ is a geodesic, then there exists a parametrization
$\lambda'\mapsto P'(\lambda')$ whose tangent vector, $\w{v'}$ say, obeys
$\wnab_{\w{v'}}\,\w{v'} = 0$. Since any two parametrizations of $\mathcal{C}$
are related by Eq.~(\ref{e:geo:acc_v_acc_vp}), we deduce that $\w{v}$ obeys
(\ref{e:geo:v_pregeodesic}) with
\[
    \kappa = \left( \frac{\D\lambda'}{\D\lambda} \right)^{-1}
        \frac{\D^2\lambda'}{\D\lambda^2} .
\]
Conversely, if $\w{v}$ obeys (\ref{e:geo:v_pregeodesic}) with $\kappa=\kappa(\lambda)$,
then Eq.~(\ref{e:geo:acc_v_acc_vp}) implies that $\wnab_{\w{v'}}\,\w{v'} = 0$,
i.e. that $\mathcal{C}$ is a geodesic, provided that the change of
parametrization $\lambda' = \lambda'(\lambda)$ fulfills
\[
    \kappa(\lambda) \frac{\D\lambda'}{\D\lambda} -  \frac{\D^2\lambda'}{\D\lambda^2} = 0 .
\]
This differential equation has the following solution:
\[
    \lambda' = a \int_{\lambda_1}^\lambda \left[ \exp \left(
    \int_{\lambda_0}^{\tilde{\lambda}} \kappa(\tilde{\tilde{\lambda}})
    \D\tilde{\tilde{\lambda}}\right) \D\tilde\lambda
    \right] + b,
\]
where $a$, $b$, $\lambda_0$ and $\lambda_1$ are constants, with $a\not = 0$ and $\lambda_0,\lambda_1\in I$.
\end{proof}
The above property motivates the following definitions:
\begin{greybox}
A vector field $\w{v}$ obeying (\ref{e:geo:v_pregeodesic}) is called
a \defin{pregeodesic vector field}\index{pregeodesic vector field}.
The scalar field $\kappa$ is then called the \defin{non-affinity coefficient}\index{non-affinity coefficient} of $\w{v}$.
If $\kappa=0$, $\w{v}$ is naturally called a \defin{geodesic vector field}\index{geodesic!vector field}.
\end{greybox}
Note that the property established above is equivalent to stating that the
field lines of a pregeodesic vector field are geodesics.

\section{The geodesic equation}

\begin{greybox}
Let $\mathcal{C}$ be a curve in a pseudo-Riemannian
manifold $(\M,\w{g})$ of dimension $n$. Let us assume that
$\mathcal{C}$ is contained in the domain of a coordinate chart $(x^\alpha)_{0\leq\alpha\leq n-1}$,
so that any parametrization of $\mathcal{C}$, $P: I \to  \mathcal{C}$, $\lambda\mapsto P(\lambda)$,
can be described by $n$ functions $X^\alpha: I\to \R$
according to Eq.~(\ref{e:bas:curve_param_equation}): $x^\alpha(P(\lambda)) = X^\alpha(\lambda)$.
Then the curve $\mathcal{C}$ is a geodesic iff there exists a paramatrization of $\mathcal{C}$
for which the functions $X^\alpha$ fullfills the system of $n$ second-order differential equations:
\be
    \encadre{ \frac{\D^2 X^\alpha}{\D\lambda^2} + \Gamma^\alpha_{\ \, \mu \nu}
    \frac{\D X^\mu}{\D\lambda} \frac{\D X^\nu}{\D\lambda} = 0 },  \qquad 0 \leq \alpha \leq n-1,
\ee
where the $\Gamma^\alpha_{\ \, \mu \nu}$'s are the Christoffel symbols\index{Christoffel symbols} of the metric $\w{g}$
with respect to the coordinates $(x^\alpha)$, as given by Eq.~(\ref{e:bas:Christoffel}).
\end{greybox}


\section{Geodesics and extremal lengths}

\section{Complete geodesics}

Uniqueness of geodesics, exponential map

\section{Geodesics and symmetries}


