\chapter{Geodesics} \label{s:geo}

\minitoc

\section{Introduction and definition}

Geodesics play a key role in general relativity, since they represent
the worldlines of test particles, either massive ones or masseless ones (photons)
(cf. ~Sec.~\ref{s:fra:worldlines}).
On a Riemannian manifold, i.e. a manifold equipped with a positive definite metric
(cf. Sec.~\ref{s:bas:signature}), a geodesic is the curve of minimal length between
two points. It is also a curve along which tangent vectors are transported parallel
to themselves. A typical example is a geodesic in the Euclidean space: this is
necessarily a straight line; it is then obvious that its tangent vectors
keep a fixed direction. In a \emph{pseudo}-Riemannian manifold, such as the
spacetimes of general relativity, one uses
the second property to define geodesics:
\begin{greybox}
A curve $\mathscr{C}$ of a pseudo-Riemannian manifold $(\M,\w{g})$ is
called a geodesics
\end{greybox}



