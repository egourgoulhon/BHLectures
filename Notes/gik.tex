\chapter{Null geodesics and images in Kerr spacetime}
\label{s:gik}

\minitoc

\section{Introduction}

Having investigated the properties of generic causal geodesics
in Schwarzschild spacetime in Chap.~\ref{s:ges}, we focus here on null
geodesics.


\section{Main properties of null geodesics} \label{s:gik:properties}

We shall distinguish the null geodesics with $E=0$ from those having
$E \neq 0$. Indeed, in the latter case, we will rescale the angular momentum
$L$ and the Carter constant $Q$ by $E$, so that only two constants of motion become
pertinent for the study: $L/E$ and $Q/E^2$.
We thus treat first the particular case $E=0$.

\subsection{Null geodesics with $E=0$}


First, we note that a geodesic $\Li$ with $E=0$ cannot exist outside the ergoregion
$\mathscr{G}$, by virtue of the result (\ref{e:gek:E_positive}). In particular,
it cannot exist far from the black hole.

Another property of null geodesics with $E=0$ is to have a non-negative Carter constant:
\be \label{e:gik:Q_nonneg_E_zero}
   \encadre{ Q \geq 0 }_{E=0} .
\ee
This follows immediately from the result of Sec.~\ref{s:gek:th_motion}
stating that a necessary condition for $Q < 0$ is $a\neq 0$ and
$|E| > \sqrt{\mu^2 + L^2/a^2}$. Specializing this last inequality to $\mu=0$
and $E=0$, we get $0 > |L|$, which is impossible.

Besides, if $\Li$ has some part in $\M_{\rm I}$ (in the outer ergoregion thus)
or in $\M_{\rm III}$ (in the inner ergoregion thus),
the constraint (\ref{e:gek:future_directed_Carter})
reduces to $L<0$:
\be
    \Li \cap (\M_{\rm I} \cup \M_{\rm III}) \neq  \varnothing \quad \Longrightarrow \quad L < 0 .
\ee

The equations of geodesic motion expressed in terms of the Mino parameter $\lambda'$
[system (\ref{e:gek:eom_Mino})] simplify considerably for a geodesic $\Li$ with $\mu=0$ and $E=0$:
\begin{subequations}
\begin{align}
&  \derd{t}{\lambda'} = - \frac{2 a m L r}{\Delta} \\
&  \derd{r}{\lambda'} = \eps_r \sqrt{ R(r) }  \label{e:gik:eom_r_E_zero}\\
&  \derd{\th}{\lambda'} = \eps_\th \sqrt{\Theta(\th)}  \\
&  \derd{\ph}{\lambda'}  = \frac{L}{\Delta\sin^2\th} \left( \rho^2 - 2 m r \right) ,
\end{align}
\end{subequations}
with [cf. Eqs.~(\ref{e:gek:R_r_powers}) and (\ref{e:gek:Theta_Q})]:
\be \label{e:gik:R_E_zero}
    R(r) = - (Q+L^2) r^2 + 2m (Q+L^2) r - a^2 Q
\ee
\be \label{e:gik:Theta_E_zero}
    \Theta(\th) = Q - \frac{L^2}{\tan^2\th} .
\ee
We shall distinguish the subcase $Q=0$ from $Q\neq 0$.

\subsubsection{Case $Q=0$}

If the zero-energy null geodesic $\Li$ has a vanishing Carter constant $Q$, Eq.~(\ref{e:gik:Theta_E_zero}) reduces to $\Theta(\th) = -L^2 / \tan^2\th$,
so that the constraint $\Theta(\th) \geq 0$ [Eq.~(\ref{e:gek:Theta_non_neg})]
implies $L=0$ or $\th = \pi/2$ .

In the first case, the four constants of motion $\mu$, $E$, $L$ and $Q$ are
zero. By virtue of the result (\ref{e:gek:all_const_zero}), $\Li$
is nothing but a null geodesic generator of the event horizon $\Hor$ or of the
inner horizon $\Hor_{\rm in}$.

In the second case ($\th=\pi/2$), $\Li$ is confined to the equatorial plane.
If $L=0$, we are back to the first case: $\Li$ is null geodesic generator of $\Hor$ or
$\Hor_{\rm in}$ lying in the equatorial plane.
If $L\neq 0$, the radial motion of $\Li$ is governed by Eq.~(\ref{e:gik:eom_r_E_zero}) with the
expression (\ref{e:gik:R_E_zero}) of $R(r)$ reduced to
\be
    R(r) = L^2 r (2m - r) .
\ee
The constraint $R(r) \geq 0$ [Eq.~(\ref{e:gek:R_non_neg})] implies then
that the motion is within the range $0 \leq r \leq 2 m$, with $r=2m$
(the outer edge of the ergoregion in the equatorial plane [Eq.~(\ref{e:ker:r_ergo_p_eq})])
being a $r$-turning point, since it is a simple root of $R(r)$ (cf. Sec.~\ref{s:gek:turning_points}).
The other boundary of the radial motion, $r=0$, is the ring singularity.

We conclude that
\begin{greybox}
Any null geodesic with $E=0$ and $Q=0$
is either a null generator of one of the two Killing horizons
$\Hor$ or $\Hor_{\rm in}$ (in which case, it has $L=0$) or it has
$L \neq 0$, lies in the equatorial plane and terminates at the ring singularity,
after possibly a turning point at the outer ergosphere.
\end{greybox}

\subsubsection{Case $Q\neq 0$}

This case actually corresponds to $Q > 0$, since $Q<0$ is forbidden by
(\ref{e:gik:Q_nonneg_E_zero}).


\subsubsection{$\theta$-motion}

Specializing the general results of Sec.~\ref{s:gek:th_motion} to $\mu=0$ and $E=0$,
and taking into account $Q \geq 0$, we get
\begin{greybox}
\begin{itemize}
\item If $Q>0$, (i) for $L\neq 0$, $\Li$ oscillates symmetrically about the equatorial plane,
between two turning points, $\th_0$ and $\pi-\th_0$, such that $0<\th_0<\pi/2$;
(ii) for $L=0$, $\Li$
crosses repeatedly the rotation axis, with $\th$ taking all values in the
range $[0,\pi]$.
\item If $Q=0$, $\Li$ is stably confined to the equatorial plane
for $L \neq 0$;
for $L=0$, $\Li$ lies at a constant value $\th=\th_0\in[0,\pi]$.
\end{itemize}
\end{greybox}


\subsection{Equations of geodesic motion for $E\neq 0$}


\subsection{$\theta$-motion}

Specializing the general results of Sec.~\ref{s:gek:th_motion} to $\mu=0$, we
get
\begin{greybox}
\begin{itemize}
\item A null geodesic $\Li$ of Kerr spacetime cannot encounter the rotation axis unless it has $L=0$.
\item If $a=0$ (Schwarzschild limit) or $|E| \leq |L|/a$,
the Carter constant $Q$ is necessarily non-negative:
\be \label{e:gik:Q_nonnegative}
    Q \geq 0 .
\ee
\item The Carter constant $Q$ can take negative values only if
$a\neq 0$ and $|E| > |L|/a$; its range is then
limited from below:
\be
    Q \geq Q_{\rm min} = - \left( a |E| - |L| \right) ^2.
\ee
If $Q<0$, $\Li$ is called a \defin{vortical null geodesic}\index{vortical geodesic}; it
never encounters the equatorial plane.
\item If $Q>0$ and $L\not=0$, $\Li$ oscillates symmetrically about the equatorial plane,
between two turning points, $\th_0$ and $\pi-\th_0$, such that $0<\th_0<\pi/2$.
If $Q>0$ and $L=0$, $\Li$
crosses repeatedly the rotation axis, with $\th$ taking all values in the
range $[0,\pi]$.
\item If $Q=0$, $\Li$ is stably confined to the equatorial plane
for $a |E|  < |L|$ or $a |E|  = |L| \neq 0$;
for $a |E| > |L|$, $\Li$ either lies unstably in the equatorial
plane or approaches it asymptotically from one side, while for $L=0$ and ($a=0$ or $|E|=\mu$),
$\Li$ lies at a constant value $\th=\th_0\in[0,\pi]$.
\item If $Q_{\rm min} < Q < 0$, $\Li$ oscillates between two turning
points strictly located in the Northern hemisphere ($0<\th<\pi/2$) or in
the Southern hemisphere ($\pi/2<\th<\pi$) for $L\neq 0$, while for $L=0$,
$\Li$ oscillates about the rotation axis, without encoutering the equatorial
plane, having a turning point at $\th_0$ such that $0<\th_0<\pi/2$
(motion in the Northern hemisphere) or $\pi/2<\th_0<\pi$
(motion in the Southern hemisphere).
\item If $Q = Q_{\rm min}$, $\Li$ lies stably at a constant value $\th=\th_0$,
with $\th_0 = 0$ or $\pi$ (the rotation axis) for $L=0$ and $0<\th_0< \pi/2$
or $\pi/2<\th_0<\pi$ for $L\neq 0$.
\end{itemize}
\end{greybox}


\section{Principal null geodesics}

\section{Photon region}

\subsection{Spherical null geodesics}

\section{Images}
