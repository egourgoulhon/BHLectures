\chapter{Extremal Kerr black hole}
\label{s:exk}

\minitoc

\section{Introduction}

The Kerr solution of Einstein equation has been introduced in Sec.~\ref{s:ker:Kerr_solution};
it depends on two parameters: the mass $m > 0$ and the spin parameter
$a \geq 0$.
In Chaps.~\ref{s:ker}--\ref{s:gik}, we have considered the Kerr solution with $0<a<m$,
while the case $a=0$ (Schwarzschild solution) has been treated in Chaps.~\ref{s:sch}--\ref{s:max}.
Here we focus on the case $a=m$. As we going to see, it has many properties that are not
shared by the Kerr solution with $a<m$. In particular, the black hole event horizon is
degenerate, in the sense defined in Sec.~\ref{s:neh:classif_KH}: the surface gravity
$\kappa$ vanishes. Note that $a=m$ corresponds to the highest value of $a$
for which the Kerr solution corresponds to a black hole.
Indeed, for $a> m$, the Kerr metric is still an exact
solution of the vacuum Einstein equation, but it describes a \emph{naked singularity}\index{naked singularity} (cf. Sec.~\ref{s:max:naked_sing}):
the ring curvature singularity is not hidden by any black hole horizon to asymptotic observers.


%%%%%%%%%%%%%%%%%%%%%%%%%%%%%%%%%%%%%%%%%%%%%%%%%%%%%%%%%%%%%%%%%%%%%%%%%%%%%%%

\section{Definition and basic properties}

\subsection{The extremal Kerr solution}

Let us consider the manifold $\R^2\times\SS^2$ and describe it by
coordinates $(\ti, r, \th,\tph)$ such that $(\ti,r)$ cover $\R^2$
and $(\th,\tph)$ are standard spherical coordinates on $\SS^2$.
The \defin{extremal Kerr spacetime}\index{extremal!Kerr spacetime}\index{Kerr!extremal -- spacetime}
of mass $m>0$ is defined as the pair $(\M, \w{g})$ where the manifold $\M$ is the following open subset of $\R^2\times\SS^2$:
\be \label{e:exk:def_M}
 \M := \R^2\times\SS^2 \setminus \ring
\ee
with
\be \label{e:exk:def_ring}
    \ring := \left\{ p \in \R^2\times\SS^2,
        \quad r(p) = 0 \ \mbox{and}\ \th(p) = \frac{\pi}{2} \right\} ,
\ee
and the metric $\w{g}$ has the following expression in terms of the coordinates
$(\tilde{x}^\alpha) = (\ti, r, \th,\tph)$:
\be \label{e:exk:metric_Kerr_3p1}
    \encadre{
    \begin{array}{ll}
    \tilde{g}_{\mu\nu}\,  \D \tilde{x}^\mu \D \tilde{x}^\nu  = &
    \displaystyle - \left( 1 - \frac{2m r}{\rho^2} \right)  \D \ti^2
    + \frac{4m r}{\rho^2} \D\ti\, \D r
    - \frac{4 m^2  r \sin^2\th}{\rho^2} \,  \D \ti\, \D\tph \\[2ex]
    &\displaystyle  + \left( 1 + \frac{2m r}{\rho^2} \right) \D r^2
     - 2 m \left( 1 + \frac{2m r}{\rho^2} \right) \sin^2\th \, \D r\, \D \tph \\[2ex]
    & \displaystyle + \rho^2 \D \th^2
    + \left( r^2 + m^2 + \frac{2 m^3 r \sin^2\th}{\rho^2} \right)
    \sin^2\th \, \D \tph^2 ,
    \end{array}
    }
\ee
with
\be
    \rho^2 := r^2 + m^2\cos^2\th .
\ee
In this context, the coordinates $(\tilde{x}^\alpha) = (\ti, r, \th,\tph)$
are called the
\defin{3+1 Kerr coordinates}\index{3+1!Kerr coordinates}\index{Kerr!coordinates!3+1 --}
and we recognize in (\ref{e:exk:metric_Kerr_3p1}) the limit $a\to m$ of
expression (\ref{e:ker:metric_Kerr_3p1}) for the Kerr metric with $a< m$
in the 3+1 Kerr coordinates. The metric (\ref{e:exk:metric_Kerr_3p1}) is regular
in all $\M$, since the components $\tilde{g}_{\mu\nu}$ are singular only
for $\rho=0$, i.e. for $r=0$ and $\th=\pi/2$, which defines  the set $\ring$ that has precisely been excluded in
the definition (\ref{e:exk:def_M}) of $\M$.




\section{Maximal analytic extension}

\section{Near-horizon extremal Kerr metric}


