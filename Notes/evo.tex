\chapter{Evolution and thermodynamics of black holes}
\label{s:evo}

\minitoc

\section{Introduction}

This chapter is in a draft stage.

\section{Towards the first law of black hole dynamics}

\subsection{The mass variation formula}

Let us consider an initially isolated Kerr black hole, of parameters $(m,a)$,
that is perturbed by the arrival of some external body or some gravitational
radiation. After some transitory dynamical regime (e.g. absorption of the
external body and emission of gravitational waves), the black hole relaxes
to a new equilibrium configuration. According to the
no-hair theorem (Sec.~\ref{s:sta:no-hair}),
the final state has to be a Kerr black hole, of
parameters $(m+\D m, a+\D a)$ say.
Let us investigate how the black hole global properties evolve during the
process. More precisely, we would like to relate the change
in the Komar mass $M = m$ [Eq.~(\ref{e:ker:M_m})] to the change in
the area $A = 8 \pi m (m + \sqrt{m^2-a^2})$ [Eq.~(\ref{e:ker:A_a_m})]
and in the angular momentum $J = a m$ [Eq.~(\ref{e:ker:J_am})].

Rewriting Eq.~(\ref{e:ker:A_a_m}) as $A = 8 \pi (M^2 + \sqrt{M^4 - J^2})$
and differentiating, we get
\[
\frac{1}{8\pi} \, \D A =  2 M\,  \D M + \frac{2M^3}{\sqrt{M^4-J^2}}\, \D M
    - \frac{J}{\sqrt{M^4-J^2}}\, \D J ,
\]
or equivalently
\[
    \D M = \frac{1}{8\pi}
\underbrace{\frac{\sqrt{M^4-J^2}}{2M(M^2+\sqrt{M^4-J^2})}}_{\kappa} \, \D A
+ \underbrace{\frac{J}{2M(M^2+\sqrt{M^4-J^2})}}_{\Omega_{\Hor}} \, \D J  ,
\]
where the identifications of the black hole's surface gravity $\kappa$ and
rotation velocity $\Omega_{\Hor}$ result from Eqs.~(\ref{e:ker:kappa_m_a})
and (\ref{e:ker:def_OmegaH}) respectively. Hence we get
\be \label{e:evo:mass_variation}
    \encadre{ \D M = \frac{\kappa}{8\pi} \, \D A + \Omega_{\Hor} \, \D J } .
\ee
\begin{remark}
The mass variation formula (\ref{e:evo:mass_variation}) is \emph{not}
the mere differential of the Smarr formula (\ref{e:ker:Smarr}). Indeed, differentiating
the latter yields
\[
    \D M = \frac{\kappa}{4\pi} \, \D A +  \frac{A}{4\pi} \, \D \kappa
        + 2 \Omega_{\Hor} \, \D J  + 2 J \, \D\Omega_{\Hor} ,
\]
with $\D\kappa \neq 0$ and $\D\Omega_{\Hor}\neq 0$ in the process
that makes the Kerr black hole evolve from $(m,a)$ to $(m+\D m, a+\D a)$.
If one were (wrongly) assuming that $\D\kappa = 0$ and $\D\Omega_{\Hor} = 0$, one would
be wrong by a factor of 2 in recovering the right-hand side of
Eq.~(\ref{e:evo:mass_variation}). Of course, if ones
expresses properly $\D\kappa$ and $\D\Omega_{\Hor}$ in terms of
$\D A$, $\D J$ and $\D M$ via expressions (\ref{e:ker:kappa_m_a})
and (\ref{e:ker:def_OmegaH}), one recovers Eq.~(\ref{e:evo:mass_variation}).
\end{remark}

\begin{remark}
\label{r:evo:general_mass_variation}
The mass variation formula (\ref{e:evo:mass_variation}) can be derived in
a more general framework, without assuming that it describes changes between
two nearby Kerr solutions \cite{BardeCH73,Carte79}. It even holds
for gravitation theories other than general relativity, more specifically
for any theory based on a diffeomorphism-covariant lagrangian; see Wald's review
article~\cite{Wald01} for details.
\end{remark}

\begin{hist}
The mass-variation formula (\ref{e:evo:mass_variation}) has been first
derived for a Kerr black hole (as here) by Jacob Bekenstein\index[pers]{Bekenstein, J.D.}
in 1973 \cite{Beken73a}. Its extension to generic stationary black hole configurations (cf. Remark~\ref{r:evo:general_mass_variation} above) has been
obtained by James Bardeen\index[pers]{Bardeen, J.M.}, Brandon Carter\index[pers]{Carter, B.}
and Stephen Hawking\index[pers]{Hawking, S.W.} in 1973 \cite{BardeCH73}.
\end{hist}



\subsection{A first law?}

At this stage, it would be premature to call formula~(\ref{e:evo:mass_variation})
the first law of black hole dynamics by analogy to the first law of thermodynamics
$\D E = T \D S - P \D V$. One can reasonably interpret $\D M$
as some energy variation and the term
$\Omega_{\Hor} \, \D J$ as the work\footnote{In Newtonian mechanics, the
work done by a torque $\tau$ during the time $\D t$ by which a body is rotating by $\D\ph$
is $\D W = \tau \, \D\ph$. Given that $\tau := \D J/\D t$, one gets $\D W = \Omega \, \D J$, where
$\Omega = \D\ph/\D t$ is the body's angular velocity.} performed by the torque
that is changing by the amount $\D J$ the angular momentum of a body rotating
at the angular velocity $\Omega_{\Hor}$. However, we do not have any argument yet to
to identify the term $(\kappa/8\pi) \, \D A$ with the classical heat exchange term $T\D S$.
For this, we need first to identify the entropy $S$ with the black hole area $A$.
This will be performed in the next section.


%%%%%%%%%%%%%%%%%%%%%%%%%%%%%%%%%%%%%%%%%%%%%%%%%%%%%%%%%%%%%%%%%%%%%%%%%%%%%%%

\section{Evolution of the area of a black hole}

\subsection{Hawking's area theorem}

The first step towards the area theorem is:

\begin{prop}[positive expansion of a black hole horizon]
Let $(\M,\w{g})$ be a $n$-dimensional spacetime containing a black hole
of event horizon $\Hor$. If the Ricci tensor $\w{R}$ obeys the null
energy condition (\ref{e:neh:null_energy_cond}) on $\Hor$, i.e. if
$\w{R}(\wl, \wl) \geq 0$ for any null vector $\wl$ on $\Hor$, then the
expansion $\theta_{(\wl)}$ of $\Hor$ along any future-directed null
normal $\wl$, as defined in Sec.~\ref{s:def:def_expansion}, is positive or zero:
\be \label{e:evo:theta_positive}
    \theta_{(\wl)} \geq 0 .
\ee
\end{prop}
\begin{proof}
Let $\wl$ be a future-directed null normal vector field of $\Hor$.
$\wl$ is necessarily tangent to the null geodesic geodesic generators of $\Hor$
(Property~\ref{p:def:null_geod_generators}) and is thus a pregeodesic
vector, i.e. it obeys Eq.~(\ref{e:def:wl_geod_kappa}):
$\wnab_{\wl}\, \wl = \kappa \, \wl $.
If $\wl$ is not geodesic, it is always possible to rescale it
to $\wl' = \alpha \wl$ with $\alpha > 0$ so that $\wl'$ is a geodesic vector field:
$\wnab_{\wl'}\, \wl' = 0$ [Eq.~(\ref{e:def:wlp_geod})].
We have then $\theta_{(\wl')} = \alpha \theta_{(\wl)}$ [cf. Eq.~(\ref{e:def:rescale_lambda})],
so that $\theta_{(\wl)} \geq 0 \iff \theta_{(\wl')} \geq 0$.
Accordingly, in order to prove (\ref{e:evo:theta_positive}), there is no loss of generality in assuming that $\wl$ is a geodesic vector field, i.e. has a vanishing non-affinity coefficient: $\kappa=0$.
Let then $\lambda$ be the affine parameter of the geodesic generators of $\Hor$ associated to $\wl$:
$\wl = \D\w{x}/\D\lambda$ along any geodesic generator $\Li$. The evolution of
$\theta_{(\wl)}$ along $\Li$
is $\D\theta_{(\wl)}/\D\lambda = \wnab_{\el}\,  \theta_{(\wl)}$ and is given by
the null Raychaudhuri equation (\ref{e:def:null_Raychaud_Ricci}).
Owing to $\kappa=0$, it simplifies to
\[
    \derd{\theta_{(\wl)}}{\lambda}  =
        - \frac{1}{n-2} \, \theta_{(\wl)}^2
        - \underbrace{\sigma_{ab} \sigma^{ab}}_{\geq 0}
        - \underbrace{\w{R}(\wl, \wl)}_{\geq 0} ,
\]
where $\sigma_{ab} \sigma^{ab} \geq 0$ has been established in Sec.~\ref{s:neh:NEH_Theta_zero}
[Eq.~(\ref{e:neh:sigma_square})] and $\w{R}(\wl, \wl) \geq 0$ by virtue of the null
energy condition on $\Hor$. We have then
\be \label{e:evo:der_theta_lower}
    \derd{\theta_{(\wl)}}{\lambda}  \leq
        - \frac{1}{n-2} \, \theta_{(\wl)}^2  .
\ee
Let us assume that at some point $p\in\Li\cap\Hor$, $\theta_{(\wl)} = \theta_0 < 0$.
Without any loss of generality, we may assume that $\lambda(p) = 0$.
Equation~(\ref{e:evo:der_theta_lower}) implies
\be \label{e:evo:theta_lower_theta_bar}
 \forall \lambda\geq 0,\quad \theta_{(\wl)}(\lambda) \leq \bar\theta(\lambda) ,
\ee
where $\bar\theta(\lambda)$ obeys
\[
    \frac{\D\bar\theta}{\D\lambda} = -  \frac{1}{2}  \bar\theta^2
    \qand \bar\theta(0) = \theta_0 .
\]
The solution of the above differential equation is
\[
\bar\theta(\lambda) = \frac{\theta_0}{1 + \theta_0\lambda/2} .
\]
It follows that $\bar\theta \to -\infty$ as $\lambda \to -2/\theta_0 > 0$.
The inequality~(\ref{e:evo:theta_lower_theta_bar}) then implies
$\theta_{(\wl)} \to -\infty$ as $\lambda \to \lambda_*$
with $0 < \lambda_* \leq -2/\theta_0$. Hence the point $p_*\in\Li$ of parameter $\lambda_*$ is a \emph{focusing point},
i.e. a point where neighbouring null geodesic generators of $\Hor$ intersect.
But according to Property~\ref{p:glo:prop3} of black hole event horizons (cf. Sec.~\ref{s:glo:properties_H}), this can happen
only if $p_*$ is a point at which the null geodesic $\Li$ enters $\Hor$; however,
this situation is excluded since $\lambda_* > 0$ implies that $p_*$ lies in the
future of $p$, where $\Li$ was already in $\Hor$. Hence the hypothesis
$\theta_0 < 0$ has lead us to a contradiction. It follows that $\theta_0 \geq 0$,
i.e. at any point $p\in\Hor$,  $\theta_{(\wl)} \geq 0$.
\end{proof}

\subsection{The second law of black hole thermodynamics}


