\chapter{Evolution and thermodynamics of black holes}
\label{s:evo}

\minitoc

\section{Introduction}

This chapter is in a draft stage.

\section{Towards the first law of black hole dynamics}

\subsection{The mass variation formula}

Let us consider an initially isolated Kerr black hole, of parameters $(m,a)$,
that is perturbed by the arrival of some external body or some gravitational
radiation. After some transitory dynamical regime (e.g. absorption of the
external body and emission of gravitational waves), the black hole relaxes
to a new equilibrium configuration. According to the
no-hair theorem (Sec.~\ref{s:sta:no-hair}),
the final state has to be a Kerr black hole, of
parameters $(m+\D m, a+\D a)$ say.
Let us investigate how the black hole global properties evolve during the
process. More precisely, we would like to relate the change
in the Komar mass $M = m$ [Eq.~(\ref{e:ker:M_m})] to the change in
the area $A = 8 \pi m (m + \sqrt{m^2-a^2})$ [Eq.~(\ref{e:ker:A_a_m})]
and in the angular momentum $J = a m$ [Eq.~(\ref{e:ker:J_am})].

Rewriting Eq.~(\ref{e:ker:A_a_m}) as $A = 8 \pi (M^2 + \sqrt{M^4 - J^2})$
and differentiating, we get
\[
\frac{1}{8\pi} \, \D A =  2 M\,  \D M + \frac{2M^3}{\sqrt{M^4-J^2}}\, \D M
    - \frac{J}{\sqrt{M^4-J^2}}\, \D J ,
\]
or equivalently
\[
    \D M = \frac{1}{8\pi}
\underbrace{\frac{\sqrt{M^4-J^2}}{2M(M^2+\sqrt{M^4-J^2})}}_{\kappa} \, \D A
+ \underbrace{\frac{J}{2M(M^2+\sqrt{M^4-J^2})}}_{\Omega_{\Hor}} \, \D J  ,
\]
where the identifications of the black hole's surface gravity $\kappa$ and
rotation velocity $\Omega_{\Hor}$ result from Eqs.~(\ref{e:ker:kappa_m_a})
and (\ref{e:ker:def_OmegaH}) respectively. Hence we get
\be \label{e:evo:mass_variation}
    \encadre{ \D M = \frac{\kappa}{8\pi} \, \D A + \Omega_{\Hor} \, \D J } .
\ee
\begin{remark}
The mass variation formula (\ref{e:evo:mass_variation}) is \emph{not}
the mere differential of the Smarr formula (\ref{e:ker:Smarr}). Indeed, differentiating
the latter yields
\[
    \D M = \frac{\kappa}{4\pi} \, \D A +  \frac{A}{4\pi} \, \D \kappa
        + 2 \Omega_{\Hor} \, \D J  + 2 J \, \D\Omega_{\Hor} ,
\]
with $\D\kappa \neq 0$ and $\D\Omega_{\Hor}\neq 0$ in the process
that makes the Kerr black hole evolve from $(m,a)$ to $(m+\D m, a+\D a)$.
If one were (wrongly) assuming that $\D\kappa = 0$ and $\D\Omega_{\Hor} = 0$, one would
be wrong by a factor of 2 in recovering the right-hand side of
Eq.~(\ref{e:evo:mass_variation}). Of course, if ones
expresses properly $\D\kappa$ and $\D\Omega_{\Hor}$ in terms of
$\D A$, $\D J$ and $\D M$ via expressions (\ref{e:ker:kappa_m_a})
and (\ref{e:ker:def_OmegaH}), one recovers Eq.~(\ref{e:evo:mass_variation}).
\end{remark}

\begin{remark}
\label{r:evo:general_mass_variation}
The mass variation formula (\ref{e:evo:mass_variation}) can be derived in
a more general framework, without assuming that it describes changes between
two nearby Kerr solutions \cite{BardeCH73,Carte79}. It even holds
for gravitation theories other than general relativity, more specifically
for any theory based on a diffeomorphism-covariant lagrangian; see Wald's review
article~\cite{Wald01} for details.
\end{remark}

\begin{hist}
The mass-variation formula (\ref{e:evo:mass_variation}) has been first
derived for a Kerr black hole (as here) by Jacob Bekenstein\index[pers]{Bekenstein, J.D.}
in 1973 \cite{Beken73a}. Its extension to generic stationary black hole configurations (cf. Remark~\ref{r:evo:general_mass_variation} above) has been
obtained by James Bardeen\index[pers]{Bardeen, J.M.}, Brandon Carter\index[pers]{Carter, B.}
and Stephen Hawking\index[pers]{Hawking, S.W.} in 1973 \cite{BardeCH73}
\end{hist}



\subsection{A first law?}

At this stage, it would be quite premature to call formula~(\ref{e:evo:mass_variation})
the first law of black hole dynamics by analogy to the first law of thermodynamics
$\D E = T \D S - P \D V$. If one can reasonably interpret the left-hand side $\D M$
as some energy variation and the term
$\Omega_{\Hor} \, \D J$ in the right-hand side as the change of some rotational
kinetic energy, it is not obvious at all to identify the term $(\kappa/4\pi) \, \D A$
with $T\D S$.




