\chapter{Mathematical framework: pseudo-Riemannian manifolds}

\section{Introduction}

The mathematical language of general relativity is mostly differential geometry.
We recall in this chapter basic definitions and results in this field, which we will use
throughout the text. The reader who has some knowledge of general relativity should be familiar with most of them. We recall them here to make the text fairly self-contained and also to provide definitions with sufficient generality, not limited to the dimension 4 --- the standard spacetime dimension. Indeed we will manipulate manifolds whose dimension differs from 4, such as hypersurfaces (the black hole event horizon) or 2-dimensional surfaces.
In the same spirit, we do not stick to Lorentzian metrics
(such as the spacetime one) but discuss pseudo-Riemannian metrics, which
encompass both Lorentzian metrics and Riemannian ones.
Accordingly, in this chapter, $\M$ denotes a generic manifold of any dimension
and $\w{g}$ a pseudo-Riemannian metric on $\M$. In the subsequent chapters,
the symbol $\M$ will be restricted to the spacetime manifold and the symbol $\w{g}$
to a Lorentzian metric on $\M$.

This chapter is not intended to a be a lecture on differential geometry, but
a collection of basic definitions and useful results. In particular,
contrary to the other chapters, we state many results without proofs,
referring the reader to classical textbooks on the topic
\cite{Berge03,Choqu09,ChoquDD77,Eschr11,Strau04,Wald84}.

\section{Differentiable manifolds}

\subsection{Notion of manifold}

Given an integer $n\geq 1$, a \defin{manifold of dimension $n$}\index{manifold}\index{dimension of a manifold} is a topological space $\M$ obeying the following properties:
\begin{enumerate}
\item $\M$ is a \defin{separated space}\index{separated space} (also called \defin{Hausdorff space}\index{Hausdorff space}): any two distinct points of $\M$
admit disjoint open neighbourhoods.
\item $\M$ has a \defin{countable base}\index{countable base}\footnote{In the language of topology, one says that $\M$ is a \emph{second-countable space}.}:
there exists a countable family
$(\mathcal{U}_k)_{k\in\mathbb{N}}$ of open sets of $\M$ such that any open set of $\M$ can be written as the union (possibly infinite) of some members of the above family.
\item Around each point of $\M$, there exists a neighbourhood which is
homeomorphic to an open subset of $\R^n$.
\end{enumerate}
Property 1 excludes manifolds with ``forks'' and is very reasonable from a physical point of view: it allows to distinguish between two points even after a small perturbation.
Property~2 excludes ``too large'' manifolds; in particular it permits setting
up the theory of integration on manifolds. It also
allows for a differentiable manifold of dimension $n$ to be embedded smoothly into the Euclidean space $\R^{2n}$
(Whitney theorem\index{Whitney theorem}).
Property~3 expresses the essence of a manifold: it means that, locally, one can label the points of $\M$ in a
continuous way by $n$ real numbers $(x^\alpha)_{\alpha\in\{0,\ldots,n-1\}}$,
which are called \defin{coordinates}\index{coordinate} (cf. Fig.~\ref{f:bas:manifold}).
More precisely, given an open subset $\mathcal{U}\subset\M$, a
\defin{coordinate system}\index{coordinate!system} or \defin{chart}\index{chart}
on $\mathcal{U}$ is a homeomorphism\footnote{Let us recall that a  \defin{homeomorphism}\index{homeomorphism} between two topological spaces
(here $\mathcal{U}$ and $\Phi(\mathcal{U})$) is a one-to-one map $\Phi$ such
that both $\Phi$ and $\Phi^{-1}$ are continuous.}
\be
  	\begin{array}{rccl}
	\Phi: & \mathcal{U}\subset \M & \longrightarrow &
				\Phi(\mathcal{U})\subset\R^n \\
		& p & \longmapsto & (x^0, \ldots, x^{n-1}) .
	\end{array}
\ee
\begin{remark}
In relativity, it is customary to label the $n$ coordinates by an index ranging from
$0$ to $n-1$. Actually, this convention is mostly used when $\M$ is the spacetime manifold ($n=4$ in standard general relativity). The computer-oriented reader will have noticed the similarity
with the index ranging of arrays in the C/C++ or Python programming languages.
\end{remark}

