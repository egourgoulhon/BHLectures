\chapter{Schwarzschild black hole}
\label{s:sch}
\index{Schwarzschild!black hole}

\minitoc

\section{Introduction}

After having discussed stationary black holes in Chap.~\ref{s:sta},
we examine here the simplest of them: the Schwarzschild black hole.
Let us recall the prime importance of the Schwarzschild black hole
in general relativity stems from the no-hair theorem (Sec.~\ref{s:sta:no-hair}),
which implies that any non-rotating black hole in an asymptotically flat
4-dimensional spacetime must a Schwarzschild black hole.

\section{Schwarzschild solution}

\subsection{Vacuum Einstein equation with a cosmological constant}

Let us search for a static and spherically symmetric solution of the
Einstein equation (\ref{e:bas:Einstein_eq}) in a vacuum
4-dimensional spacetime $(\M,\w{g})$ with some arbitrary cosmological constant
$\Lambda$. Setting $\w{T}=0$ in Eq.~(\ref{e:bas:Einstein_eq}) shows
that the equation to solve is
\be \label{e:sch:vac_Einstein_eq}
     \w{R} + \left(\Lambda - \frac{1}{2}\, R\right) \w{g} = 0 ,
\ee
$\w{R}$ being the Ricci tensor of $\w{g}$ and $R:=g^{\mu\nu} R_{\mu\nu}$ its
trace with respect to $\w{g}$, i.e. the so-called Ricci scalar
(cf. Sec.~\ref{s:bas:Ricci_tensor} in Appendix~\ref{s:bas}).
Let us first note that Eq.~(\ref{e:sch:vac_Einstein_eq}) implies a
constraint on the Ricci scalar. Indeed the trace of Eq.~(\ref{e:sch:vac_Einstein_eq})
with respect to $\w{g}$ is
\[
    R + \left(\Lambda - \frac{1}{2}\, R\right) \times 4 = 0 ,
\]
hence
\be \label{e:sch:R_4Lamb}
    \encadre{R = 4\Lambda} .
\ee
In particular the Ricci scalar is constant.
Inserting this value back into (\ref{e:sch:vac_Einstein_eq}), we get
\be \label{e:sch:vac_Einstein_eq_Lamb}
    \encadre{ \w{R} = \Lambda \, \w{g} } .
\ee
Since this equation yields (\ref{e:sch:R_4Lamb}) as well, we conclude
that it is equivalent to (\ref{e:sch:vac_Einstein_eq}).

\subsection{Static and spherically symmetric metric}

Let us assume that the spacetime $(\M,\w{g})$
is \defin{static}\index{static spacetime}, i.e. that it is
admits a Killing vector field $\w{\xi}$ that is timelike and
hypersurface-orthogonal (cf. Sec.~\ref{s:sta:staticity_thm}).
We may then foliate $\M$ by a 1-parameter family of hypersurfaces
$\left(\Sigma_t\right)_{t\in\R}$, such that $\w{\xi}$ is normal to
all $\Sigma_t$'s and $t$ is a parameter associated to $\w{\xi}$:
\be \label{e:sch:xi_t}
    \w{\xi}(t) = 1
\ee
or equivalently,
\[
    \langle \dd t , \w{\xi} \rangle = 1.
\]

In addition to be static, we assume that $(\M,\w{g})$ is \defin{spherically symmetric},
i.e. that it is invariant under the action of the rotation group $\mathrm{SO}(3)$,
whose orbits are spacelike 2-spheres (cf. Sec.~\ref{s:neh:symmetries}).
Let $\Sp$ be a generic orbit 2-sphere. The static Killing vector field $\w{\xi}$
must be orthogonal to $\Sp$, otherwise the orthogonal projection of $\w{\xi}$
onto $\Sp$ would define some privileged directions on $\Sp$, which is incompatible
with spherical symmetry. The orthogonality of $\w{\xi}$ and $\Sp$ implies
that $\Sp\subset\Sigma_t$. Let $(x^a)=(\th,\ph)$ be spherical coordinates on
$\Sp$. The metric $\w{q}$ induced by $\w{g}$ on $\Sp$ is given by
\be
    q_{ab}\, \D x^a\, \D x^b = r^2 \left( \D\th^2 + \sin^2\th\, \D\ph^2 \right) .
\ee
The coefficient $r^2$ in front of the standard spherical element must be
constant over $\Sp$, by virtue of spherical symmetry. The area of $\Sp$ is
then $A=4\pi r^2$. For this reason, $r$ is called the \defin{areal radius}\index{areal!radius}
of $\Sp$. Letting $\Sp$ vary, $r$ can be considered as a scalar field on
$\M$. If $\dd r \not = 0$, we may use it a coordinate. Since $\Sp\subset \Sigma_t$,
$(r,\th,\ph)$ is a coordinate system on each hypersurface $\Sigma_t$.
The set $(t,r,\th,\ph)$,
where $t$ is adapted to $\w{\xi}$ thanks to (\ref{e:sch:xi_t}) is then a
coordinate system and, by construction, the expression of the metric tensor
with respect to this system is
\be \label{e:sch:g_AB}
    g_{\mu\nu}\, \D x^\mu \, \D x^\nu = -A(r)\, \D t + B(r)\, \D r +
        r^2 \left( \D\th^2 + \sin^2\th\, \D\ph^2 \right) .
\ee
Note that in this coordinate system
\be
    \w{\xi} = \wpar_t
\ee
and that $g_{tt} = -A(r)$ and $g_{rr} = B(r)$ do not depend on $t$
as a result of the spacetime stationarity, while
$g_{tr} = g_{t\th} = g_{t\ph} = 0$ expresses the orthogonality of $\w{\xi}$
and $\Sigma_t$, i.e. the spacetime staticity.
The coordinates $(t,r,\th,\ph)$ are called \defin{areal coordinates}\index{areal!coordinates},
reflecting the fact that $r$ is the areal radius.

The Christoffel symbols of the metric (\ref{e:sch:g_AB}) with respect to the
areal coordinates are (cf. Sec.~\ref{s:sam:Kottler_solution} for the computation):
\be
\begin{array}{l}
\displaystyle  \Gamma^t_{\ \, tr} = \Gamma^t_{\ \, rt} = \frac{1}{2A}\derd{A}{r}\qquad
\Gamma^r_{\ \, tt} = \frac{1}{2B}\derd{A}{r} \qquad
\Gamma^r_{\ \, rr} = \frac{1}{2B}\derd{B}{r} \qquad
\Gamma^r_{\ \, \th\th} = -\frac{r}{B} \\[2ex]
\displaystyle  \Gamma^r_{\ \, \ph\ph} = -\frac{r\sin^2\th}{B} \qquad
\Gamma^\th_{\ \, r\th} = \Gamma^\th_{\ \, \th r} = \frac{1}{r} \qquad
\Gamma^\th_{\ \, \ph\ph} = -\sin\th\cos\th \\[2ex]
\displaystyle \Gamma^\ph_{\ \, r\ph} = \Gamma^\ph_{\ \, \ph r} = \frac{1}{r} \qquad
\Gamma^\ph_{\ \, \th\ph} = \Gamma^\ph_{\ \, \ph\th} = \frac{1}{\tan\th} ,
\end{array}
\ee
the Christoffel symbols not listed above being zero.

The $tt$ component of the Einstein equation (\ref{e:sch:vac_Einstein_eq})
leads to (cf. Sec.~\ref{s:sam:Kottler_solution} for the computation)
\be \label{e:sch:EE_tt}
        r \derd{B}{r} - B + (1 - \Lambda r^2) B^2 = 0 ,
\ee
while the $rr$ component leads to
\be \label{e:sch:EE_rr}
        r \derd{A}{r} + A - (1 - \Lambda r^2) AB = 0 .
\ee
Finally, the $\th\th$ and $\ph\ph$ components lead to the same equation:
\be
    2  \frac{\D^2 A}{\D r^2} + \frac{2}{r} \derd{A}{r}
        - \frac{1}{B} \left( \derd{A}{r} + \frac{2A}{r} \right) \derd{B}{r}
        - \frac{1}{A} \left( \derd{A}{r} \right) ^2
        + 4 \Lambda  A B  = 0 .
\ee
All the other components of the Einstein equation (\ref{e:sch:vac_Einstein_eq})
are identically zero.

Adding Eq.~(\ref{e:sch:EE_tt}) multiplied by $A$ to
Eq.~(\ref{e:sch:EE_rr}) multiplied by $B$ yields
\[
    B \derd{A}{r} + A \derd{B}{r} = \derd{}{r}(AB) = 0 .
\]
The solution of this equation is obviously $A(r)B(r) = C$, where $C$ is a constant.
Without any loss of generality, we may choose $C=1$. Indeed, substituting
$C/B(r)$ for $A(r)$ in Eq.~(\ref{e:sch:g_AB}) results in
\[
    g_{\mu\nu}\, \D x^\mu \, \D x^\nu = -\frac{C}{B(r)}\, \D t + B(r)\, \D r +
        r^2 \left( \D\th^2 + \sin^2\th\, \D\ph^2 \right) .
\]
Assuming $C>0$, the change of variable $t' = \sqrt{C} t$, which is equivalent
of changing the stationary Killing vector from $\w{\xi}$ to
$\w{\xi}'=  1/\sqrt{C}\, \w{\xi}$,
yields
\[
    g_{\mu\nu}\, \D x^\mu \, \D x^\nu = -\frac{1}{B(r)}\, \D t' + B(r)\, \D r +
        r^2 \left( \D\th^2 + \sin^2\th\, \D\ph^2 \right) ,
\]
which is exactly the solution corresponding to $C=1$. Hence from now on,
we set $C=1$, i.e.
\be
    B(r) = \frac{1}{A(r)} .
\ee
Substituting this expression in Eq.~(\ref{e:sch:EE_rr}) yields an ordinary
differential equation for $A(r)$:
\[
    r \derd{A}{r} + A - 1 + \Lambda r^2 = 0 ,
\]
the solution of which is
\be
    A(r) = 1 - \frac{2 m}{r} - \frac{\Lambda}{3} \,  r^2 ,
\ee
where $m$ is a constant.
The general static and spherically symmetric solution of the vacuum
Einstein equation (\ref{e:sch:vac_Einstein_eq}) is therefore
\be \label{e:sch:Kottler_metric}
    \encadre{
        g_{\mu\nu}\, \D x^\mu \, \D x^\nu =
            -\left( 1 - \frac{2 m}{r} - \frac{\Lambda}{3} \,  r^2\right)\, \D t
            + \left( 1 - \frac{2 m}{r} - \frac{\Lambda}{3} \,  r^2\right) ^{-1}\, \D r+
        r^2 \left( \D\th^2 + \sin^2\th\, \D\ph^2 \right) }.
\ee
It is called the \defin{Kottler metric}\index{Kottler metric}. The particular case $\Lambda=0$
is called the \defin{Schwarzschild metric}\index{Schwarzschild!metric}. If $\Lambda>0$,
(\ref{e:sch:Kottler_metric}) is called the
\defin{Schwarzschild-de Sitter metric}\index{Schwarzschild!de Sitter metric},
often abridged as \defin{Schwarzschild-dS metric}, while if $\Lambda<0$, it
is called the \defin{Schwarzschild-anti-de Sitter metric}\index{Schwarzschild!anti-de Sitter metric},
often abridged as \defin{Schwarzschild-AdS metric}\index{Schwarzschild!AdS metric}.

In the rest of this chapter, we will focuss on the Schwarzschild metric,
i.e. the version $\Lambda=0$ of Eq.~(\ref{e:sch:Kottler_metric}):
\be \label{e:sch:Schwarz_metric_SD}
    \encadre{
        g_{\mu\nu}\, \D x^\mu \, \D x^\nu =
            -\left( 1 - \frac{2 m}{r} \right)\, \D t
            + \left( 1 - \frac{2 m}{r} \right) ^{-1}\, \D r +
        r^2 \left( \D\th^2 + \sin^2\th\, \D\ph^2 \right) }.
\ee

\begin{hist}
Karl Schwarzchild \cite{Schwa1916},  Johannes Droste \cite{Drost1917},
Friedrich Kottler \cite{Kottl1918}, Hermann Weyl \cite{Weyl1919}.
\end{hist}


\section{Maximal extension}

\begin{figure}
\centerline{\includegraphics[height=0.37\textheight]{sch_coord_schwarz.pdf}\qquad
\includegraphics[height=0.37\textheight]{sch_kruskal_diag.pdf}}
\caption[]{\label{f:sch:kruskal_diag} \footnotesize
Schwarzschild spacetime depicted in Schwarzschild-Droste coordinates $(t,r)$
(left) and in Kruskal-Szekeres coordinates $(T,X)$ (right). In both figures,
green (resp. red) solid curves denote the hypersurfaces $t=\mathrm{const}$
in Region~I (resp. II), while green (resp. red) dashed curves
denote the hypersurfaces $r=\mathrm{const}$ in Region~I (resp. II).
The future and past event horizons are marked by thick black lines, while the
singularity at $r=0$ is depicted in orange. Regions III and IV are depicted
in grey and pink respectively. Note that the left figure covers only Regions I and II.}
\end{figure}

\begin{figure}
\centerline{\includegraphics[width=0.9\textwidth]{sch_carter-penrose.pdf}}
\caption[]{\label{f:sch:sch_carter-penrose} \footnotesize
Schwarzschild spacetime depicted in Carter-Penrose coordinates $(\tilde{T},\tilde{X})$; the color code
is the same as in Fig.~\ref{f:sch:kruskal_diag}.
As Fig.~\ref{f:sch:kruskal_diag}, this figure has been produced with
SageManifolds (cf. Appendix~\ref{s:sam}).}
\end{figure}

