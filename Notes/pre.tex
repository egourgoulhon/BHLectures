\chapter*{Preface}

These notes correspond to lectures given
\begin{itemize}
\item at \emph{Institut d'Astrophysique de Paris} (France) in March-April 2016, within the
framework of the \emph{IAP Advanced Lectures}:\\
{\footnotesize\url{https://www.iap.fr/vie_scientifique/cours/cours.php?nom=cours_iap&annee=2016}}
\item at the \emph{Centre for Cosmology, Particle Physics and Phenomenology} in Louvain-la-Neuve
(Belgium) in November 2016, within the framework of the \emph{Chaire Georges Lemaître}:\\
{\footnotesize \url{https://uclouvain.be/fr/instituts-recherche/irmp/chaire-georges-lemaitre-2016.html}}
\item at the
\emph{Bogoliubov Laboratory of Theoretical Physics}, in Dubna (Russia) in May 2017,
within the framework of the \emph{Dubna International Advanced School of Theoretical Physics}:\\
{\footnotesize\url{http://www.jinr.ru/posts/lecture-course-geometry-and-physics-of-black-holes/}}
\item at the Summer School \emph{Gravitational Waves 2018}, taking place at Les Houches
(France) in July 2018:\\
{\footnotesize\url{https://www.lkb.upmc.fr/gravitationalwaves2018/}}
\item remotely at the \emph{School on Black Holes and Gravitational Waves}
organized at the
\emph{Centre for Strings, Gravitation and Cosmology} of the
\emph{Indian Institute of Technology Madras}, Chennai (India) in January 2022:\\
{\footnotesize\url{https://physics.iitm.ac.in/~csgc/events/sbhgw}}
\item at the \emph{École Normale Supérieure}, Paris (France) in May-June 2023, as
part of the PSL graduate programs in Physics and in Astrophysics:\\
{\footnotesize\url{https://relativite.obspm.fr/blackholes/paris23}}
\item at the \emph{Albert Einstein Institute}, Potsdam (Germany) in December 2023:\\
{\footnotesize\url{https://relativite.obspm.fr/blackholes/aei23}}
\item at the \emph{Institut Henri Poincaré}, Paris (France) in March 2024, within the
program \emph{Quantum and classical fields interacting with geometry}:\\
{\footnotesize\url{https://relativite.obspm.fr/blackholes/ihp24}}
\end{itemize}

\vspace{2ex}

In complement to these notes, one may recommend various monographs devoted to black holes:
O'Neill (1995) \cite{ONeil95}, Heusler (1996) \cite{Heusl96}, Frolov \& Novikov (1998) \cite{FroloN98},
Poisson (2004) \cite{Poiss04}, Frolov \& Zelnikov (2011) \cite{FroloZ11}, Bambi (2017) \cite{Bambi17},
Chru\'sciel (2020) \cite{Chrus20}, Grumiller \& Sheikh-Jabbari (2022) \cite{GrumiS22}
and King (2023) \cite{King23},
as well as review articles by
Carter (1987) \cite{Carte87}, Wald (2001) \cite{Wald01},
Chru\'sciel (2002, 2005) \cite{Chrus02, Chrus05} and Chru\'sciel, Lopes Costa \& Heusler (2012) \cite{ChrusLH12}.
In addition, let us point out other lecture notes on black holes:
Hawking (1994) \cite{Hawki94,HawkiP15}, Townsend (1997) \cite{Towns97},
Compère (2006, 2019) \cite{Compe06,Compe19}, Dafermos and Rodnianski (2008) \cite{DaferR13},
Deruelle (2009) \cite{Derue09}, Andersson, Bäckdahl \& Blue (2016) \cite{AnderBB18}
and Reall (2020) \cite{Reall20}.

The history of black holes in theoretical physics and astrophysics is
very rich and fascinating. It is however not discussed here, except in some
small historical notes. The interested
reader is referred to Nathalie Deruelle's lectures \cite{Derue09}, to Kip Thorne's
book \cite{Thorn94}, to Carter's article \cite{Carte06}
and to Jean Eisenstaedt's articles \cite{Eisen82,Eisen93}.


The web pages associated to these notes are
\begin{center}
\url{https://relativite.obspm.fr/blackholes}
\end{center}
They contain supplementary material, such as the SageMath notebooks presented in
Appendix~\ref{s:sam}.

\vspace{2ex}

I warmly thank Cyril Pitrou for having organized the Paris 2016 lectures,
Fabio Maltoni and Christophe Ringeval for the Louvain-la-Neuve ones,
Anastasia Golubtsova and Irina Pirozhenko for the Dubna ones,
Bruce Allen, Marie-Anne Bizouard, Nelson Christensen and Pierre-François Cohadon for the Les Houches ones,
Chandra Kant Mishra for the Chennai ones,
Jean-François Allemand for the Paris 2023 and 2025 ones, Masaru Shibata and Karim Van Aeslt for the Potsdam ones and Dietrich Häfner, Frédéric Hélein and Michał Wrochna for the Paris 2024 ones.

Besides, I am deeply indebted to
Imène Belahcene, Jack Borthwick, Brandon Carter, Marc Casals,
Udit Narayan Chowdhury, Stéphane Collion, Xiangyang Chen,
Sumit Dey, Jean Eisenstaedt, Romain Gervalle, David Hirondel,
Ted Jacobson, Michel Le Bellac, Alexandre Le Tiec,
Jean-Philippe Nicolas, Jordan Nicoules, Micaela Oertel,
Paul Ramond, Nicolas Seroux and Frédéric Vincent for spotting mistakes, correcting typos and making
nice suggestions in preliminary versions of the text.


\vspace{3ex}
These notes are released under the
\begin{center}
\href{https://creativecommons.org/licenses/by-nc-sa/4.0/}{{Creative Commons Attribution-NonCommercial-ShareAlike 4.0 International License}}\\[1ex]
\includegraphics[height=0.03\textheight]{cc_license.png}
\end{center}

