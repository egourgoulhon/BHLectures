\chapter{Black hole formation 2: Vaidya collapse}
\label{s:vai}

\minitoc

\section{Introduction}

Having investigated the gravitational collapse of a star, modeled as a ball of dust,
in the preceding chapter, we turn now to a less astrophysical scenario: the
formation of a black hole by the collapse of a shell of radiation,
known as \emph{Vaidya collapse}. Albeit quite
academic, this process illustrates various features of black hole birth and dynamics.

\section{The ingoing Vaidya metric}

\subsection{General expression}

Let us consider a spherically symmetric spacetime $(\M,\w{g})$ described by
coordinates $(v,r,\th,\ph)$ such that $v\in \R$, $r\in(0, +\infty)$,
$\th\in(0,\pi)$ and $\ph\in(0,2\pi)$, $(\th,\ph)$ being standard
spherical coordinates on $\SS^2$ and $r$ being the areal radius associated
with spherical symmetry (cf. Sec.~\ref{s:sch:static_spher}).
The \defin{ingoing Vaidya metric}\index{ingoing!Vaidya metric}\index{Vaidya!metric!ingoing --}
is the metric tensor
\be \label{e:vai:Vaidya_metric_null}
    \encadre{ \w{g} =
            -\left( 1 - \frac{2 M(v)}{r} \right)\, \dd v^2
            + 2 \, \dd v \, \dd r
        + r^2 \left( \dd\th^2 + \sin^2\th\, \dd\ph^2 \right) } ,
\ee
where $M(v)$ is a real-valued function of $v$ that is monotonically increasing.
We immediately notice that this expression strongly resembles that
of the Schwarzschild metric expressed in ingoing Eddington-Finkelstein (IEF)
coordinates, as given by Eq.~(\ref{e:sch:Schwarz_metric_NIEF}). Actually, the
only difference is the constant $m$ in Eq.~(\ref{e:sch:Schwarz_metric_NIEF})
replaced by the function $M(v)$ in Eq.~(\ref{e:vai:Vaidya_metric_null}).

An important property of the ingoing Vaidya metric is
\begin{greybox}
The vector field $\w{k}$, defined as the metric-dual\footnote{Cf. Sec.~\ref{s:bas:metric_dual};
in index notation: $k^\alpha := - g^{\alpha\mu} \partial_\mu v = - \nabla^\alpha v \iff
k_\alpha := - \partial_\alpha v $}
of the 1-form $-\dd v$:
\be
    \w{k} := - \vp{\dd v} = - \vw{\nabla} v \quad\iff\quad
    \uu{k} := - \dd v ,
\ee
is a null vector.
\end{greybox}
\begin{proof}
In view of the metric components (\ref{e:vai:Vaidya_metric_null}),
the coordinate vector field $\wpar_r$ is a null vector, since $g_{rr} = 0$.
Its metric-dual is the 1-form $\uu{\wpar_r} = g_{\mu r}\dd x^\mu = g_{vr} \dd v = \dd v$.
Hence we $\w{k} = - \wpar_{r}$, so that $\w{k}$ is a null vector.
\end{proof}

As shown in the notebook~\ref{s:sam:Vaidya},
the Ricci tensor of $\w{g}$ is very simple:
\be
    \w{R} = \frac{2 M'(v)}{r^2}\, \dd v \otimes \dd v ,
\ee
where $M'(v)$ stands for the derivative of the function $M(v)$.
The Ricci scalar $R = g^{\mu\nu} R_{\mu\nu} = (2M'(v)/r^2) \, \nabla_\mu v \nabla^\mu v =  (2M'(v)/r^2) \, k_\mu k^\mu$ is identically zero, since $\w{k}$ is a null vector.
It follows that
\begin{greybox}
The ingoing Vaidya metric (\ref{e:vai:Vaidya_metric_null}) is a solution
of the Einstein equation (\ref{e:fra:Einstein_eq})
with $\Lambda = 0$ and with the energy-momentum tensor
\be
   \encadre{ \w{T} = \frac{M'(v)}{4\pi r^2}\, \uu{k} \otimes \uu{k}  } .
\ee
\end{greybox}

\subsection{Case of a infalling shell of radiation}

\subsection{Black hole formation}

\section{Trapping horizon}
