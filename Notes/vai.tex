\chapter{Black hole formation 2: Vaidya collapse}
\label{s:vai}

\minitoc

\section{Introduction}

Having investigated the gravitational collapse of a star, modeled as a ball of dust,
in the preceding chapter, we turn now to a less astrophysical scenario: the
formation of a black hole by the collapse of a spherical shell of pure radiation
(no matter!),
known as \emph{Vaidya collapse}. Albeit quite
academic, this process illustrates various features of black hole birth and dynamics,
in a way somehow complementary to the collapse of a dust ball.

\section{The ingoing Vaidya metric}

\subsection{General expression} \label{s:vai:general}

Let us consider a spherically symmetric spacetime $(\M,\w{g})$ described by
coordinates $(v,r,\th,\ph)$ such that $v\in \R$, $r\in(0, +\infty)$,
$\th\in(0,\pi)$ and $\ph\in(0,2\pi)$, $(\th,\ph)$ being standard
spherical coordinates on $\SS^2$ and $r$ being the areal radius associated
with spherical symmetry (cf. Sec.~\ref{s:sch:static_spher}).
The \defin{ingoing Vaidya metric}\index{ingoing!Vaidya metric}\index{Vaidya!metric!ingoing --}
is the metric tensor
\be \label{e:vai:Vaidya_metric_null}
    \encadre{ \w{g} =
            -\left( 1 - \frac{2 M(v)}{r} \right)\, \dd v^2
            + 2 \, \dd v \, \dd r
        + r^2 \left( \dd\th^2 + \sin^2\th\, \dd\ph^2 \right) } ,
\ee
where $M(v)$ is a real-valued function of $v$.
We immediately notice that this expression strongly resembles that
of the Schwarzschild metric expressed in the
\emph{null ingoing Eddington-Finkelstein}\index{Eddington-Finkelstein!coordinates}\index{null!ingoing Eddington-Finkelstein coordinates}
coordinates, as given by Eq.~(\ref{e:sch:Schwarz_metric_NIEF}). Actually, the
only difference is the constant $m$ in Eq.~(\ref{e:sch:Schwarz_metric_NIEF})
replaced by the function $M(v)$ in Eq.~(\ref{e:vai:Vaidya_metric_null}).
We may say that the Schwarzschild metric is the special case $M(v) = m$ of
the ingoing Vaidya metric.

A key property of the ingoing Vaidya metric is
\begin{greybox}
The hypersurfaces $v = \mathrm{const}$ are null; a normal to them is the (null)
vector\footnote{in index notation: $k^\alpha := - g^{\alpha\mu} \partial_\mu v = - \nabla^\alpha v \iff
k_\alpha := - \partial_\alpha v $.}:
\be \label{e:vai:def_k}
    \w{k} := - \vp{\dd v} = - \vw{\nabla} v \quad\iff\quad
    \uu{k} := - \dd v .
\ee
Moreover, $\w{k}$ is equal to minus the vector $\wpar_r$ of coordinates
$(v,r,\th,\ph)$:
\be \label{e:vai:k_d_dr}
    \w{k} = - \left. \der{}{r} \right| _{v,\th,\ph} ,
\ee
where we have rewritten $\wpar_r$ as $\left. \dert{}{r} \right| _{v,\th,\ph}$
to distinguish it from the vector $\wpar_r$ of the IEF coordinates, to be
introduced in Sec.~\ref{s:vai:IEF}.
\end{greybox}
\begin{proof}
Let $\Sigma_v$ be a hypersurface defined by $v = \mathrm{const}$.
The metric induced by $\w{g}$ on $\Sigma_v$
is obtained by setting $\dd v = 0$ in Eq.~(\ref{e:vai:Vaidya_metric_null}):
$\left.\w{g}\right| _{\Sigma_v} = r^2 \left( \dd\th^2 + \sin^2\th\, \dd\ph^2 \right)$.
This metric has clearly the signature $(0, +, +)$, i.e. it is degenerate, hence
$\Sigma_v$ is a null hypersurface (cf. Sec.~\ref{s:def:hor_as_null}).
By construction, $\w{k}$ is normal to $\Sigma_v$. It is thus a null vector.
Besides, the metric dual of
the coordinate vector field $\wpar_r$ is the 1-form
$\uu{\wpar_r} = g_{\mu r}\dd x^\mu = g_{vr} \dd v = \dd v = - \uu{k}$,
which proves Eq.~(\ref{e:vai:k_d_dr}).
\end{proof}

Since $\dd v$ is nowhere vanishing, $\w{k}$ is a nonzero null vector field on $\M$.
We may use it to set up the \emph{time orientation} of $(\M,\w{g})$
by declaring that $\w{k}$ is future-oriented (cf. Sec.~\ref{s:fra:time_orientation}).

As shown in the notebook~\ref{s:sam:Vaidya},
the Ricci tensor of $\w{g}$ takes a simple form:
\be \label{e:vai:Ricci_tensor}
    \w{R} = \frac{2 M'(v)}{r^2}\, \dd v \otimes \dd v ,
\ee
where $M'(v)$ stands for the derivative of the function $M(v)$.
The Ricci scalar $R = g^{\mu\nu} R_{\mu\nu} = (2M'(v)/r^2) \, \nabla_\mu v \nabla^\mu v =  (2M'(v)/r^2) \, k_\mu k^\mu$ is identically zero, since $\w{k}$ is a null vector.
It follows that
\begin{greybox}
The ingoing Vaidya metric (\ref{e:vai:Vaidya_metric_null}) is a solution
of the Einstein equation (\ref{e:fra:Einstein_eq})
with $\Lambda = 0$ and with the energy-momentum tensor
\be \label{e:vai:ener_mom_tensor}
   \encadre{ \w{T} = \frac{M'(v)}{4\pi r^2}\, \uu{k} \otimes \uu{k}  } .
\ee
\end{greybox}

\begin{remark}
We have already noticed that Vaidya metric reduces to Schwarzschild metric for $M(v) = \mathrm{const}$.
This corresponds to $M'(v) = 0$, so that
Equation~(\ref{e:vai:ener_mom_tensor}) reduces to $\w{T} = 0$ and we recover that
Schwarzschild metric is a solution of the vacuum Einstein equation.
\end{remark}

The tensor (\ref{e:vai:ener_mom_tensor}) has the same
structure as the energy-momentum tensor of the dust model considered in Chap.~\ref{s:lem}:
$\w{T}_{\rm dust} = \rho  \, \uu{u} \otimes \uu{u} $ [Eq.~(\ref{e:lem:T_pressureless})].
The main difference is that $\w{u}$ is a timelike vector (the dust 4-velocity), while
$\w{k}$ is a null vector. For this reason, the energy-momentum tensor  (\ref{e:vai:ener_mom_tensor})
is sometimes referred to as a \defin{null dust}\index{null!dust}\index{dust!null --} model \cite{Poiss04}.
It corresponds physically to the energy-momentum tensor of some \emph{monochromatic electromagnetic radiation} in the
\emph{geometrical optics}\index{geometrical optics} approximation (see Box~22.4 of Ref.~\cite{MisneTW73}):
$\w{T}_{\rm rad} = q  \, \uu{K} \otimes \uu{K}$, where $q\geq 0$ and $\w{K} := \vw{\nabla}\Phi$ is the \emph{wave vector}\index{wave!vector}, $\Phi$ being the rapidly-varying
phase in the geometrical optics decomposition $\w{A} = \mathrm{Re}(\mathrm{e}^{\mathrm{i}\Phi} \, \w{a})$ of the electromagnetic potential 1-form $\w{A}$. The quantity $q$ is related to the energy density
$\veps$ of the electromagnetic field as measured by an observer $\Obs$ of 4-velocity $\w{u}$
by $\veps = \omega^2 q$, where $\omega = - \w{K}\cdot\w{u}$ is the frequency of the electromagnetic
radiation as measured by $\Obs$.
Maxwell equations\index{Maxwell equations} imply that $\w{K}$ is a null vector and that it is geodesic:
$\w{\nabla}_{\w{K}}\, \w{K} = 0$. The geodesic integral curves of $\w{K}$ are nothing but
the \emph{light rays}\index{light!ray} of the geometrical optics framework.
The geodesic character also holds for $\w{k}$ as defined by Eq.~(\ref{e:vai:def_k}),
since $\w{k}$ is normal to the null hypersurfaces $v = \mathrm{const}$:
the integral curves of $\w{k}$ are the null geodesic generators of these hypersurfaces
(cf. Sec.~\ref{s:def:null_geod_gen}); moreover,
we have exactly
\be \label{e:vai:k_geodesic}
    \w{\nabla}_{\w{k}}\, \w{k} = 0 .
\ee
This follows from Eq.~(\ref{e:def:wl_geod_kappa}) with $\wl = \w{k}$ and $\kappa = 0$
by virtue of Eq.~(\ref{e:def:def_kappa}) with $\rho=0$ implied by
Eq.~(\ref{e:def:wl_rho_u}) given that $u = v$ and $\w{k} =  - \vw{\nabla} v$ [Eq.~(\ref{e:vai:def_k})].
Equations~(\ref{e:vai:k_geodesic}) and (\ref{e:vai:k_d_dr})
imply that the curves $(v,\th,\ph) = \mathrm{const}$
are null geodesics, $\lambda:=-r$ is an affine parameter along them and $\w{k}$ is
the corresponding tangent vector.

Another condition for the identification of $\w{T}$ with $\w{T}_{\rm rad}$ is that the coefficient
in front of $\uu{k} \otimes \uu{k}$ in (\ref{e:vai:ener_mom_tensor})
is non-negative, since $q\geq 0$ in $\w{T}_{\rm rad}$. This constraint can also be seen as the \emph{null energy condition} (\ref{e:neh:null_energy_cond}) introduced in
Sec.~\ref{s:neh:NEH_Theta_zero}: for any null vector $\wl$, we have
$\w{T}(\wl,\wl) = M'(v)/(4\pi r^2)\, (\w{k}\cdot\wl)^2$ and hence $\w{T}(\wl,\wl) \geq 0$ $\iff$
$M'(v) \geq 0$. In other words, the function $M(v)$ must be monotically increasing. Then, we can set
$\w{k} = \alpha \w{K}$, where $\alpha$ is a constant, to have a perfect match of (\ref{e:vai:ener_mom_tensor}) with the electromagnetic radiation energy-momentum
tensor $\w{T}_{\rm rad}$.
To summarize:
\begin{greybox}
Provided that $M(v)$ is an increasing function\footnote{by \emph{increasing}, it is meant
strictly increasing ($M'(v)>0$) or locally constant ($M'(v) = 0$).},
the ingoing Vaidya spacetime $(\M,\w{g})$ is generated by a spherical symmetric electromagnetic radiation
within the geometrical optics approximation. The corresponding light rays are the
\defin{ingoing radial null geodesics}\index{ingoing!radial!null geodesics}
defined by $v=\mathrm{const}$, $\th=\mathrm{const}$ and $\ph=\mathrm{const}$.
These geodesics admit $\w{k}$ as the tangent vector associated with their affine parameter
$\lambda := -r$.
\end{greybox}

\subsection{Expression in ingoing Eddington-Finkelstein coordinates} \label{s:vai:IEF}

To deal with the black hole formation in Vaidya spacetime, it is quite
convenient to work with the
\defin{ingoing Eddington-Finkelstein (IEF) coordinates}\index{Eddington-Finkelstein!coordinates}\index{ingoing!Eddington-Finkelstein!coordinates}\index{IEF}
$(t,r,\th,\ph)$ defined from the null coordinates $(v,r,\th,\ph)$ in the
same way as for the Schwarzschild case, i.e. such that $v$ is the
\defin{advanced time}\index{advanced!time}\index{time!advanced --} with respect to $t$
[cf. Eq.~(\ref{e:sch:ti_v_r})]:
\be  \label{e:vai:t_v_r}
    \encadre{t := v - r} \iff \encadre{v = t + r} .
\ee
\begin{remark}
The coordinate $t$ was denoted by $\ti$ in Chap.~\ref{s:sch}, where $t$
was reserved for the Schwarzschild-Droste time coordinate.
\end{remark}
We have $\dd v = \dd t + \dd r$, from which the IEF expression of the
metric tensor is immediately deduced from Eq.~(\ref{e:vai:Vaidya_metric_null}):
\be \label{e:vai:metric_IEF}
    \encadre{
        \begin{array}{ll}
        \w{g} = &
           \displaystyle -\left( 1 - \frac{2 M(t+ r)}{r} \right)\, \dd t^2
            + \frac{4 M(t + r)}{r} \, \dd t \, \dd r
            + \left( 1 + \frac{2 M(t+r)}{r} \right)\, \dd r^2 \\[1ex]
       & + r^2 \left( \dd\th^2 + \sin^2\th\, \dd\ph^2 \right) .
       \end{array}
       }
\ee
The expression of $\w{k}$ in terms of the IEF coordinate frame
is deduced from Eq.~(\ref{e:vai:k_d_dr}) and the chain rule:
\be \label{e:vai:k_IEF}
    \w{k} = \wpar_t - \wpar_r .
\ee
\begin{remark}
The analog equation in Schwarzschild spacetime is Eq.~(\ref{e:sch:null_vector_k}).
\end{remark}

\subsection{Outgoing radial null geodesics}

Let us determine the radial null directions at each point by searching for
null vectors of the form $\wl = \wpar_t + V \wpar_r$. The condition
$\w{g}(\wl, \wl) = 0$ with the expression (\ref{e:vai:metric_IEF}) for
$\w{g}$ yields immediately a quadratic equation for $V$:
\[
    \left( 1 + \frac{2 M(t+r)}{r} \right) V^2
    + \frac{4 M(t + r)}{r}\, V +
    1 - \frac{2 M(t+ r)}{r}  = 0 .
\]
The two solutions are $V = -1$ and $V = [r - 2M(t + r)]/[r + 2M(t+r)]$.
The first solution gives back the ingoing null vector $\w{k}$ introduced in Sec.~\ref{s:vai:general}
[cf. Eq.~(\ref{e:vai:k_IEF})].
The null vector $\wl$ corresponding to the second solution is
\be
    \wl = \wpar_t + \frac{r - 2M(t + r)}{r + 2M(t+r)}\, \wpar_r .
\ee

\begin{greybox}
The integral curves of the vector fields $\w{k}$ and $\wl$ are null geodesics.
Those of $\w{k}$ are the \emph{ingoing radial null geodesics}\index{ingoing!radial!null geodesic}
already discussed in Sec.~\ref{s:vai:general},
while those of $\wl$ are called the \defin{outgoing radial null geodesics}\index{ingoing!null geodesic}.
\end{greybox}

\begin{proof}
A direct computation shows that $\wl$ is a pregeodesic vector field:
$\wnab_{\wl} \wl = \kappa \wl$, with $\kappa = - 4 [rM'(t+r) - M(t+r)]/[r + 2M(t+r)]^2$
(cf. the notebook~\ref{s:sam:Vaidya}). It follows that the integral curves of
$\wl$ are geodesics (cf. Sec.~\ref{s:geo:gener_param}).
\end{proof}
The differential equation governing the outgoing radial null geodesics
is obtained by demanding that $\wl$ is their tangent vector:
\be \label{e:vai:ODE_outgoing_null}
    \frac{\D r}{\D t} = \frac{r - 2M(t + r)}{r + 2M(t+r)} .
\ee

\begin{remark}
As in the Schwarzschild case (cf. Remark~\ref{r:sch:outgoing_ingoing}
on p.~\pageref{r:sch:outgoing_ingoing}),
the outgoing radial null geodesics are actually \emph{ingoing}, i.e. have $r$ decreasing towards the future,
as soon as $r < 2 M(t+r)$. This corresponds to the region bounded by the red curve in Fig.~\ref{f:vai:diag_S0}.
\end{remark}

%%%%%%%%%%%%%%%%%%%%%%%%%%%%%%%%%%%%%%%%%%%%%%%%%%%%%%%%%%%%%%%%%%%%%%%%%%%%%%%

\section{Infalling shell of radiation} \label{s:vai:infall}

\subsection{The infalling shell model} \label{s:vai:infalling_shell}

An \defin{infalling shell of radiation}\index{shell!infalling -- of radiation}\index{infalling!shell of radiation}
is defined by the ingoing Vadya metric with the function $M(v)$ obeying
$M'(v) \neq 0$ only on a finite interval of $v$. By choosing properly
the origin of $v$, we may consider this interval to be $[0, v_0]$, with
the parameter $v_0>0$ governing the
thickness of the shell.
The function $M(v)$ is thus constant outside the interval $[0, v_0]$.
In order to describe the formation of a black hole, we choose
$M(v) = 0$ for $v < 0$. This corresponds to a piece of Minkowski spacetime,
since the metric (\ref{e:vai:metric_IEF})
cleary reduces to Minkowski metric wherever $M(v)=0$ [compare Eq.~(\ref{e:glo:Mink_metric_spher})].
Denoting by $m>0$ the constant value of $M(v)$ for $v > v_0$, we have then
\be \label{e:vai:mass_function}
    M(v) = \left\{ \begin{array}{ll}
     0 \quad \mbox{for} \ v < 0 \qquad & \mbox{(Minkowski region)} \\
     m \, S(v/v_0) \quad \mbox{for} \ 0 \leq v \leq v_0
        & \mbox{(radiation region)} \\
      m \quad \mbox{for} \ v > v_0 \qquad & \mbox{(Schwarzschild region),}
      \end{array} \right.
\ee
where $S: [0,1] \to [0,1],\ x \mapsto S(x)$ is an increasing function
obeying $S(0) = 0$ and $S(1) = 1$.
The region for $v>v_0$ is qualified as \emph{Schwarzschild} since for $M(v) = m = \mathrm{const}$,
the Vaidya metric reduces to Schwarzschild metric, as noticed
in Sec.~\ref{s:vai:general}.
The three regions are shown in terms of coordinates $(t, r)$ on Fig.~\ref{f:vai:diag_S0}:
the radiation region (the infalling shell) is painted in yellow, the Minkowski
region lies below it and the Schwarzschild region lies above it.

\begin{figure}
\centerline{\includegraphics[width=0.6\textwidth]{vai_diag_S0.pdf}}
\caption[]{\label{f:vai:diag_S0} \footnotesize
Spacetime diagram of the Vaidya collapse based on the IEF coordinates $(t, r)$
and for the linear mass function $M(v)=m v/v_0$ with $v_0 = 3 m$.
The yellow area is the radiation region [cf. Eq.~(\ref{e:vai:mass_function})],
below it lies the Minkowski region and above it the Schwarzchild one.
The solid (resp. dashed) green curves are outgoing (resp. ingoing) radial
null geodesics. The thick black line marks the event horizon and
the red one the future outer trapping horizon. The curvature singularity
is indicated by the orange zigzag line. The part of the figure corresponding
to the Minkowski region can be compared with Fig.~\ref{f:glo:null_coord},
while that corresponding to the Schwarzschild region can be
compared with Fig.~\ref{f:sch:rad_null_geod_EF}.
\textsl{[Figure generated by the notebook \ref{s:sam:Vaidya}]}
}
\end{figure}


The simplest example of a function $M(v)$
obeying (\ref{e:vai:mass_function}) is obtained for $S(x) = x$:
\be \label{e:vai:S_linear}
     S(x) = x \quad \iff \quad M(v) = m \frac{v}{v_0} \quad (0 \leq v \leq v_0).
\ee
It is shown as the blue curve in Fig.~\ref{f:vai:mass_function}.
This choice of $S$ makes $M(v)$ piecewise linear. The resulting metric tensor
(\ref{e:vai:Vaidya_metric_null}) is continuous but not $C^1$ at $v=0$ and $v=v_0$. A choice
of $S$ that yields a $C^2$ metric tensor is
\be
    S(x) = 6 x^5 - 15 x^4 + 10 x^3 .
\ee
This choice is depicted by the red curve in Fig.~\ref{f:vai:mass_function}.

\begin{figure}
\centerline{\includegraphics[width=0.6\textwidth]{vai_mass_function.pdf}}
\caption[]{\label{f:vai:mass_function} \footnotesize
Function $M(v)$ for the infalling shell model, for $v_0 = 3 m$ and
two different choices of $S(x)$ in formula~(\ref{e:vai:mass_function}).
}
\end{figure}

\subsection{Solution for $M(v)$ piecewise linear} \label{s:vai:sol_M_linear}

Let us consider the simplest choice for $M(v)$, i.e.
Eq.~(\ref{e:vai:S_linear}).
In the radiation region, the metric tensor (\ref{e:vai:Vaidya_metric_null})
takes the form
\be \label{e:vai:self_similar_metric}
    \w{g} = -\left( 1 - \alpha \frac{v}{r} \right)\, \dd v^2
            + 2 \, \dd v \, \dd r
        + r^2 \left( \dd\th^2 + \sin^2\th\, \dd\ph^2 \right) \qquad
        (0 \leq v \leq v_0),
\ee
where $\alpha$ is the positive constant defined by
\be \label{e:vai:def_alpha}
   \encadre{ \alpha := \frac{2m}{v_0} }.
\ee
It is immediately apparent on (\ref{e:vai:self_similar_metric})
that for any $\lambda > 0$, the homothety $H_\lambda:\ (v, r) \mapsto (\lambda v, \lambda r)$
maps $\w{g}$ to $\lambda^2 \w{g}$. Hence $H_\lambda$ is a
conformal isometry\index{conformal!isometry}\index{isometry!conformal --} of
$\w{g}$ with a constant conformal factor $\lambda^2$. The homotheties $(H_\lambda)_{\lambda\in \R_{>0}}$
form a 1-dimensional
group, the generator of which is obtained by considering infinitesimal transformations,
i.e. homotheties of ratio $\lambda = 1 + \D\lambda$ where $\D\lambda$ is
infinitely small. The components of the corresponding displacement vector are $\D v = \D\lambda \, v$
and $\D r = \D\lambda\,  r$, so that formula~(\ref{e:neh:xi_dxdt}) (with $t \leftrightarrow \lambda$)
leads to the generator
\be \label{e:vai:hom_Killing}
    \w{\xi} = v \, \wpar_v + r \left. \wpar_r \right| _{v,\th,\ph}
            = t \, \wpar_t + r \, \wpar_r .
\ee
The second equality follows form the change of coordinates (\ref{e:vai:t_v_r}).
That $\w{\xi}$ has the same expression with respect to $(v, r)$ and $(t, r)$
coordinates should not be surprising since the homothety $H_\lambda$ has the
same expression in both coordinate systems:  $H_\lambda:\ (t, r) \mapsto (\lambda t, \lambda r)$,
given that $\lambda v = \lambda(t + r) = \lambda t + \lambda r$.
The vector field $\w{\xi}$ is called a
\defin{homothetic Killing vector}\index{homothetic!Killing!vector}\index{Killing!vector!homothetic --}.
The Lie derivative of $\w{g}$ along $\w{\xi}$ is twice $\w{g}$ (cf. the notebook~\ref{s:sam:Vaidya}
for the computation):
\be \label{e:vai:Lie_xi_g}
    \Lie{\xi} \w{g} = 2 \w{g} .
\ee

\begin{remark}
A \emph{homothetic Killing vector} is not a \emph{Killing vector},
for the right-hand side of Eq.~(\ref{e:vai:Lie_xi_g}) would be zero if
$\w{\xi}$ were a Killing vector [cf. Eq.~(\ref{e:neh:Lie_xi_g})].
In other words, except for $\lambda=1$, the homotheties $H_\lambda$ are not isometries,
but only conformal isometries.
Generally, vector fields generating conformal isometries are
called \defin{conformal Killing vectors}\index{conformal!Killing vector}.
They fulfill $\Lie{\xi} \w{g} = \sigma \w{g}$, where $\sigma$ is a scalar field.
Equation~(\ref{e:vai:Lie_xi_g}) constitutes the particular subcase $\sigma = 2$.
\end{remark}

Let us introduce the variable
\be \label{e:vai:def_x_v_r}
    \encadre{ x := \frac{v}{r} },
\ee
which is invariant under the homotheties $H_\lambda$.
The differential equation governing the outgoing radial null
geodesics, Eq.~(\ref{e:vai:ODE_outgoing_null}),
can be rewritten as $\D t / \D r = (1 + \alpha x)/(1 - \alpha x)$ [cf. Eq.~(\ref{e:vai:def_alpha})].
Given that $t = v - r = r(x - 1)$ implies $\D t / \D r = x - 1 + r \D x/\D r$,
we get the equivalent form
\be \label{e:vai:ODE_outgoing_x}
    r \frac{\D x}{\D r} = \frac{\alpha x^2 - x + 2}{1 - \alpha x} .
\ee
Fortunately, this ordinary differential equation is separable, so that its
solutions are easily obtained by quadrature. They depend on whether the
quadratic polynomial $P_\alpha(x) := \alpha x^2 - x + 2$ admits real roots
or not.
Let us first focus on the case where $P_\alpha$ has no real root.
The discriminant being $1 - 8\alpha$, this occurs if, and only if,
\be \label{e:vai:v0_small}
    \alpha > \frac{1}{8} \iff v_0 < 16 \, m .
\ee
Since $v_0$ is bounded from above, we may say that this case corresponds to
a \defin{thin radiation shell}\index{thin!radiation shell} or, equivalently,
to a \defin{large energy density of radiation}; the last denomination is
justified by considering the energy-momentum tensor (\ref{e:vai:ener_mom_tensor}) with
expression (\ref{e:vai:S_linear}) substituted for $M(v)$:
\be \label{e:vai:T_alpha}
\w{T} = \frac{\alpha}{8\pi r^2}\, \uu{k} \otimes \uu{k} ,
\ee
where $\alpha$ is bounded from below by Eq.~(\ref{e:vai:v0_small}).
We shall discuss the case of a thick/low energy density radiation shell
in Sec.~\ref{s:vai:naked_sing}.
For the moment, assuming (\ref{e:vai:v0_small}),
we have $P_\alpha(x) > 0$ for any $x\in\R$ and we may rewrite Eq.~(\ref{e:vai:ODE_outgoing_x})
as
\be \label{e:vai:ODE_outgoing_x_sep}
    \D \ln r = \frac{1 - \alpha x}{\alpha x^2 - x + 2}\, \D x ,
\ee
the solution of which is $r = r_0 f_\alpha(x)$, where
\be \label{e:vai:sol_r_x_v0_small}
 \encadre{   f_\alpha(x) := \frac{\sqrt{2}}{\sqrt{\alpha x^2 - x + 2}}
    \exp\left\{ \frac{1}{\sqrt{8\alpha - 1}} \left[
    \arctan \left(\frac{2\alpha x - 1}{\sqrt{8\alpha - 1}} \right)
    + \arctan\left( \frac{1}{\sqrt{8\alpha - 1}} \right)\right] \right\} },
\ee
and the integration constant $r_0>0$ is the value of $r$ at $x=0$, or equivalently
at $v = 0$.
Hence:
\begin{greybox}
In the radiation region, the outgoing radial null geodesics form
a 3-parameter family of curves $\left(\Li_{(r_0,\th,\ph)}\right)$,
where the parameter $r_0 \in \R^+$ is the value of $r$
at the inner edge of the radiation shell ($v=0$).
The parametric equation of $\Li_{(r_0,\th,\ph)}$ in terms of the IEF
coordinates is
\be \label{e:vai:eq_out_v0_small}
\begin{cases}
t = r_0 (x - 1) f_\alpha(x) \\
r = r_0 f_\alpha(x) \\
\th = \mathrm{const}, \ph = \mathrm{const}
\end{cases}
\qquad 0 \leq x \leq x_{\rm max},
\ee
where the function $f_\alpha(x)$ is defined by Eq.~(\ref{e:vai:sol_r_x_v0_small})
and either $x_{\rm max} = +\infty$ ($\Li_{(r_0,\th,\ph)}$ reaches $r=0$ for some $v < v_0$)
or $x_{\rm max}$ is the solution of $r_0 x_{\rm max} f_\alpha(x_{\rm max}) = v_0$
($\Li_{(r_0,\th,\ph)}$ reaches the outer edge of the radiation shell).
\end{greybox}
The function $f_\alpha(x)$ is plotted in Fig.~\ref{f:vai:f_alpha_x}.
It increases from $1$ at $x=0$ to some maximum reached for $x=1/\alpha$
and then decreases to $0$ as $x\to +\infty$. This behavior follows
directly from the sign of the numerator $1 - \alpha x$ in Eq.~(\ref{e:vai:ODE_outgoing_x_sep}),
given that the denominator $\alpha x^2 - x + 2$ is always positive for $\alpha > 1/8$.
Note that it could be that $x_{\rm max} < 1/\alpha$ so that the maximum
of $f_\alpha(x)$ is actually not reached along $\Li_{(r_0,\th,\ph)}$. In that
case, $r$ increases monotically along $\Li_{(r_0,\th,\ph)}$ in the radiation region,
from $r_0$ to $r_0 f_\alpha(x_{\rm max})$.

\begin{figure}
\centerline{\includegraphics[width=0.6\textwidth]{vai_f_alpha_x.pdf}}
\caption[]{\label{f:vai:f_alpha_x} \footnotesize
Function $f_\alpha(x)$, defined by Eq.~(\ref{e:vai:sol_r_x_v0_small}),
for some selected values of $\alpha > 1/8$. For each value of $\alpha$, the maximum
of $f_\alpha(x)$ is achieved for $x=1/\alpha$.
\textsl{[Figure generated by the notebook \ref{s:sam:Vaidya_solve_ode_out}]}
}
\end{figure}

It appears clearly on Eq.~(\ref{e:vai:eq_out_v0_small}) that
the homothety $H_\lambda:\ (t, r) \mapsto (\lambda t, \lambda r)$
transforms the geodesic $\Li_{(r_0,\th,\ph)}$ into the geodesic $\Li_{(\lambda r_0,\th,\ph)}$,
in agreement with the homothetic symmetry of the radiation region discussed
above.

Some outgoing radial null geodesics are depicted as solid green lines
in Fig.~\ref{f:vai:diag_S0}. In the radiation region, they obey
Eq.~(\ref{e:vai:eq_out_v0_small}).
Note that the homothetic symmetry appears clearly on the figure.
If a geodesic $\Li_{(r_0,\th,\ph)}$ has a turning point in terms of $r$ in the
radiation region, it must located at $x = 1/\alpha$, i.e. at $t/r = 1/\alpha - 1$,
or equivalently at
\be \label{e:vai:r_max_out}
  t = \left( \frac{v_0}{2m} - 1 \right) r .
\ee
The above equation defines a straight line through $(t,r) = (0,0)$,
whose intersection with the radiation region is depicted by a red segment
in Fig.~\ref{f:vai:diag_S0}.
\begin{remark}
The turning point value (\ref{e:vai:r_max_out}) can be obtained
directly by setting $\D r / \D t = 0$ in Eq.~(\ref{e:vai:ODE_outgoing_null})
and using the value (\ref{e:vai:S_linear}) for $M(v)$.
\end{remark}

In the Minkowski region, the outgoing radial null geodesics are straight line
segments inclined at $+45^\circ$ in Fig.~\ref{f:vai:diag_S0}, while
in the Schwarzschid region, they are curves obeying Eq.~(\ref{e:sch:outgoing_null_geod_EF})
(with the change of notation $\ti \leftrightarrow t$).


\subsection{Black hole formation} \label{s:vai:BH_formation}

In spherical symmetry, the inspection of radial null geodesics gives
a direct access to the black hole event horizon.
Let us determine the location of the latter for the homothetic
shell collapse considered above, since we have at disposal the exact solution
(\ref{e:vai:eq_out_v0_small}) for the outgoing radial null geodesics.
We arrive at the following result:
\begin{greybox}
The infalling radiation shell described by the Vaidya metric with $M(v)$ piecewise linear
and $v_0 < 16\,  m$ generates a black hole. The black hole event horizon $\Hor$
is the future light cone of the point of IEF coordinates
$(t, r) = (t_{\rm hb}, 0)$, with\footnote{As in Chap.~\ref{s:lem}, the subscript `hb' stands
for \emph{horizon birth}.}
\be \label{e:vai:t_hb}
    \encadre{ t_{\rm hb} = - 4 m \exp \left[ - \frac{2}{\sqrt{8\alpha - 1}}
    \arctan\left( \frac{1}{\sqrt{8\alpha - 1}} \right) \right] } ,
\ee
where $\alpha := 2m/v_0$ [Eq.~(\ref{e:vai:def_alpha})].
Outside the radiation shell, $\Hor$ coincides with the
Killing horizon of Schwarzschid spacetime located at $r=2m$.
\end{greybox}

\begin{figure}
\centerline{\includegraphics[width=0.6\textwidth]{vai_thb_v0.pdf}}
\caption[]{\label{f:vai:thb_v0} \footnotesize
Value $t_{\rm hb}$ of the coordinate $t$ at the black hole birth [Eq.~(\ref{e:vai:t_hb})]
as a function of the radiation shell thickness $v_0$, for
the homothetic Vaidya collapse [Eq.~(\ref{e:vai:S_linear})].
}
\end{figure}


\begin{proof}
In the Schwarzschild region, the Killing horizon is the hypersurface
$r = 2m$. It is generated by the null geodesics $\Li^{{\rm out},\Hor}_{(\th,\ph)}$
discussed in Sec.~\ref{s:sch:radial_null_IEF} [cf. Eq.~(\ref{e:sch:outgoing_null_geod_H})].
Let us consider one such geodesic, $\hat{\Li}$ say.
It has a fixed value of $(\th,\ph)$ and, when followed in the past direction,
it encounters the outer edge of the radiation region
(hypersurface $v=v_0$) at the point $A$ such that $r_A = 2m$ and $t_A = v_0 - 2m$
(cf. Fig.~\ref{f:vai:diag_S0}, where $\hat{\Li}$ can be identified with the
black curve). If $\hat{\Li}$ is prolonged into the radiation
region, still in the past direction, it encounters the inner edge of
the radiation region (hypersurface $v=0$) at the point $B$
(cf. Fig.~\ref{f:vai:diag_S0}).
The portion $AB$ of $\hat{\Li}$ coincides with the geodesic $\Li_{(r_B,\th,\ph)}$
of the outgoing radial null family given by Eq.~(\ref{e:vai:eq_out_v0_small}).
We have then $r_A = r_B f_\alpha(x_A)$.
Now, by definition of $x$ [Eq.~(\ref{e:vai:def_x_v_r})] and
$\alpha$ [Eq.~(\ref{e:vai:def_alpha})],
$x_A = v_A / r_A = v_0 / (2m) = 1/\alpha$.
We have thus $2 m = r_B f_\alpha(1/\alpha)$. In view of expression
(\ref{e:vai:sol_r_x_v0_small}) for $f_\alpha(x)$, there comes
\[
    r_B = 2 m \exp \left[ - \frac{2}{\sqrt{8\alpha - 1}}
    \arctan\left( \frac{1}{\sqrt{8\alpha - 1}} \right) \right] .
\]
Since $B$ is located on the hypersurface $v=0$, we have $t_B = - r_B$.
If the radial null geodesic $\hat{\Li}$
is prolonged further to the past in the Minkowksi
region, it becomes the straight line of equation $t = r - 2 r_B$.
$\hat{\Li}$ thus reaches $r=0$ at some event $C$ of
coordinate $t = t_{\rm hb} = -2 r_B$, hence Eq.~(\ref{e:vai:t_hb}).

There remains to prove that the null hypersuface generated by
$\hat{\Li}$ when $(\th,\ph)$ varies, i.e. the future light cone $\Hor$
of the point $(t, r) = (t_{\rm hb}, 0)$, is indeed the black hole event
horizon. To this aim, let us consider an outgoing radial null geodesic $\Li$
in the Minkowski region such that $\Li$ crosses $r=0$ at $t < t_{\rm hb}$,
i.e. outside $\Hor$.
$\Li$ arrives then at the inner edge of the radiation region
with $r = r_0 > r_B$.  In the radiation region,
according to Eq.~(\ref{e:vai:eq_out_v0_small}),
$\Li$ is homothetic to a part of $\hat{\Li}$ with a ratio $r_0 / r_B > 1$. It emerges
then at the outer edge of the radiation region with $r > r_A = 2 m$. It is
then in the exterior of the Schwarzschild black hole and can reach
the future null infinity $\scri^+$ of Schwarzschild spacetime.
On the contrary, if $\Li$ crosses $r=0$ with $t > t_{\rm hb}$,
i.e. inside $\Hor$,
it encounters the inner edge of the radiation region with
$r = r_0 < r_B$. A part of $\Li$ is then homothetic to the segment $BA$ of
$\hat{\Li}$ with a ratio $r_0 / r_B < 1$. Then either (i) $\Li$ has a
$r$-turning point and reaches $r=0$
in the radiation region or (ii) $\Li$ reaches
the outer edge of the radiation region ($v = v_0$) at some point $A'$.
Given that $x_A = 1/\alpha$ corresponds to the maximum of $f_\alpha(x)$, one
has necessarily $f_\alpha(x_{A'}) \leq f_\alpha(x_A)$ and thus
$r_0 f_\alpha(x_{A'}) < r_B f_\alpha(x_A)$. By
Eq.~(\ref{e:vai:eq_out_v0_small}), this implies $r_{A'} < r_A = 2m$, so
that $\Li$ emerges in the black hole region of Schwarzschild spacetime.
So none of the two possible cases (i) or (ii) leads to $\Li$ reaching
$\scri^+$. We conclude that $\Hor$ is a black hole
horizon.
\end{proof}

The black hole event horizon $\Hor$ is depicted as the thick black curve
in Fig.~\ref{f:vai:diag_S0}. Since $t_{\rm hb} < 0$ [cf. Eq.~(\ref{e:vai:t_hb})],
one immediately notes that the black hole forms in the Minkowski
region of Vaidya spacetime, i.e. in a region where the spacetime curvature
is zero! This striking feature reflects the non-local character of black holes and
will be discussed further in Chap.~\ref{s:loc}.

The dependency of $t_{\rm hb}$
on the width $v_0$ of the radiation shell is shown in Fig.~\ref{f:vai:thb_v0}.
One has $-4m < t_{\rm hb} < 0$, with
\be \label{e:vai:limit_thb_v0_0}
    \lim_{v_0 \to 0} t_{\rm hb} = - 4m \qand
     \lim_{v_0 \to 16m} t_{\rm hb} = 0 .
\ee
\begin{remark}
The first limit in (\ref{e:vai:limit_thb_v0_0}), which corresponds to $\alpha\to +\infty$ in Eq.~(\ref{e:vai:t_hb}),
is easily recovered by a pure geometric construction: a zero-width
shell implies the equality of the two points $A$ and $B$ considered in the
proof of (\ref{e:vai:t_hb}), as well as $t_A = t_B = - 2m$, hence
$t_C = t_{\rm hb} = - 4 m$.
\end{remark}


\subsection{Curvature singularity}

The Kretschmann scalar\index{Kretschmann scalar!of Vaidya metric}
$K := R_{\mu\nu\rho\sigma} R^{\mu\nu\rho\sigma}$
(cf. Sec.~\ref{s:sch:singularities})
is computed in the notebook~\ref{s:sam:Vaidya}:
\be \label{e:vai:Kretschmann}
    K = \frac{48 M(t + r)^2}{r^6} .
\ee
$K$ is identically zero in the Minkowski region ($M(r+t) = 0$), as it should!

It diverges at $r=0$ in the radiation and Schwarzschild regions ($M(r+t) > 0$), tracing the
existence of a curvature singularity there.
In other words:
\begin{greybox}
The Vaidya shell collapse introduced in Sec.~\ref{s:vai:infalling_shell}
generates a spacetime with a curvature singularity
located at $r=0$ and $t \geq 0$.
\end{greybox}

\begin{remark}
The Kretschmann scalar of Vaidya metric has the same structural form
as that of the Schwarzschild metric, compare Eq.~(\ref{e:sch:value_Kretschmann}),
while a priori $K$ could have
contained some term involving the derivative of $M(v)$. Indeed $M'(v)$
appears in some components of the Riemann tensor, since it is present in the
components of the Ricci tensor, as given by Eq.~(\ref{e:vai:Ricci_tensor}).
\end{remark}

\begin{remark}
Since the Ricci tensor of the Vaidya metric is not identically zero
[cf. Eq.~(\ref{e:vai:Ricci_tensor})], other curvature invariants that one might
have think of in order to track the curvature singularity are the Ricci scalar $R := g^{\mu\nu} R_{\mu\nu}$
and the Ricci ``squared'' $R_{\mu\nu} R^{\mu\nu}$. However, they are both identically zero
since $\w{k}$ is a null vector.
\end{remark}

The curvature singularity is depicted as the orange broken line in Fig.~\ref{f:vai:diag_S0}.
For the homothetic collapse with $v_0 < 16\, m$ considered in Secs.~\ref{s:vai:sol_M_linear} and
\ref{s:vai:BH_formation}, one has $t_{\rm hb} < 0$ [Eq.~(\ref{e:vai:t_hb})], so that
the curvature singularity is entirely located in the black hole region. It is therefore hidden
from a remote observer. In the next section, we are going to see that this is no longer the
case for $v_0 > 16 ùm$: the singularity is then naked.


\section{Cases with a naked singularity} \label{s:vai:naked_sing}

\subsection{The low radiation density case}

In Sec.~\ref{s:vai:infall}, we have focussed on a homothetic radiation shell
($M(v) = \alpha v / 2$) with a large energy density:
$\alpha > 1/8$ [Eq.~(\ref{e:vai:v0_small})].
Let us now discuss the opposite case\footnote{The marginal case $\alpha = 1/8$
will not be discussed here; it is actually qualitatively similar to the
case $\alpha < 1/8$ insofar as it leads to a naked singularity as well.}:
\be \label{e:vai:v0_large}
    \alpha < \frac{1}{8} \iff v_0 > 16 \, m .
\ee
In view of the form (\ref{e:vai:T_alpha}) of the energy momentum tensor, this
corresponds to a low energy density of the radiation field. When (\ref{e:vai:v0_large})
is fullfilled, the polynomial $P_\alpha(x) := \alpha x^2 - x + 2$, which appears in the
numerator of the ODE (\ref{e:vai:ODE_outgoing_x}) ruling outgoing radial null geodesics,
admits two real roots:
\be \label{e:vai:x1_x2}
    x_1 := \frac{1 - \sqrt{1 - 8\alpha}}{2\alpha}
    \qand
    x_2 := \frac{1 + \sqrt{1 - 8\alpha}}{2\alpha} .
\ee
Accordingly, Eq.~(\ref{e:vai:ODE_outgoing_x}) can be recast as
\be \label{e:vai:ODE_outgoing_x_naked}
    r \frac{\D x}{\D r} = \frac{(x - x_1)(x - x_2)}{x_1 + x_2 - x} .
\ee
This differential equation admits two special solutions, corresponding
to outgoing radial null geodesics with constant value of $x$:
\be
    x = x_1  \qand x = x_2 .
\ee
Let us denote by respectively
$\Li^{*1}_{(\th,\ph)}$ and $\Li^{*2}_{(\th,\ph)}$ these geodesics.
Since $x := v/r = 1 + t/r$, their equations in terms of the IEF
coordinates $(t, r, \th, \ph)$ and in the radiation region is simply
\be
    \Li^{*1}_{(\th,\ph)}: \quad t = (x_1 - 1)r
    \qand
    \Li^{*2}_{(\th,\ph)}: t = (x_2 - 1) r.
\ee
We note that, in the radiation region,
$\Li^{*1}_{(\th,\ph)}$ and $\Li^{*2}_{(\th,\ph)}$ are straight line segments
through the origin $(t, r) = (0,0)$. They are depicted
as the segments $OC$ and $OB$ respectively in Fig.~\ref{f:vai:diag_naked_S0}.
When $(\th,\ph)$ varies, $\Li^{*1}_{(\th,\ph)}$ and $\Li^{*2}_{(\th,\ph)}$
generate null hypersurfaces, $\Hor_1$ and $\Hor_2$ respectively, that are
\defin{homothetic Killing horizons}\index{homothetic!Killing!horizon}\index{Killing!horizon!homothetic --}
in the radiation region: the homothetic Killing vector $\w{\xi}$
[cf. Eq.~(\ref{e:vai:hom_Killing})] is normal to these null hypersurfaces. Indeed,
their equations being $t=(x_1 - 1)r$ and $t=(x_2 - 1)r$, $ \Li^{*1}_{(\th,\ph)}$ and
$ \Li^{*2}_{(\th,\ph)}$ are orbits of the
homothety group $(H_\lambda)_{\lambda\in \R_{>0}}$ discussed in Sec.~\ref{s:vai:sol_M_linear}.
The group generator $\w{\xi}$ is thus tangent to $\Li^{*1}_{(\th,\ph)}$ on $\Hor_1$
and to $\Li^{*2}_{(\th,\ph)}$ on $\Hor_2$. Consequently, $\w{\xi}$ is null there. Since the
only null direction in a null hypersurface is the normal direction, it follows
that  $\w{\xi}$  is normal to $\Hor_1$ and $\Hor_2$.

\begin{figure}
\centerline{\includegraphics[width=0.6\textwidth]{vai_diag_naked_S0.pdf}}
\caption[]{\label{f:vai:diag_naked_S0} \footnotesize
Spacetime diagram of the Vaidya collapse based on the IEF coordinates $(t, r)$
and for the linear mass function $M(v)=m v/v_0$ with $v_0 = 18\, m$,
which corresponds to $\alpha = 1/9$, $x_1 = 3$
and $x_2 = 6$.
The legend is the same as in Fig.~\ref{f:vai:diag_S0}, with in addition
the blue line marking the Cauchy horizon induced by the naked singularity
at $(t,r) = (0,0)$. In the radiation region, the straight line segments
$OB$ and $OC$ are the traces in the $(t,r)$ plane of the homothetic Killing horizons $x=x_2$ and $x=x_1$, the last one being a
part of the Cauchy horizon.
\textsl{[Figure generated by the notebook \ref{s:sam:Vaidya}]}
}
\end{figure}

Let us now consider \emph{generic} outgoing radial null geodesics, i.e.
geodesics with
$x\neq x_1$ and $x \neq x_2$; we may then rewrite Eq.~(\ref{e:vai:ODE_outgoing_x_naked}) as
\[
    \D \ln r = \frac{x_1 + x_2 - x}{(x - x_1)(x - x_2)} \, \D x .
\]
This equation is easily integrated to
\be \label{e:vai:sol_r_x_out_naked}
   \encadre{ r = c \, \frac{| x - x_2 | ^{x_1/(x_2 - x_1)}}{| x - x_1 | ^{x_2/(x_2 - x_1)}} },
\ee
where $c$ is constant along the considered geodesic.

\begin{example} \label{x:vai:sol_out_1o9}
For $\alpha=1/9$, one has $x_1 = 3$ and $x_2 = 6$ [cf. Eq.~(\ref{e:vai:x1_x2})]
and Eq.~(\ref{e:vai:sol_r_x_out_naked}) reduces to
$r = c |x - 6|/(x - 3)^2$. Such a function of $x$ is
plotted in Fig.~\ref{f:vai:r_x_naksing}.
\end{example}

\begin{figure}
\centerline{\includegraphics[width=0.6\textwidth]{vai_r_x_naksing.pdf}}
\caption[]{\label{f:vai:r_x_naksing} \footnotesize
The coordinate $r$ as a function of $x:=v/r$ along outgoing radial
null geodesics for the homothetic Vaidya collapse with
$\alpha = 1/9$ (which implies $x_1 = 3$ and $x_2 = 6$).
The graph of $r(x)$ admits a local
maximum for $x=1/\alpha$, i.e. $x=9$ in the present case, although this
is barely noticeable on the figure.
\textsl{[Figure generated by the notebook \ref{s:sam:Vaidya_solve_ode_out}]}
}
\end{figure}

Having determined the outgoing radial null geodesics, one can establish
the following results:
\begin{greybox}
The infalling radiation shell described by the Vaidya metric with $M(v)$ piecewise linear
and $v_0 > 16\,  m$ generates a black hole, the event horizon of which orginates
at $(t,r) = (0, 0)$, with a slope $t/r = x_2 - 1$. Moreover, the curvature singularity at $(t,r)=(0,0)$
is \defin{naked}\index{naked singularity}: it is connected to remote observers
by the outgoing radial null geodesics $\Li^{*1}_{(\th,\ph)}$, as well as
by the family of outgoing radial
null geodesics that cross the outer edge of the radiation shell ($v=v_0$) with
$x_1 < x < 1/\alpha$; all geodesics of this family, which contains
$\Li^{*2}_{(\th,\ph)}$, emanate from $(t,r)=(0,0)$ with a slope $t/r = x_2 - 1$.
\end{greybox}
\begin{proof}
As in Sec.~\ref{s:vai:BH_formation}, let us consider a null generator
$\hat{\Li}$ of the Killing horizon at $r=2m$ in the Schwarzschild region.
$\hat{\Li}$ intersects the outer edge $v=v_0$ of the radiation shell at a
point $A$ such that $r_A=2m$ and hence $x_A=1/\alpha$ (cf. Fig.~\ref{f:vai:diag_naked_S0}). When
prolonged to the past in the radiation region, i.e. to $x < 1/\alpha$,
$\hat{\Li}$ has $r$ decreasing (cf. the graph of $r(x)$ at the left of
$x=1/\alpha$ in Fig.~\ref{f:vai:r_x_naksing}), until $r=0$ is reached for
$x=x_2$. Since the Kretschmann scalar (\ref{e:vai:Kretschmann}) is
$K = 12 \alpha^2 x^2 / r^4$ for the homothetic model, we get $K\to \infty$
for $r\to 0$ when $x\to x_2 > 0$. We thus conclude that $\hat{\Li}$ hits
the curvature singularity at $x=x_2$ and cannot be extended further in the
past, contrary to the case $\alpha > 1/8$ dealt with in Sec.~\ref{s:vai:BH_formation}.

Let us now consider any outgoing radial null geodesic $\Li$ in the Schwarzschild
region that intersects the outer edge $v=v_0$ of the radiation shell with
$x$ such that $x_1 < x < 1/\alpha$, i.e. between the points $C$ and $A$
in Fig.~\ref{f:vai:diag_naked_S0}. $\Li$ has $r> 2m$ in the Schwarzschild
region, hence it reaches the future null infinity $\scri^+$. When $\Li$
is prolonged backward in the radiation region, it starts with
$r$ decaying since
Eq.~(\ref{e:vai:ODE_outgoing_null}) can be rewritten $\D r/\D t = (1 - \alpha x)/(1 + \alpha x)$
for the homothetic model, implying $\D r / \D t <0$. The entire past of $\Li$ in the
radiation region can then be read on Fig.~\ref{f:vai:r_x_naksing}: if one
follows the decaying $r$ direction either for $x_1 < x < x_2$ or $x_2 < x < 1/\alpha$,
one ends up to $r=0$ for $x=x_2$, as for $\hat{\Li}$. Hence the same conclusion holds:
$\Li$ hits the curvature singularity in the past direction at $x=x_2$.
Given that $\Li$ extends to $\scri^+$ in the future, we conclude that
the curvature singularity is naked.

Finally, let us consider an outgoing radial null geodesic $\Li$ in the Schwarzschild
region that intersects the outer edge $v=v_0$ of the radiation shell with
$x < x_1$, i.e. below the point $C$ in Fig.~\ref{f:vai:diag_naked_S0}. When $\Li$ is prolonged backward in the radiation region, $r$ decreases along it
and we read on the left part of Fig.~\ref{f:vai:r_x_naksing} that $r$ reaches
a finite nonzero value at $x=0$, i.e. at the inner edge of the radiation
shell. It can then be extended backward to the Minkowski region.
Since for $x\to x_1$, $r/r_*\to +\infty$, we conclude that the value of $r$
at $x=0$ can be made arbitraly small. This proves that the entire Minkowski
region can be connected to $\scri^+$ by this type of null geodesics. It follows
that the black hole event horizon is the null hypersurface generated by $\hat{\Li}$.
\end{proof}

\subsection{Analysis in double-null coordinate systems}

The above result contains something puzzling at first glance: for a fixed
value of $(\th,\ph)$, there are distinct radial null geodesics emanating
from the ``point'' $(t, r) = (0,0)$, namely all the geodesics that
cross the outer edge of the radiation shell with a slope $t/r = x - 1$
with $x\in [x_1, 1/\alpha)$. %]$
This appears clearly on Fig.~\ref{f:vai:r_x_naksing}.
It looks like there is an infinity of future light cones emanating
from a single spacetime point!
This state of affairs results actually from a bad behaviour of the coordinate
$(t, r)$ near $(0,0)$. To clarify this, let us introduce new coordinates
$(u, v, \th, \ph)$ on spacetime, such that $u$ is constant along the
outgoing radial null geodesics, $v$ being by construction constant along
the ingoing ones. Let us start by substituting $v/r$ for $x$ in
the geodesic equation (\ref{e:vai:sol_r_x_out_naked}); we get
\be \label{e:vai:v_r_out_naked}
    \frac{\left| {v}/{x_2} - r \right| ^{x_1/(x_2 - x_1)}}{\left| {v}/{x_1} - r \right| ^{x_2/(x_2 - x_1)}} = \tilde{c},
\ee
where $\tilde{c} := c^{-1} x_1^{x_2/(x_2 - x_1)} x_2^{-x_1/(x_2 - x_1)}$ is
constant along a given outgoing radial null geodesic.
To proceed, we introduce two overlaping subregions of the radiation
region:
\be
    \mathscr{N}_{\rm I}: r > \frac{v}{x_2} \qand
    \mathscr{N}_{\rm II}: r < \frac{v}{x_1} .
\ee
Note that $\mathscr{N}_{\rm I}$ (resp.  $\mathscr{N}_{\rm II}$)
contains the homothetic Killing horizon $\Hor_1$ (resp. $\Hor_2$).

\subsubsection{The $\mathscr{N}_{\rm I}$ region}

On $\mathscr{N}_{\rm I}$, let us define the parameter $u$ so that
\be
    \tilde{c} = |u|^{-x_2/(x_2 - x_1)} .
\ee
As $\tilde{c}$, $u$ is constant along a given outgoing radial null geodesic.
From Eq.~(\ref{e:vai:v_r_out_naked}), we get\footnote{We have made use of
the identity $|v/x_2 - r| = r - v/x_2$, which holds on $ \mathscr{N}_{\rm I}$.}
$|u| = |v/x_1 - r| / (r - v/x_2)^{x_1/x_2}$. This equation determines
$u$ in terms of $v$ and $r$ up to some sign. We choose the latter so that
$u \to +\infty$ at the inner boundary of $\mathscr{N}_{\rm I}$
($r \to v/x_2$). This yields
\be \label{e:vai:u_N_I}
    u = \frac{v/x_1 - r}{(r - v/x_2)^{x_1/x_2}}, \quad -\infty < u < +\infty ,
\ee
with $\lim_{r\to +\infty} u = - \infty$.
We may use $(u, v, \th, \ph)$ as a coordinate system on $\mathscr{N}_{\rm I}$,
instead of IEF coordinates $(t, r, \th, \ph)$.
The metric components in the coordinates $(u, v, \th, \ph)$
are deduced from those in the $(v, r, \th, \ph)$
coordinates, i.e. Eq.~(\ref{e:vai:self_similar_metric}), via
the identity $\alpha=2/(x_1 x_2)$. We get
(see the notebook~\ref{s:sam:Vaidya_nk_sing} for the computation):
\be \label{e:vai:metric_N_I}
    \w{g} = - \frac{2 x_2}{(x_2 - x_1)r} \left(r - \frac{v}{x_2} \right)^{x_1/2}
        \dd u \, \dd v
        + r^2 \left( \dd\th^2 + \sin^2\th\, \dd\ph^2 \right) \qquad
        \left(0 \leq v \leq v_0,\ r > \frac{v}{x_2}\right) .
\ee
In this expression, $r$ shall be considered as a function of $(u,v)$ defined
implicitely by Eq.~(\ref{e:vai:u_N_I}).
The metric component $g_{uv}$ is regular and non-vanishing in all $\mathscr{N}_{\rm I}$.
Moreover we have a double-null coordinate system:
$g_{uu} = 0$ and $g_{vv} = 0$ imply that the coordinate vectors $\wpar_u$
and $\wpar_v$ are both null. The vector $\wpar_u$ is actually tangent to the
ingoing radial null geodesics, which are defined by $(v,\th,\ph) = \mathrm{const}$,
and the vector $\wpar_v$ is tangent to the outgoing ones, which are defined
by $(u,\th,\ph) = \mathrm{const}$.

\begin{example} \label{x:vai:u_r_1o9}
As in Example~\ref{x:vai:sol_out_1o9}, let us consider the case $\alpha=1/9$,
for which $x_1 = 3$ and $x_2 = 6$ [cf. Eq.~(\ref{e:vai:x1_x2})]. Equation~(\ref{e:vai:u_N_I})
reduces then to
\be
    u = \frac{v/3 - r}{\sqrt{r - v/6}} .
\ee
This relation can be inverted, yielding an explicit expression for $r(u, v)$:
\be \label{e:vai:r_uv_1o9}
    r = \frac{v}{3} + \frac{u}{2} \left( u - \sqrt{ u^2 + \frac{2}{3} v} \right) .
\ee
\end{example}

Let us investigate the structure of the inner boundary $v=0$ of the radiation
shell in the double-null coordinates $(u, v, \th, \ph)$. For $r\neq 0$,
the limit $v\to 0$ in Eq.~(\ref{e:vai:u_N_I}) leads to $u = - r^{1 - x_1/x_2}$.
This implies $u < 0$. In this part, which is the future boundary of the Minkowski
region of Vaidya spacetime, this relation defines a bijection between
$r\in(0,+\infty)$ and $u\in(0, -\infty)$. So we may say that $r$ is a regular
coordinate of $\mathscr{N}_{\rm I}$ for $u < 0$. On the other side,
the limit $r\to 0$ in $\mathscr{N}_{\rm I}$ implies $v\to 0$ for $r > v /x_2$
in $\mathscr{N}_{\rm I}$. Actually if $u$ has to remain finite in this limit,
Eq.~(\ref{e:vai:u_N_I}) implies the following behaviour
\be \label{e:vai:r_v_0_N_I}
    r \underset{v \to 0}{\sim} \frac{v}{x_2} .
\ee
This, in turns, implies that the numerator of Eq.~(\ref{e:vai:u_N_I})
is equivalent to $(x_2 - x_1)/(x_1 x_2) v$ in the limit $v\to 0$ and thus
is positive, so that $u > 0$ on the part of the boundary of $\mathscr{N}_{\rm I}$
where $r\to 0$.
Hence the ``point'' $(t,r) = (v,r) = (0,0)$ in IEF coordinates
becomes the hypersurface $v = 0$, $u > 0$ in the double-null coordinates. This means
that the IEF coordinates are not adapted to describe the vicinity of $(t,r) = (0,0)$
in Vaidya spacetime when $\alpha < 1/8$.

\begin{example}
Let us perform a first order expansion in $v$ of the explicit expression of $r(u,v)$
found in Example~\ref{x:vai:u_r_1o9} [Eq.~(\ref{e:vai:r_uv_1o9})].
Taking into account the identity $\sqrt{u^2} = |u|$, we get
\be
\begin{cases}
 \displaystyle r = u^2 + \frac{1}{3} v + O(v^2) & \quad\mbox{for}\ u < 0 \\
 \displaystyle r = \frac{1}{6} v + O(v^2) & \quad\mbox{for}\ u > 0 .
\end{cases}
\ee
The first equation agrees with the relation $u = - r^{1 - x_1/x_2}$ found
above for $r \neq 0 $ and $v = 0$, given that $x_1/x_2 = 1/2$ in the present
case, while the
last equation is exactly (\ref{e:vai:r_v_0_N_I}), given that $x_2=6$.
\end{example}

As discussed above, the double null coordinates $(u, v, \th, \ph)$
are regular coordinates on $\mathscr{N}_{\rm I}$, for the components
(\ref{e:vai:metric_N_I}) of the metric tensor are regular. Another advantage
of these coordinates is that they allow for an immediate drawing of a
Carter-Penrose diagram. It suffices to set $U = \arctan u$, $V = v$ and
$T = U + V$ and $X = V - U$ to get such a diagram of $\mathscr{N}_{\rm I}$.
It is depicted in Fig.~??

\subsubsection{The $\mathscr{N}_{\rm II}$ region}

On $\mathscr{N}_{\rm II}$, let us introduce the parameter $u'$ to rewrite
$\tilde{c}$ in Eq.~(\ref{e:vai:v_r_out_naked}) as
\be
    \tilde{c} = |u'|^{x_1/(x_2 - x_1)} .
\ee
Then Eq.~(\ref{e:vai:v_r_out_naked}) yields $|u'| = |v/x_2 - r| / (v/x_1 - r)^{x_2/x_1}$.
Choosing the sign of $u'$ such that $u'\to -\infty$ at the outer boundary of
$\mathscr{N}_{\rm II}$ ($r\to v/x_1$), we get
\be \label{e:vai:up_N_II}
    u' = \frac{v/x_2 - r}{\left(v/x_1 - r \right)^{x_2/x_1}} .
\ee
Let us consider $(u',v,\th,\ph)$ as a coordinate system on $\mathscr{N}_{\rm II}$.
As above, the metric components in these coordinates
are deduced from those in the $(v,r,\th,\ph)$ coordinates,
as given by Eq.~(\ref{e:vai:self_similar_metric})
(see the notebook~\ref{s:sam:Vaidya_nk_sing} for the computation):
\be
    \w{g} = - \frac{2 x_1}{(x_2 - x_1)r} \left(\frac{v}{x_1} - r \right)^{x_2/2}
        \dd u' \, \dd v
        + r^2 \left( \dd\th^2 + \sin^2\th\, \dd\ph^2 \right) \qquad
        \left(0 \leq v \leq v_0,\ r < \frac{v}{x_1}\right) ,
\ee
where $r = r(u', v)$ is defined implicitely by Eq.~(\ref{e:vai:up_N_II}).
Again, we get a double-null coordinate system, adapted to the ingoing and
outgoing radial null geodesics. Moreover, the coordinates $(u',v,\th,\ph)$ are regular
since $g_{u'v}$ is finite and non-vanishing in all $\mathscr{N}_{\rm II}$.

\begin{example}
For $\alpha=1/9$, i.e. $x_1 = 3$ and $x_2 = 6$,
Eq.~(\ref{e:vai:up_N_II}) reduces to
\be
    u' = \frac{3}{2} \frac{v - 6r}{(v - 3 r)^2} .
\ee
This relation can be inverted, yielding  an explicit expression for $r(u', v)$:
\be
    r = \frac{v}{3} + \frac{1}{2u'} \left( \sqrt{1 - \frac{2}{3} u' v } - 1 \right) .
\ee
\end{example}



The homothetic Killing horizon $\Hor_1 \subset \mathscr{N}_{\rm I}$,
which is defined by $v/r = x_1$, is the hypersurface $u = 0$,
while the homothetic Killing horizon $\Hor_2 \subset \mathscr{N}_{\rm II}$,
which is defined by $v/r = x_2$, is the hypersurface $u' = 0$.


%%%%%%%%%%%%%%%%%%%%%%%%%%%%%%%%%%%%%%%%%%%%%%%%%%%%%%%%%%%%%%%%%%%%%%%%%%%%%%%

\section{Trapping horizon}






%\section{Going further}

%\cite{Krish14}









