\chapter{Black hole formation 2: Vaidya collapse}
\label{s:vai}

\minitoc

\section{Introduction}

Having investigated the gravitational collapse of a star, modeled as a ball of dust,
in the preceding chapter, we turn now to a less astrophysical scenario: the
formation of a black hole by the collapse of a shell of radiation,
known as \emph{Vaidya collapse}. Albeit quite
academic, this process illustrates various features of black hole birth and dynamics.

\section{The Vaidya collapse solution}

\subsection{General expression}

Let us consider a spherically symmetric spacetime $(\M,\w{g})$ described by
coordinates $(v,r,\th,\ph)$ such that $v\in \R$, $r\in(0, +\infty)$,
$\th\in(0,\pi)$ and $\ph\in(0,2\pi)$, $(\th,\ph)$ being standard
spherical coordinates on $\SS^2$ and $r$ being the areal radius associated
with spherical symmetry (cf. Sec.~\ref{s:sch:static_spher}).
The \defin{Vaidya metric} is the metric tensor
\be \label{e:vai:Vaidya_metric_null}
    \encadre{ \w{g} =
            -\left( 1 - \frac{2 M(v)}{r} \right)\, \dd v^2
            + 2 \, \dd v \, \dd r
        + r^2 \left( \dd\th^2 + \sin^2\th\, \dd\ph^2 \right) } ,
\ee
where $M(v)$ is a real-valued function of $v$ that is monotonically increasing.
We immediately notice that this expression strongly resembles that
of the Schwarzschild metric expressed in ingoing Eddington-Finkelstein (IEF)
coordinates, as given by Eq.~(\ref{e:sch:Schwarz_metric_NIEF}). Actually, the
only difference is the constant $m$ in Eq.~(\ref{e:sch:Schwarz_metric_NIEF})
replaced by the function $M(v)$ in Eq.~(\ref{e:vai:Vaidya_metric_null}).

As shown in the notebook~??, the Ricci tensor of $\w{g}$ is very simple:
\be
    \w{R} = \frac{2 M'(v)}{r^2}\, \dd v \otimes \dd v .
\ee

\subsection{Case of a shell of radiation}

\subsection{Black hole formation}

\section{Trapping horizon}
